Consider a buggy input program $P$.
%
As discussed in Section~\ref{sec:c11}, in a valid \cc execution 
sequence of $P$ (including a buggy execution or a counter example), 
the \sc ordered events must form a total order ($\setTO$).
%
Contrarily, if we synthesize \sc ordered fences in the input program 
such that the total order requirement in the buggy sequence gets violated 
then we can invalidate the sequence and stop the program behavior.
This concept forms the base of our synthesis technique.

We introduce an irreflexive and possibly cyclic relation
on \sc ordered events called \sc-order ($\setSO$).
%
To construct the $\setSO$ order, we introduce the following \lso-rules
(diagrammatically shown in Figure~\ref{fig:so rules}).
The \lso-construction-rules are an interpretation or simply a derivative 
of the \lto-rules, \hlref{toHbMo}, \hlref{toFr} and \hlref{toRf} 
(Section~\ref{sec:c11}).
%formed by direct derivative and dual of the rules .

\begin{figure}[t]
	\begin{tabular}{|c||c|c|c|}
		\hline
%		\resizebox{0.19\textwidth}{!}{\tikzset{every picture/.style={line width=0.75pt}} %set default line width to 0.75pt        
\begin{tikzpicture}[x=1em,y=1em,yscale=-1,xscale=-1]
\tikzstyle{every node}=[font=\normalfont]
\node (wr1) {$ wr^\sc_1 $};
\node (wr2) [below=20pt of wr1] {$ wr^\sc_2 $};
\node (so1) [below=-5pt of wr2] {\hlref{sohb},\hlref{somo},};
\node (so2) [below=-5pt of so1] {\hlref{sorf}};

\draw [->,>=stealth,color=RedOrange,thin] ($ (wr2.south east)+(.3,-7pt) $) to[out=-135,in=135] node[above left=0pt and -5pt,color=black,-]{/} node[right=-2pt,pos=.5,font=\scriptsize] {\textcolor{black}{\lmo/}} node[right=-0.5pt,pos=.25,font=\scriptsize] {\textcolor{black}{\lhb/\lrf}} ($ (wr1.south east)+(0.4,-5pt) $);
\draw [->,>=stealth,color=Mahogany,thin] ($ (wr1.south west)+(-.3,-5pt) $) to[out=45,in=-45] node[midway,left=-2pt,font=\scriptsize] {\textcolor{black}{\lso}} ($ (wr2.south west)+(-0.3,-7pt) $);




%\node (nodeB) {$ b: r_1 \assign x  $};
%\node (sigB1) [below left=-3pt and -5pt of nodeB, color=green] {$\sigma_{b11} = $};
%\node (poB1empty) [right=-1pt of sigB1, color=green] {$ $};
%\node (poB1) [draw,fit=(poB1empty), color=orange, thin, inner sep=0pt] {};
%%\node (poB1n1) [right=-3pt of sigB1, color=green] {$ $};
%%\node (poB1) [draw,fit=(poB1n1), color=orange, thin, inner sep=0pt] {};
%\node (B1Val) [right=-3pt of poB1, color=green] {$,0$};
%\node (sigB1State) [fit=(poB1)(B1Val)(sigB1), inner sep=-1pt] {}; 

%\node (sigB2) [right=-3pt of sigB1State, color=green] {$\sigma_{b22} = $};
%\node (poB2N1) [right=6pt of sigB2, circle,fill=black,inner sep=0pt,minimum size=3pt] {};
%\node (poB2a) [left=1pt of poB2N1, inner sep=1pt] {$a$};
%\node (poB2) [draw,fit=(poB2N1)(poB2a), color=orange, thin, inner sep=1pt] {};
%\node (B2Val) [right=-3pt of poB2, color=green] {$,1$};
%\node (sigB2State) [fit=(poB2)(B2Val)(sigB2), inner sep=0pt] {}; 
%
%
%
%%\node (rhoB1) [below =10pt of sigB1State, color=red] {$\rho_{b1}:=0$};
%%\node (rhoB2) [below =10pt of sigB2State, color=red] {$\rho_{b2}:=1$};
%
%
%\node (nodeC) [below =2 of nodeB] {$c: x \assign 2$};
%\node (sigC1) [below left=3pt and -4pt of nodeC, color=green] {$\sigma_{c11} = $};
%\node (poC1N1) [right=6pt of sigC1, circle,fill=black,inner sep=0pt,minimum size=3pt] {};
%\node (poC1c) [left=1pt of poC1N1, inner sep=1pt] {$c$};
%\node (poC1) [draw,fit=(poC1c)(poC1N1), color=orange, thin, inner sep=1pt] {};
%\node (C1Val) [right=-3pt of poC1, color=green] {$,2$};
%\node (sigC1State) [fit=(poC1)(C1Val)(sigC1), inner sep=0pt] {}; 
%
%\node (sigC2) [right=-3pt of sigC1State, color=green] {$\sigma_{c22} = $};
%\node (poC2n1) [above right=0pt and 5pt of sigC2, circle,fill=black,inner sep=0pt,minimum size=3pt] {};
%\node (poC2a) [left=1pt of poC2n1, inner sep=0pt] {$a$};
%\node (poC2n2) [below=9pt of poC2n1, circle,fill=black,inner sep=0pt,minimum size=3pt] {};
%\node (poC2c) [left=1pt of poC2n2, inner sep=0pt] {$c$};
%\draw [color=purple] (poC2n1) -- (poC2n2);
%\node (poC2) [draw,fit=(poC2n1)(poC2a)(poC2n2)(poC2c), color=orange, thin, inner sep=1pt] {};
%\node (C2Val) [right=-3pt of poC2, color=green] {$,2$};
%\node (sigC2State) [fit=(poC2)(C2Val)(sigC2), inner sep=0pt] {}; 
%
%
%
%%\node (rhoC) [below=30pt of nodeC, color=red] {$\rho_{c}=2$};
%
%\node (nodeA) [left =3.5 of nodeB] {$a: x \assign 1$};
%\node (sigA) [below left=-2pt and -25pt of nodeA, color=green] {$\sigma_{a11} = $};
%\node (poAN1) [right=6pt of sigA, circle,fill=black,inner sep=0pt,minimum size=3pt] {};
%\node (poAa) [left=1pt of poAN1, inner sep=1pt] {$a$};
%\node (poA) [draw,fit=(poAN1)(poAa), color=orange, thin, inner sep=1pt] {};
%\node (AVal) [right=-3pt of poA, color=green] {$,1$};
%\node (sigAState) [fit=(poA)(AVal)(sigA), inner sep=-1pt] {}; 
%
%%\node (rhoA) [below=1pt of sigAState, color=red] {$\rho_{a} := 1$};
%
%
%\node (nodeD) [right =5.5 of nodeB] {$ d: r_2 \assign x  $};
%\node (sigD1) [below left=3pt and -5pt of nodeD, color=green] {$\sigma_{d22} = $};
%\node (poD1N1) [right=6pt of sigD1, circle,fill=black,inner sep=0pt,minimum size=3pt] {};
%\node (poD1c) [left=1pt of poD1N1, inner sep=1pt] {$c$};
%\node (poD1) [draw,fit=(poD1c)(poD1N1), color=orange, thin, inner sep=1pt] {};
%\node (D1Val) [right=-3pt of poD1, color=green] {$,2$};
%\node (sigD1State) [fit=(poD1)(D1Val)(sigD1), inner sep=0pt] {}; 
%
%\node (sigD2) [right=-3pt of sigD1State, color=green] {$\sigma_{d33} = $};
%\node (poD2n1) [above right=0pt and 5pt of sigD2, circle,fill=black,inner sep=0pt,minimum size=3pt] {};
%\node (poD2a) [left=1pt of poD2n1, inner sep=0pt] {$a$};
%\node (poD2n2) [below=9pt of poD2n1, circle,fill=black,inner sep=0pt,minimum size=3pt] {};
%\node (poD2c) [left=1pt of poD2n2, inner sep=0pt] {$c$};
%\draw [color=purple] (poD2n1) -- (poD2n2);
%\node (poD2) [draw,fit=(poD2n1)(poD2a)(poD2n2)(poD2c), color=orange, thin, inner sep=1pt] {};
%\node (D2Val) [right=-3pt of poD2, color=green] {$,2$};
%\node (sigD2State) [fit=(poD2)(D2Val)(sigD2), inner sep=0pt] {};  
%
%
%
%\node (nodeE) [below =2 of nodeD] {$e: r_3 \assign x$};
%\node (sigE1) [below left=3pt and -5pt of nodeE, color=green] {$\sigma_{e23} = $};
%\node (poE1N1) [right=6pt of sigE1, circle,fill=black,inner sep=0pt,minimum size=3pt] {};
%\node (poE1c) [left=1pt of poE1N1, inner sep=1pt] {$c$};
%\node (poE1) [draw,fit=(poE1c)(poE1N1), color=orange, thin, inner sep=1pt] {};
%\node (E1Val) [right=-3pt of poE1, color=green] {$,2$};
%\node (sigE1State) [fit=(poE1)(E1Val)(sigE1), inner sep=0pt] {}; 
%
%\node (sigE2) [right=-3pt of sigE1State, color=green] {$\sigma_{e34} = $};
%\node (poE2n1) [above right=0pt and 5pt of sigE2, circle,fill=black,inner sep=0pt,minimum size=3pt] {};
%\node (poE2a) [left=1pt of poE2n1, inner sep=0pt] {$a$};
%\node (poE2n2) [below=9pt of poE2n1, circle,fill=black,inner sep=0pt,minimum size=3pt] {};
%\node (poE2c) [left=1pt of poE2n2, inner sep=0pt] {$c$};
%\draw [color=purple] (poE2n1) -- (poE2n2);
%\node (poE2) [draw,fit=(poE2n1)(poE2a)(poE2n2)(poE2c), color=orange, thin, inner sep=1pt] {};
%\node (E2Val) [right=-3pt of poE2, color=green] {$,2$};
%\node (sigE2State) [fit=(poE2)(E2Val)(sigE2), inner sep=0pt] {}; 
%
%\node (sigE3) [below =9pt of sigE1, color=green] {$\sigma_{e32} = $};
%\node (poE3n1) [above right=0pt and 5pt of sigE3, circle,fill=black,inner sep=0pt,minimum size=3pt] {};
%\node (poE3c) [left=1pt of poE3n1, inner sep=0pt] {$c$};
%\node (poE3n2) [below=9pt of poE3n1, circle,fill=black,inner sep=0pt,minimum size=3pt] {};
%\node (poE3a) [left=1pt of poE3n2, inner sep=0pt] {$a$};
%\draw [color=purple] (poE3n1) -- (poE3n2);
%\node (poE3) [draw,fit=(poE3n1)(poE3a)(poE3n2)(poE3c), color=orange, thin, inner sep=1pt] {};
%\node (E3Val) [right=-3pt of poE3, color=green] {$,1$};
%\node (sigE3State) [fit=(poE3)(E3Val)(sigE3), inner sep=0pt] {};
%
%\node (sigE4) [right =-11pt of sigE3State, color=green] {$\quad \sigma_{e35} = \bot$};
%\node (sigE4State) [fit=(sigE4), inner sep=1pt] {};
%
%
%
%
%%\draw [dashed,->,>=stealth,color=brown,thin] (sigB1State.west) to[out=45,in=-45] (sigC1State.west);
%%\draw [dashed,->,>=stealth,color=brown,thin] (sigB2State.west) to[out=45,in=-45] (sigC2State.west);
%%\draw [dashed,->,>=stealth,color=brown,thin] (sigD1State.west) to[out=45,in=-45] (sigE1State.west);
%%\draw [dashed,->,>=stealth,color=brown,thin] (sigD2State.west) to[out=45,in=-45] (sigE2State.west);
%%\draw [dashed,->,>=stealth,color=brown,thin] (sigD1State.east) to[out=135,in=-135] (sigE3State.east);
%%\draw [dashed,->,>=stealth,color=brown,thin] (sigD2State.east) to[out=135,in=-135] (sigE4State.east);
%%\draw [dashed,->,>=stealth,color=blue,thin] (sigC1State.north east) to[out=-110,in=0] node[midway,left] {rf} (sigD1State.west);
%%\draw [dashed,->,>=stealth,color=blue,thin] (sigC2State.east) to[out=-150,in=0] node[midway,left] {rf} (sigD2State.west);
%
%\draw [dashed,->,>=stealth,color=blue,thin] (sigAState) to[out=165,in=25] node[midway,above] {rf} (sigB2State.south);
%\draw [dashed,->,>=stealth,color=blue,thin] (sigC1State.north east) -- node[midway,above] {rf} (sigD1State.south west);
%\draw [dashed,->,>=stealth,color=blue,thin] (sigC2State.north east) -- node[midway,above] {rf} (sigD2State.south west);
%\draw [dashed,->,>=stealth,color=blue,thin] (sigAState) to[out=105,in=15] node[midway,above] { rf} (sigE3State.west);
%\draw [dashed,->,>=stealth,color=blue,thin] (sigAState) to[out=100,in=20] node[midway,above] {rf} (sigE4State.south);

%\draw [dashed,->,>=stealth,color=blue,thin] (rhoA) -- node[midway,above] {rf} (rhoB2);
%\vspace{-10pt}
\end{tikzpicture}
} &
		\resizebox{0.19\textwidth}{!}{\tikzset{every picture/.style={line width=0.75pt}} %set default line width to 0.75pt        
\begin{tikzpicture}[x=1em,y=1em,yscale=-1,xscale=-1]
\tikzstyle{every node}=[font=\normalfont]
\node (w1) [inner sep=2pt] {$ w_1 $};
\node (w2) [right=25pt of w1,inner sep=2pt] {$ w^\sc_2 $};
\node (r1) [below left=20pt and -15pt of w1,inner sep=2pt] {$ r^\sc_1 $};
\node (sofr) [below right=-2pt and -2pt of r1] {\hlref{sofr}};

%\draw [->,>=stealth,color=Mahogany,thin] (er1)+(-7pt, 0) -- node[midway,above=-2pt,font=\scriptsize,color=black] { $\lchb$ } (ew1)+(7pt, 0);
\draw [->,>=stealth,color=RedOrange,thin] (w1) -- node[midway,above=-2pt,font=\scriptsize,color=black] { $\lmo$ } (w2);
\draw [->,>=stealth,color=Mahogany,thin] (r1.east) -- 
node[midway,right=1pt,font=\scriptsize,color=black] { $\lfr/\lso$ } (w2);
\draw [->,>=stealth,color=PineGreen,thin] (w1) -- node[midway,left=-2pt,font=\scriptsize,color=black] { $\lrf$ } (r1);`

\end{tikzpicture}
} &
		\resizebox{0.19\textwidth}{!}{\tikzset{every picture/.style={line width=0.75pt}} %set default line width to 0.75pt        
\begin{tikzpicture}[x=1em,y=1em,yscale=-1,xscale=-1]
\tikzstyle{every node}=[font=\normalfont]
\node (f1) [inner sep=2pt] {$ f^\sc_1 $};
\node (w2) [right=25pt of f1,inner sep=2pt] {$ w^\sc_2 $};
\node (r1) [below left=22pt and -13pt of f1,inner sep=2pt] {$ r_1 $};
\node (w1) [below left=22pt and -15pt of w2,inner sep=2pt] {$ w_1 $};
\node (sofwfr) [below right=0pt and -5pt of r1] {\hlref{soFWfr}};

`\draw [->,>=stealth,color=Mahogany,thin] (f1) -- node[midway,above=-2pt,font=\scriptsize,color=black] { $\lso$ } (w2);
\draw [->,>=stealth,color=PineGreen,thin] (w1) -- node[midway,below=-2pt,font=\scriptsize,color=black] { $ \lrf $ } (r1);
\draw [->,>=stealth,color=CarnationPink,thin] (f1) -- node[midway,left=-2pt,font=\scriptsize,color=black] { $\lsb$ } (r1);
\draw [->,>=stealth,color=RedOrange,thin] (w1) -- node[midway,left=-2pt,font=\scriptsize,color=black] { $\lmo$ } (w2);

\end{tikzpicture}
} &
		\resizebox{0.19\textwidth}{!}{\tikzset{every picture/.style={line width=0.75pt}} %set default line width to 0.75pt        
\begin{tikzpicture}[x=1em,y=1em,yscale=-1,xscale=-1]
\tikzstyle{every node}=[font=\normalfont]
\node (w1) {$ w_1 $};
\node (w2) [right=25pt of w1] {$ w_2 $};
\node (r1) [below left=20pt and -15pt of w1] {$ r^\sc_1 $};
\node (f1) [below left=20pt and -15pt of w2] {$ f^\sc_1 $};
\node (sorffr) [below right=0pt and -5pt of r1] {\hlref{soRFfr}};

`\draw [->,>=stealth,color=RedOrange,thin] (w1) -- node[midway,above=-2pt,font=\scriptsize,color=black] { $\lmo$ } (w2);
\draw [->,>=stealth,color=PineGreen,thin] (w1) -- node[midway,left=-2pt,font=\scriptsize,color=black] { $ \lrf $ } (r1);
\draw [->,>=stealth,color=CarnationPink,thin] (f1) -- node[midway,left=-2pt,font=\scriptsize,color=black] { $\lsb$ } (w2);
\draw [->,>=stealth,color=Mahogany,thin] (r1) -- node[midway,below=-2pt,font=\scriptsize,color=black] { $\lso$ } (f1);

\end{tikzpicture}
} &
		\resizebox{0.19\textwidth}{!}{\tikzset{every picture/.style={line width=0.75pt}} %set default line width to 0.75pt        
\begin{tikzpicture}[x=1em,y=1em,yscale=-1,xscale=-1]
\tikzstyle{every node}=[font=\normalfont]
\node (f1) [inner sep=2pt] {$ f^\sc_1 $};
\node (f2) [right=25pt of f1,inner sep=2pt] {$ f^\sc_2 $};
\node (r1) [below left=25pt and -13pt of f1, inner sep=1pt] {$ r_1 $};
\node (w1) [below left=25pt and -15pt of f2, inner sep=1pt] {$ w_1 $};
\node (w2) [below right=2pt and 1pt of f1,inner sep=2pt] {$ w_2 $};
\node (sofffr) [below right=0pt and -5pt of r1] {\hlref{soFFfr}};

`\draw [->,>=stealth,color=RedOrange,thin] (w2) -- node[pos=0.7,left=-2pt,font=\scriptsize,color=black] { $\lmo$ } (w1);
\draw [->,>=stealth,color=PineGreen,thin] (w2) -- node[pos=0.7,right=-2pt,font=\scriptsize,color=black] { $ \lrf $ } (r1);
\draw [->,>=stealth,color=CarnationPink,thin] (f1) -- node[midway,left=-2pt,font=\scriptsize,color=black] { $\lsb$ } (r1);
\draw [->,>=stealth,color=CarnationPink,thin] (w1) -- node[midway,left=-2pt,font=\scriptsize,color=black] { $\lsb$ } (f2);
\draw [->,>=stealth,color=Mahogany,thin] (f1) -- node[midway,above=-2pt,font=\scriptsize,color=black] { $\lso$ } (f2);

\end{tikzpicture}
} \\
		\hline
	\end{tabular}
	\caption{\lso-rules}
	\label{fig:so rules}
\end{figure}

\begin{itemize}[label=soFFnrf,align=left,leftmargin=*]
	\item [\hl{sohb}:] $\forall wr^{\sc}_1, wr^{\sc}_2 \in \ordevents{\sc}_\tau$ if 
			$\hb{\tau}{wr^{\sc}_1}{wr^{\sc}_2}$ then 
			$\so{\tau}{wr^{\sc}_1}{wr^{\sc}_2}$ \newline
			($\setHB$ order implies $\setSO$)
	
	\item [\hl{somo}:] $\forall w^{\sc}_1, w^{\sc}_2 \in \ordwrites{\sc}_\tau$ if 
			$\mo{\tau}{w^{\sc}_1}{w^{\sc}_2}$
			then $\so{\tau}{w^{\sc}_1}{w^{\sc}_2}$ \newline
			($\setMO$ order implies $\setSO$)
			
	\item [\hl{sorf}:] $\forall w^{\sc}_1 \in \ordwrites{sc}_\tau$, $r^{\sc}_1 \in 
			\ordreads{\sc}_\tau$ if $\rf{\tau}{w^{\sc}_1}{r^{\sc}_1}$ then 
			$\so{\tau}{w^{\sc}_1}{r^{\sc}_1}$ \newline
			($\setRF$ order implies $\setSO$)
			
	\item [\hl{sofr}:] $\forall r^\sc_1 \in \ordreads{\sc}_\tau$, $w^\sc_1 \in
			\ordwrites{\sc}_\tau$ if $\fr{\tau}{r^\sc_1}{w^sc_1}$ then 
			$\so{\tau}{r^{\sc}_1}{w^{\sc}_1}$ 
			\newline
			($\setFR$ order implies $\setSO$)
			
	\item [\hl{soFWfr}:] $\forall f^{\sc}_1 \in \ordfences{\sc}_\tau$, $w^{\sc}_1 \in 
			\ordwrites{\sc}_\tau$ if $\exists w_2 \in \writes_\tau$, $r_1 \in 
			\reads_\tau$ \st $\seqb{\tau}{f^{\sc}_1}{r_1}$, $\mo{\tau}{w_2}{w^{\sc}_1}$ 
			and $\rf{\tau}{w_2}{r_1}$ then $\so{\tau}{f^{\sc}_1}{w^{\sc}_1}$ \newline
			($\setFR$ though \sc write-fence synchronization implies $\setSO$)
			
	\item [\hl{soRFfr}:] $\forall f^\sc_1 \in \ordfences{\sc}_\tau$, $r^\sc_1 \in
			\ordreads{\sc}_\tau$ if $\exists w_1, w_2 \in \writes_\tau$ \st
			$\seqb{\tau}{w_2}{f^\sc_1}$, $\mo{\tau}{w_1}{w_2}$ and
			$\rf{\tau}{w_1}{r^\sc_1}$ then $\so{\tau}{r^\sc_1}{f^\sc_1}$\newline
			($\setFR$ though \sc read-fence synchronization implies $\setSO$)
	
%	snj: the next 3 are covered under hb implies so		
%	\item [\hl{soFFfr}:] $\forall f^\sc_1, f^\sc_2 \in \ordfences{\sc}_\tau$,
%			$w_1, w_2 \in \writes_\tau$, $r_1 \in \reads_\tau$ \st 
%			$\seqb{\tau}{f^\sc_1}{r_1}$, $\seqb{\tau}{w_2}{f^\sc_2}$, 
%			$\mo{\tau}{w_1}{w_2}$ and $\rf{\tau}{w_1}{r_1}$ then
%			$\so{\tau}{f^\sc_1}{f^\sc_2}$\newline
%			($\setFR$ though \sc fence-fence synchronization implies $\setSO$)
%			
%	\item [\hl{soWFsw}:] $\forall$ $w^\sc_1 \in \ordwrites{\sc}_\tau$ $f^\sc_1 \in
%			\ordfences{\sc}_\tau$ if $\exists r_1 \in \reads_\tau$ \st 
%			$\rf{\tau}{w^\sc_\tau}{r_1}$ and $\seqb{\tau}{r_1}{f^\sc_1}$ then 
%			$\so{\tau}{w^\sc_1}{f^\sc_1}$\newline
%			($\setSW$ between \sc write-fence implies $\setSO$)
%			
%	\item [\hl{soFRsw}:] $\forall r^\sc_1 \in \ordreads{\sc}_\tau$, $f^\sc_1 \in
%			\ordfences{\sc}_\tau$ if $\exists w_1 \in \writes_\tau$ \st
%			$\rf{\tau}{w_1}{r^\sc_1}$ and $\seqb{\tau}{f^\sc_1}{w_1}$ then
%			$\so{\tau}{f^\sc_1}{r^\sc_1}$\newline
%			($\setSW$ between \sc fence-read implies $\setSO$)
			
			
	\item [\hl{soFFsw}:]  $\forall f^{\sc}_1, f^{\sc}_2 \in \ordfences{\sc}_\tau$, 
			if $\exists w_1 \in \writes_\tau$, $r_1 \in \reads_\tau$ \st 
			$\seqb{\tau}{f^{\sc}_1}{w_1}$, $\seqb{\tau}{r_1}{f^{\sc}_2}$ 
			and $\rf{\tau}{w_1}{r_1}$ then $\so{\tau}{f^{\sc}_1}{f^{\sc}_2}$ \newline
			($\setSW$ between \sc fence-fences implies $\setSO$)
\end{itemize}

Consider a \cc execution sequence $\tau$ and a transformation
$\fixed{\tau}$ of $\tau$ \st $\events_{\fixed{\tau}}$ = $\events_\tau$
$\union$ set of synthesized fences and $\to{\tau}{}{}$ $\subseteq$
$\so{\fixed{\tau}}{}{}$.
%
Cyclic $\so{\fixed{\tau}}{}{}$ implies that there does not exist a 
total order on \sc ordered events of $\fixed{\tau}$. The sequence $\fixed{\tau}$, 
then, is not a valid \cc execution and the buggy sequence $\tau$ 
has been invalidated.

Thus, the aim of this work is to synthesis fences in the input program $P$
at appropriate locations and force a cyclic $\setSO$ order in the \sc 
ordered events of a buggy trace $\tau$ of $P$ and invalidating the trace 
for the transformed program $\fixed{P}$.