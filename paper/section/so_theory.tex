\ourtechnique attempts to invalidate a buggy trace (\aka counter
example) by one of the following two strategies:
\begin{itemize}[label=strategy1,align=left,leftmargin=*]
	\item [Strategy1]:
		Invalidate by violating \sc-\lto-order.
	\item [Strategy2]:
		Invalidate by increasing inter-thread synchronization.
\end{itemize}

\noindent
{\bf Strategy1: Invalidate by violating \sc-\lto-order. 
	(\sfence)}\newline
Consider a buggy input program $P$.
%
As discussed in Section~\ref{sec:c11}, in a valid trace of $P$ 
(including a buggy trace), 
the \sc ordered events must form a total order ($\setTO$).
%
Contrarily, if we synthesize \sc ordered fences in the input program 
such that the total order requirement in the buggy trace gets violated 
then we can invalidate the trace and stop the program behavior.

\begin{figure}[t]
	\begin{tabular}{|c||c|c|c|}
		\multicolumn{1}{c}{base rule} & 
		\multicolumn{3}{c}{extended fence rules} \\\hline
		
		\resizebox{0.24\textwidth}{!}{\tikzset{every picture/.style={line width=0.75pt}} %set default line width to 0.75pt        
\begin{tikzpicture}[x=1em,y=1em,yscale=-1,xscale=-1]
\tikzstyle{every node}=[font=\normalfont]
\node (wr1) {$ wr^\sc_1 $};
\node (wr2) [below=20pt of wr1] {$ wr^\sc_2 $};
\node (so1) [below=-5pt of wr2] {\hlref{sohb},\hlref{somo},};
\node (so2) [below left=-5pt and -66pt of so1] {\hlref{sorf},\hlref{sofr}};

\draw [->,>=stealth,color=Mahogany,thin] ($ (wr1.south east)+(.3,-5pt) $) to[out=135,in=-135] node[midway,right=-2pt,font=\scriptsize] {\textcolor{black}{\lso}} ($ (wr2.south east)+(0.4,-7pt) $);
\draw [->,>=stealth,color=RedOrange,thin] ($ (wr1.south west)+(-.3,-5pt) $) to[out=45,in=-45] node[left=-2pt,pos=.25,font=\scriptsize] {\textcolor{black}{\lmo/\lrf}} node[left=-0.5pt,pos=.5,font=\scriptsize] {\textcolor{black}{\lhb/\lfr}} ($ (wr2.south west)+(-0.3,-7pt) $);




%\node (nodeB) {$ b: r_1 \assign x  $};
%\node (sigB1) [below left=-3pt and -5pt of nodeB, color=green] {$\sigma_{b11} = $};
%\node (poB1empty) [right=-1pt of sigB1, color=green] {$ $};
%\node (poB1) [draw,fit=(poB1empty), color=orange, thin, inner sep=0pt] {};
%%\node (poB1n1) [right=-3pt of sigB1, color=green] {$ $};
%%\node (poB1) [draw,fit=(poB1n1), color=orange, thin, inner sep=0pt] {};
%\node (B1Val) [right=-3pt of poB1, color=green] {$,0$};
%\node (sigB1State) [fit=(poB1)(B1Val)(sigB1), inner sep=-1pt] {}; 

%\node (sigB2) [right=-3pt of sigB1State, color=green] {$\sigma_{b22} = $};
%\node (poB2N1) [right=6pt of sigB2, circle,fill=black,inner sep=0pt,minimum size=3pt] {};
%\node (poB2a) [left=1pt of poB2N1, inner sep=1pt] {$a$};
%\node (poB2) [draw,fit=(poB2N1)(poB2a), color=orange, thin, inner sep=1pt] {};
%\node (B2Val) [right=-3pt of poB2, color=green] {$,1$};
%\node (sigB2State) [fit=(poB2)(B2Val)(sigB2), inner sep=0pt] {}; 
%
%
%
%%\node (rhoB1) [below =10pt of sigB1State, color=red] {$\rho_{b1}:=0$};
%%\node (rhoB2) [below =10pt of sigB2State, color=red] {$\rho_{b2}:=1$};
%
%
%\node (nodeC) [below =2 of nodeB] {$c: x \assign 2$};
%\node (sigC1) [below left=3pt and -4pt of nodeC, color=green] {$\sigma_{c11} = $};
%\node (poC1N1) [right=6pt of sigC1, circle,fill=black,inner sep=0pt,minimum size=3pt] {};
%\node (poC1c) [left=1pt of poC1N1, inner sep=1pt] {$c$};
%\node (poC1) [draw,fit=(poC1c)(poC1N1), color=orange, thin, inner sep=1pt] {};
%\node (C1Val) [right=-3pt of poC1, color=green] {$,2$};
%\node (sigC1State) [fit=(poC1)(C1Val)(sigC1), inner sep=0pt] {}; 
%
%\node (sigC2) [right=-3pt of sigC1State, color=green] {$\sigma_{c22} = $};
%\node (poC2n1) [above right=0pt and 5pt of sigC2, circle,fill=black,inner sep=0pt,minimum size=3pt] {};
%\node (poC2a) [left=1pt of poC2n1, inner sep=0pt] {$a$};
%\node (poC2n2) [below=9pt of poC2n1, circle,fill=black,inner sep=0pt,minimum size=3pt] {};
%\node (poC2c) [left=1pt of poC2n2, inner sep=0pt] {$c$};
%\draw [color=purple] (poC2n1) -- (poC2n2);
%\node (poC2) [draw,fit=(poC2n1)(poC2a)(poC2n2)(poC2c), color=orange, thin, inner sep=1pt] {};
%\node (C2Val) [right=-3pt of poC2, color=green] {$,2$};
%\node (sigC2State) [fit=(poC2)(C2Val)(sigC2), inner sep=0pt] {}; 
%
%
%
%%\node (rhoC) [below=30pt of nodeC, color=red] {$\rho_{c}=2$};
%
%\node (nodeA) [left =3.5 of nodeB] {$a: x \assign 1$};
%\node (sigA) [below left=-2pt and -25pt of nodeA, color=green] {$\sigma_{a11} = $};
%\node (poAN1) [right=6pt of sigA, circle,fill=black,inner sep=0pt,minimum size=3pt] {};
%\node (poAa) [left=1pt of poAN1, inner sep=1pt] {$a$};
%\node (poA) [draw,fit=(poAN1)(poAa), color=orange, thin, inner sep=1pt] {};
%\node (AVal) [right=-3pt of poA, color=green] {$,1$};
%\node (sigAState) [fit=(poA)(AVal)(sigA), inner sep=-1pt] {}; 
%
%%\node (rhoA) [below=1pt of sigAState, color=red] {$\rho_{a} := 1$};
%
%
%\node (nodeD) [right =5.5 of nodeB] {$ d: r_2 \assign x  $};
%\node (sigD1) [below left=3pt and -5pt of nodeD, color=green] {$\sigma_{d22} = $};
%\node (poD1N1) [right=6pt of sigD1, circle,fill=black,inner sep=0pt,minimum size=3pt] {};
%\node (poD1c) [left=1pt of poD1N1, inner sep=1pt] {$c$};
%\node (poD1) [draw,fit=(poD1c)(poD1N1), color=orange, thin, inner sep=1pt] {};
%\node (D1Val) [right=-3pt of poD1, color=green] {$,2$};
%\node (sigD1State) [fit=(poD1)(D1Val)(sigD1), inner sep=0pt] {}; 
%
%\node (sigD2) [right=-3pt of sigD1State, color=green] {$\sigma_{d33} = $};
%\node (poD2n1) [above right=0pt and 5pt of sigD2, circle,fill=black,inner sep=0pt,minimum size=3pt] {};
%\node (poD2a) [left=1pt of poD2n1, inner sep=0pt] {$a$};
%\node (poD2n2) [below=9pt of poD2n1, circle,fill=black,inner sep=0pt,minimum size=3pt] {};
%\node (poD2c) [left=1pt of poD2n2, inner sep=0pt] {$c$};
%\draw [color=purple] (poD2n1) -- (poD2n2);
%\node (poD2) [draw,fit=(poD2n1)(poD2a)(poD2n2)(poD2c), color=orange, thin, inner sep=1pt] {};
%\node (D2Val) [right=-3pt of poD2, color=green] {$,2$};
%\node (sigD2State) [fit=(poD2)(D2Val)(sigD2), inner sep=0pt] {};  
%
%
%
%\node (nodeE) [below =2 of nodeD] {$e: r_3 \assign x$};
%\node (sigE1) [below left=3pt and -5pt of nodeE, color=green] {$\sigma_{e23} = $};
%\node (poE1N1) [right=6pt of sigE1, circle,fill=black,inner sep=0pt,minimum size=3pt] {};
%\node (poE1c) [left=1pt of poE1N1, inner sep=1pt] {$c$};
%\node (poE1) [draw,fit=(poE1c)(poE1N1), color=orange, thin, inner sep=1pt] {};
%\node (E1Val) [right=-3pt of poE1, color=green] {$,2$};
%\node (sigE1State) [fit=(poE1)(E1Val)(sigE1), inner sep=0pt] {}; 
%
%\node (sigE2) [right=-3pt of sigE1State, color=green] {$\sigma_{e34} = $};
%\node (poE2n1) [above right=0pt and 5pt of sigE2, circle,fill=black,inner sep=0pt,minimum size=3pt] {};
%\node (poE2a) [left=1pt of poE2n1, inner sep=0pt] {$a$};
%\node (poE2n2) [below=9pt of poE2n1, circle,fill=black,inner sep=0pt,minimum size=3pt] {};
%\node (poE2c) [left=1pt of poE2n2, inner sep=0pt] {$c$};
%\draw [color=purple] (poE2n1) -- (poE2n2);
%\node (poE2) [draw,fit=(poE2n1)(poE2a)(poE2n2)(poE2c), color=orange, thin, inner sep=1pt] {};
%\node (E2Val) [right=-3pt of poE2, color=green] {$,2$};
%\node (sigE2State) [fit=(poE2)(E2Val)(sigE2), inner sep=0pt] {}; 
%
%\node (sigE3) [below =9pt of sigE1, color=green] {$\sigma_{e32} = $};
%\node (poE3n1) [above right=0pt and 5pt of sigE3, circle,fill=black,inner sep=0pt,minimum size=3pt] {};
%\node (poE3c) [left=1pt of poE3n1, inner sep=0pt] {$c$};
%\node (poE3n2) [below=9pt of poE3n1, circle,fill=black,inner sep=0pt,minimum size=3pt] {};
%\node (poE3a) [left=1pt of poE3n2, inner sep=0pt] {$a$};
%\draw [color=purple] (poE3n1) -- (poE3n2);
%\node (poE3) [draw,fit=(poE3n1)(poE3a)(poE3n2)(poE3c), color=orange, thin, inner sep=1pt] {};
%\node (E3Val) [right=-3pt of poE3, color=green] {$,1$};
%\node (sigE3State) [fit=(poE3)(E3Val)(sigE3), inner sep=0pt] {};
%
%\node (sigE4) [right =-11pt of sigE3State, color=green] {$\quad \sigma_{e35} = \bot$};
%\node (sigE4State) [fit=(sigE4), inner sep=1pt] {};
%
%
%
%
%%\draw [dashed,->,>=stealth,color=brown,thin] (sigB1State.west) to[out=45,in=-45] (sigC1State.west);
%%\draw [dashed,->,>=stealth,color=brown,thin] (sigB2State.west) to[out=45,in=-45] (sigC2State.west);
%%\draw [dashed,->,>=stealth,color=brown,thin] (sigD1State.west) to[out=45,in=-45] (sigE1State.west);
%%\draw [dashed,->,>=stealth,color=brown,thin] (sigD2State.west) to[out=45,in=-45] (sigE2State.west);
%%\draw [dashed,->,>=stealth,color=brown,thin] (sigD1State.east) to[out=135,in=-135] (sigE3State.east);
%%\draw [dashed,->,>=stealth,color=brown,thin] (sigD2State.east) to[out=135,in=-135] (sigE4State.east);
%%\draw [dashed,->,>=stealth,color=blue,thin] (sigC1State.north east) to[out=-110,in=0] node[midway,left] {rf} (sigD1State.west);
%%\draw [dashed,->,>=stealth,color=blue,thin] (sigC2State.east) to[out=-150,in=0] node[midway,left] {rf} (sigD2State.west);
%
%\draw [dashed,->,>=stealth,color=blue,thin] (sigAState) to[out=165,in=25] node[midway,above] {rf} (sigB2State.south);
%\draw [dashed,->,>=stealth,color=blue,thin] (sigC1State.north east) -- node[midway,above] {rf} (sigD1State.south west);
%\draw [dashed,->,>=stealth,color=blue,thin] (sigC2State.north east) -- node[midway,above] {rf} (sigD2State.south west);
%\draw [dashed,->,>=stealth,color=blue,thin] (sigAState) to[out=105,in=15] node[midway,above] { rf} (sigE3State.west);
%\draw [dashed,->,>=stealth,color=blue,thin] (sigAState) to[out=100,in=20] node[midway,above] {rf} (sigE4State.south);

%\draw [dashed,->,>=stealth,color=blue,thin] (rhoA) -- node[midway,above] {rf} (rhoB2);
%\vspace{-10pt}
\end{tikzpicture}
} &
		\resizebox{0.24\textwidth}{!}{\tikzset{every picture/.style={line width=0.75pt}} %set default line width to 0.75pt        
\begin{tikzpicture}[x=1em,y=1em,yscale=-1,xscale=-1]
\tikzstyle{every node}=[font=\normalfont]
\node (wr1) [inner sep=2pt] {$ wr^\sc_1 $};
\node (wr2) [right=25pt of wr1,inner sep=2pt] {$ wr_2 $};
\node (f1) [below left=21pt and -15pt of wr2,inner sep=2pt] {$ f^\sc_1 $};
\node (soef) [below left=0pt and -10pt of f1] {\hlref{soEF}};

`\draw [->,>=stealth,color=Mahogany,thin] (wr1.south) -- node[midway,left=0pt,font=\scriptsize,color=black] { $\lso$ } (f1.west);
\draw [->,>=stealth,color=Cerulean,thin] (wr1) -- node[midway,above=-2pt,font=\scriptsize,color=black] { $ \lmo/ $ } node[midway,below=-2pt,font=\scriptsize,color=black] { $ \lhb/\lrf $ } (wr2);
\draw [->,>=stealth,color=CarnationPink,thin] (wr2) -- node[midway,left=-2pt,font=\scriptsize,color=black] { $\lsb$ } (f1);

\end{tikzpicture}
} &
		\resizebox{0.24\textwidth}{!}{\tikzset{every picture/.style={line width=0.75pt}} %set default line width to 0.75pt        
\begin{tikzpicture}[x=1em,y=1em,yscale=-1,xscale=-1]
\tikzstyle{every node}=[font=\normalfont]
\node [inner sep=2pt] (f1) {$ f^\sc_1 $};
\node (wr1) [below left=20pt and -15pt of f1,inner sep=2pt] {$ wr_1 $};
\node (wr2) [right=25pt of wr1,inner sep=2pt] {$ wr^\sc_2 $};
\node (sofe) [below right=0pt and -5pt of wr1] {\hlref{soFE}};

`\draw [->,>=stealth,color=Mahogany] (f1.east) -- node[midway,right=0pt,font=\scriptsize,color=black] { $\lso$ } (wr2.north);
\draw [->,>=stealth,color=Cerulean] (wr1) -- node[midway,above=-2pt,font=\scriptsize,color=black] { $ \lmo/\lrf $ } node[midway,below=-2pt,font=\scriptsize,color=black] { $ \lhb/\lfr $ } (wr2);
\draw [->,>=stealth,color=CarnationPink] (f1) -- node[midway,left=-2pt,font=\scriptsize,color=black] { $\lsb$ } (wr1);

\end{tikzpicture}
} &
		\resizebox{0.24\textwidth}{!}{\tikzset{every picture/.style={line width=0.75pt}} %set default line width to 0.75pt        
\begin{tikzpicture}[x=1em,y=1em,yscale=-1,xscale=-1]
\tikzstyle{every node}=[font=\normalfont]
\node (f1) [inner sep=2pt] {$ f^\sc_1 $};
\node (f2) [right=25pt of f1,inner sep=2pt] {$ f^\sc_2 $};
\node (wr1) [below left=20pt and -15pt of f1,inner sep=2pt] {$ wr_1 $};
\node (wr2) [below left=20pt and -15pt of f2,inner sep=2pt] {$ wr_2 $};
\node (soff) [below right=0pt and -5pt of wr1] {\hlref{soFF}};

`\draw [->,>=stealth,color=Mahogany] (f1) -- node[midway,above=-2pt,font=\scriptsize,color=black] { $\lso$ } (f2);
\draw [->,>=stealth,color=Cerulean] (wr1) -- node[midway,above=-2pt,font=\scriptsize,color=black]{\lmo/\lrf} node[midway,below=-2pt,font=\scriptsize,color=black]{ $ \lhb/\lfr $ } (wr2);
\draw [->,>=stealth,color=CarnationPink] (f1) -- node[midway,left=-2pt,font=\scriptsize,color=black] { $\lsb$ } (wr1);
\draw [->,>=stealth,color=CarnationPink] (wr2) -- node[midway,left=-2pt,font=\scriptsize,color=black] { $\lsb$ } (f2);

%\draw [->,>=stealth,color=orange] ($ (ew1.south east)+(.5,-5pt) $) to[out=135,in=-135] node[midway,right=-2pt,font=\scriptsize] {mo} ($ (ew2.south east)+(0.4,-5pt) $);
%\draw [->,>=stealth,color=red] ($ (ew1.south west)+(-.3,-5pt) $) to[out=45,in=-45] node[midway,left=-2pt,font=\scriptsize] {c::hb} ($ (ew2.south west)+(-0.3,-5pt) $);


\end{tikzpicture}
} \\
		\hline
		\multicolumn{4}{c}{(a) \lso-rules} \\
		\hline
		
		\resizebox{0.24\textwidth}{!}{\tikzset{every picture/.style={line width=0.75pt}} %set default line width to 0.75pt        
\begin{tikzpicture}[x=1em,y=1em,yscale=-1,xscale=-1]
\tikzstyle{every node}=[font=\normalfont]
\node (w1) {$ w^\rel_1 $};
\node (r1) [below=20pt of w1] {$ r^\acq_1 $};
\node (sw) [below=-5pt of r1] {\hlref{sw}};

\draw [->,>=stealth,color=Magenta] ($ (w1.south east)+(.3,-5pt) $) to[out=135,in=-135] node[midway,right=-2pt,font=\scriptsize] {\textcolor{black}{\lsw}} ($ (r1.south east)+(0.4,-7pt) $);
\draw [->,>=stealth,color=PineGreen] ($ (w1.south west)+(-.3,-5pt) $) to[out=45,in=-45] node[left=-2pt,pos=.25,font=\scriptsize] {\textcolor{black}{\lrf}}($ (r1.south west)+(-0.3,-7pt) $);

\end{tikzpicture}
} &
		\resizebox{0.24\textwidth}{!}{\tikzset{every picture/.style={line width=0.75pt}} %set default line width to 0.75pt        
\begin{tikzpicture}[x=1em,y=1em,yscale=-1,xscale=-1]
\tikzstyle{every node}=[font=\normalfont]
\node (w1) [inner sep=2pt] {$ w^\rel_1 $};
\node (r1) [right=25pt of w1,inner sep=2pt] {$ r_1 $};
\node (f1) [below left=21pt and -15pt of r1,inner sep=2pt] {$ f^\acq_1 $};
\node (swef) [below left=0pt and -10pt of f1] {\hlref{swEF}};

`\draw [->,>=stealth,color=Magenta,thin] (w1.south) -- node[midway,left=0pt,font=\scriptsize,color=black] { $\lsw$ } (f1.west);
\draw [->,>=stealth,color=PineGreen,thin] (w1) -- node[midway,above=-2pt,font=\scriptsize,color=black] { $ \lrf $ }  (r1);
\draw [->,>=stealth,color=CarnationPink,thin] (r1) -- node[midway,left=-2pt,font=\scriptsize,color=black] { $\lsb$ } (f1);

\end{tikzpicture}
} &
		\resizebox{0.24\textwidth}{!}{\tikzset{every picture/.style={line width=0.75pt}} %set default line width to 0.75pt        
\begin{tikzpicture}[x=1em,y=1em,yscale=-1,xscale=-1]
\tikzstyle{every node}=[font=\normalfont]
\node [inner sep=2pt] (f1) {$ f^\rel_1 $};
\node (w1) [below left=21pt and -15pt of f1,inner sep=2pt] {$ w_1 $};
\node (r1) [right=25pt of w1,inner sep=2pt] {$ r^\acq_1 $};
\node (swfe) [below right=0pt and -15pt of wr1] {\hlref{sw-dobFE}};

`\draw [->,>=stealth,color=Magenta,thin] (f1.east) -- node[midway,right=0pt,font=\scriptsize,color=black] { $\lsw$ } (r1.north);
\draw [->,>=stealth,color=PineGreen,thin] (w1) -- node[midway,above=-2pt,font=\scriptsize,color=black] { $ \lrf $ } (r1);
\draw [->,>=stealth,color=CarnationPink,thin] (f1) -- node[midway,left=-2pt,font=\scriptsize,color=black] { $\lsb$ } (w1);

\end{tikzpicture}
} &
		\resizebox{0.24\textwidth}{!}{\tikzset{every picture/.style={line width=0.75pt}} %set default line width to 0.75pt        
\begin{tikzpicture}[x=1em,y=1em,yscale=-1,xscale=-1]
\tikzstyle{every node}=[font=\normalfont]
\node (f1) [inner sep=2pt] {$ f^\rel_1 $};
\node (f2) [right=25pt of f1,inner sep=2pt] {$ f^\acq_2 $};
\node (w1) [below left=20pt and -15pt of f1,inner sep=2pt] {$ w_1 $};
\node (r1) [below left=20pt and -15pt of f2,inner sep=2pt] {$ r_1 $};
\node (swff) [below right=-2pt and -5pt of w1] {\hlref{swFF}};

`\draw [->,>=stealth,color=Magenta] (f1) -- node[midway,above=-2pt,font=\scriptsize,color=black] { $\lsw$ } (f2);
\draw [->,>=stealth,color=PineGreen] (w1) -- node[midway,above=-2pt,font=\scriptsize,color=black]{\lrf} (r1);
\draw [->,>=stealth,color=CarnationPink] (f1) -- node[midway,left=-2pt,font=\scriptsize,color=black] { $\lsb$ } (w1);
\draw [->,>=stealth,color=CarnationPink] (r1) -- node[midway,left=-2pt,font=\scriptsize,color=black] { $\lsb$ } (f2);

%\draw [->,>=stealth,color=orange] ($ (ew1.south east)+(.5,-5pt) $) to[out=135,in=-135] node[midway,right=-2pt,font=\scriptsize] {mo} ($ (ew2.south east)+(0.4,-5pt) $);
%\draw [->,>=stealth,color=red] ($ (ew1.south west)+(-.3,-5pt) $) to[out=45,in=-45] node[midway,left=-2pt,font=\scriptsize] {c::hb} ($ (ew2.south west)+(-0.3,-5pt) $);


\end{tikzpicture}
} \\
		
		\resizebox{0.24\textwidth}{!}{\tikzset{every picture/.style={line width=0.75pt}} %set default line width to 0.75pt        
\begin{tikzpicture}[x=1em,y=1em,yscale=-1,xscale=-1]
\tikzstyle{every node}=[font=\normalfont]
\node (w1) [inner sep=2pt] {$ w^\rel_1 $};
\node (r1) [right=25pt of w1,inner sep=2pt] {$ r^\acq_1 $};
\node (w2) [below left=21pt and -15pt of w1,inner sep=2pt] {$ w_2 $};
\node (dob) [below right=0pt and -5pt of w2] {\hlref{dob}};

`\draw [->,>=stealth,color=Mulberry] (w1.east) -- node[midway,above=-2pt,font=\scriptsize,color=black] { $\ldob$ } (r1.west);
\draw [->,>=stealth,color=PineGreen] (w2) -- node[midway,below=-2pt,font=\scriptsize,color=black] { $ \lrf $ }  (r1);
\draw [->,>=stealth,color=CarnationPink] (w1) -- node[midway,left=-2pt,font=\scriptsize,color=black] { $\lsb$ } (w2);

\end{tikzpicture}
} &
		\resizebox{0.24\textwidth}{!}{\tikzset{every picture/.style={line width=0.75pt}} %set default line width to 0.75pt        
\begin{tikzpicture}[x=1em,y=1em,yscale=-1,xscale=-1]
\tikzstyle{every node}=[font=\normalfont]
\node (w1) [inner sep=2pt] {$ w^\rel_1 $};
\node (f1) [right=25pt of w1,inner sep=2pt] {$ f^\acq_1 $};
\node (w2) [below left=20pt and -15pt of w1,inner sep=2pt] {$ w_2 $};
\node (r1) [below left=20pt and -13pt of f1,inner sep=2pt] {$ r_1 $};
\node (dobEF) [below right=0pt and -5pt of w2] {\hlref{dobEF}};

`\draw [->,>=stealth,color=PineGreen,thin] (w2) -- node[midway,above=-2pt,font=\scriptsize,color=black] { $\lrf$ } (r1);
\draw [->,>=stealth,color=CarnationPink,thin] (w1) -- node[midway,left=-2pt,font=\scriptsize,color=black] { $ \lsb $ } (w2);
\draw [->,>=stealth,color=CarnationPink,thin] (r1) -- node[midway,left=-2pt,font=\scriptsize,color=black] { $\lsb$ } (f1);
\draw [->,>=stealth,color=Mulberry,thin] (w1) -- node[midway,above=-2pt,font=\scriptsize,color=black] { $\ldob$ } (f1);

\end{tikzpicture}
} &&\\
		\hline
		\multicolumn{4}{c}{(b) \lsw- and \ldob-rules} \\
		\hline
		
		\resizebox{0.24\textwidth}{!}{\tikzset{every picture/.style={line width=0.75pt}} %set default line width to 0.75pt        
\begin{tikzpicture}[x=1em,y=1em,yscale=-1,xscale=-1]
\tikzstyle{every node}=[font=\normalfont]
\node (r1) [inner sep=2pt] {$ r^\acq_1 $};
\node (w1) [right=25pt of r1,inner sep=2pt] {$ w^\rel_1 $};
\node (r2) [below left=25pt and -13pt of r1, inner sep=1pt] {$ r_2 $};
\node (w2) [below left=25pt and -15pt of w1, inner sep=1pt] {$ w_2 $};
%\node (w3) [below right=2pt and 1pt of r1,inner sep=2pt] {$ w_3 $};
\node (ws) [below right=0pt and 5pt of r2] {\hlref{ws}};

%`\draw [->,>=stealth,color=RedOrange,thin] (w3) -- node[pos=0.7,left=-2pt,font=\scriptsize,color=black] { $\lmo$ } (w2);
%\draw [->,>=stealth,color=PineGreen,thin] (w3) -- node[pos=0.7,right=-2pt,font=\scriptsize,color=black] { $ \lrf $ } (r2);
\draw [->,>=stealth,color=CarnationPink,thin] (r1) -- node[midway,left=-2pt,font=\scriptsize,color=black] { $\lsb$ } (r2);
\draw [->,>=stealth,color=CarnationPink,thin] (w2) -- node[midway,left=-2pt,font=\scriptsize,color=black] { $\lsb$ } (w1);
\draw [->,>=stealth,color=Mahogany,thin] (r1) -- node[midway,above=-2pt,font=\scriptsize,color=black] { $\lws$ } (w1);
\draw [->,>=stealth,color=RoyalPurple,thin] (r2) -- node[midway,above=-2pt,font=\scriptsize,color=black] { $\lfr$ } (w2);

\end{tikzpicture}
} &
		\resizebox{0.24\textwidth}{!}{\tikzset{every picture/.style={line width=0.75pt}} %set default line width to 0.75pt        
\begin{tikzpicture}[x=1em,y=1em,yscale=-1,xscale=-1]
\tikzstyle{every node}=[font=\normalfont]
\node (r1) [inner sep=2pt] {$ r^\acq_1 $};
\node (f1) [right=25pt of r1,inner sep=2pt] {$ f^\rel_1 $};
\node (r2) [below left=25pt and -13pt of r1, inner sep=1pt] {$ r_2 $};
\node (w2) [below left=25pt and -15pt of f1, inner sep=1pt] {$ w_2 $};
%\node (w1) [below right=2pt and 1pt of r1,inner sep=2pt] {$ w_1 $};
\node (wsEF) [below right=0pt and 0pt of r2] {\hlref{wsEF}};

%`\draw [->,>=stealth,color=RedOrange,thin] (w1) -- node[pos=0.7,left=-2pt,font=\scriptsize,color=black] { $\lmo$ } (w2);
%\draw [->,>=stealth,color=PineGreen,thin] (w1) -- node[pos=0.7,right=-2pt,font=\scriptsize,color=black] { $ \lrf $ } (r2);
\draw [->,>=stealth,color=CarnationPink,thin] (r1) -- node[midway,left=-2pt,font=\scriptsize,color=black] { $\lsb$ } (r2);
\draw [->,>=stealth,color=CarnationPink,thin] (w2) -- node[midway,left=-2pt,font=\scriptsize,color=black] { $\lsb$ } (f1);
\draw [->,>=stealth,color=Mahogany,thin] (r1) -- node[midway,above=-2pt,font=\scriptsize,color=black] { $\lws$ } (f1);
\draw [->,>=stealth,color=RoyalPurple,thin] (r2) -- node[midway,above=-2pt,font=\scriptsize,color=black] { $\lfr$ } (w2);

\end{tikzpicture}
} & 
		\resizebox{0.24\textwidth}{!}{\tikzset{every picture/.style={line width=0.75pt}} %set default line width to 0.75pt        
\begin{tikzpicture}[x=1em,y=1em,yscale=-1,xscale=-1]
\tikzstyle{every node}=[font=\normalfont]
\node (f1) [inner sep=2pt] {$ f^\acq_1 $};
\node (wx) [right=25pt of f1,inner sep=2pt] {$ w^\rel_1 $};
\node (r1) [below left=25pt and -13pt of f1, inner sep=1pt] {$ r_1 $};
\node (w2) [below left=25pt and -15pt of wx, inner sep=1pt] {$ w_2 $};
%\node (w1) [below right=2pt and 1pt of f1,inner sep=2pt] {$ w_2 $};
\node (wsFE) [below right=0pt and 5pt of r1] {\hlref{wsFE}};

\draw [->,>=stealth,color=CarnationPink] (f1) -- node[midway,left=-2pt,font=\scriptsize,color=black] { $\lsb$ } (r1);
\draw [->,>=stealth,color=CarnationPink] (w2) -- node[midway,left=-2pt,font=\scriptsize,color=black] { $\lsb$ } (wx);
\draw [->,>=stealth,color=Mahogany] (f1) -- node[midway,above=-2pt,font=\scriptsize,color=black] { $\lws$ } (wx);
\draw [->,>=stealth,color=RoyalPurple] (r2) -- node[midway,above=-2pt,font=\scriptsize,color=black] { $\lfr$ } (w2);

\end{tikzpicture}
} &
		\resizebox{0.24\textwidth}{!}{\tikzset{every picture/.style={line width=0.75pt}} %set default line width to 0.75pt        
\begin{tikzpicture}[x=1em,y=1em,yscale=-1,xscale=-1]
	\tikzstyle{every node}=[font=\normalfont]
	\node (f1) [inner sep=2pt] {$ f^\acq_1 $};
	\node (f2) [right=25pt of f1,inner sep=2pt] {$ f^\rel_2 $};
	\node (r1) [below left=25pt and -13pt of f1, inner sep=1pt] {$ r_1 $};
	\node (w1) [below left=25pt and -15pt of wx, inner sep=1pt] {$ w_1 $};
	%\node (w1) [below right=2pt and 1pt of f1,inner sep=2pt] {$ w_2 $};
	\node (wsFF) [below right=0pt and 5pt of r1] {\hlref{wsFF}};
	
	%`\draw [->,>=stealth,color=RedOrange,thin] (w1) -- node[pos=0.7,left=-2pt,font=\scriptsize,color=black] { $\lmo$ } (w2);
	%\draw [->,>=stealth,color=PineGreen,thin] (w1) -- node[pos=0.7,right=-2pt,font=\scriptsize,color=black] { $ \lrf $ } (r1);
	\draw [->,>=stealth,color=CarnationPink,thin] (f1) -- node[midway,left=-2pt,font=\scriptsize,color=black] { $\lsb$ } (r1);
	\draw [->,>=stealth,color=CarnationPink,thin] (w2) -- node[midway,left=-2pt,font=\scriptsize,color=black] { $\lsb$ } (wx);
	\draw [->,>=stealth,color=Mahogany,thin] (f1) -- node[midway,above=-2pt,font=\scriptsize,color=black] { $\lws$ } (wx);
	\draw [->,>=stealth,color=RoyalPurple,thin] (r2) -- node[midway,above=-2pt,font=\scriptsize,color=black] { $\lfr$ } (w2);
	
\end{tikzpicture}
} \\
		\hline
		\multicolumn{4}{c}{(c) \lws-rules} \\	
	\end{tabular}
	\caption{Rules to form relation between program events and 
		candidate fences}
	\label{fig:so rules}
\end{figure}


We introduce an irreflexive and possibly cyclic relation
on \sc ordered events called \sc-order ($\setSO$).
%
To construct $\setSO$, we introduce a set of \lso-rules
(diagrammatically shown in Figure~\ref{fig:so rules}(a) and
discussed below).
%
The \lso base rules are simply implied from
the \lto-rules, \hlref{toHb}, \hlref{toMo} \hlref{toFr} 
and \hlref{toRf} (Section~\ref{sec:c11}).
%
The extended rules apply respective $\lto$-rules to relaxed
events with appropriately placed fences as described in
\cc \cite{C11}\cite{Batty-POPL12}.


\begin{longtable}{|p{0.11\textwidth} p{0.88\textwidth}|}
	\hline
	\multicolumn{2}{|l|}{\bf \lso-base rules:}\\
	
	\hl{sohb}, & 
	$\forall wr^{\sc}_1, wr^{\sc}_2 \in \ordevents{\sc}_\tau$ if \\
	\hl{somo}, &
	$\hb{\tau}{wr^{\sc}_1}{wr^{\sc}_2}$ $\v$ $\mo{\tau}{wr^{\sc}_1}{wr^{\sc}_2}$
	$\v$ $\rf{\tau}{wr^{\sc}_1}{wr^{\sc}_2}$ 
	$\v$ $\fr{\tau}{wr^{\sc}_1}{wr^{\sc}_2}$\\
	\hl{sorf}, & 
	then $\so{\tau}{wr^{\sc}_1}{wr^{\sc}_2}$ \\
	\hl{sofr}: & ($\setHB$, $\setMO$, $\setRF$. $\setFR$ between \sc events 
				implies $\setSO$) \\
	& \\
	
	\multicolumn{2}{|l|}{\bf \lso-rules extended for fences:} \\
	
	\hl{soEF}: & $\forall wr^\sc_1 \in \ordevents{\sc}_\tau$, $f^\sc_1 \in
	\ordfences{\sc}_\tau$ if $\exists wr_2 \in \events_\tau$ \st 
	($\mo{\tau}{wr^\sc_1}{wr_2}$ $\v$ $\hb{\tau}{wr^\sc_1}{wr_2}$
	$\v$ $\rf{\tau}{wr^\sc_1}{wr_2}$) $\^$ $\seqb{\tau}{wr_2}{f^\sc_1}$ 
	then $\so{\tau}{wr^\sc_1}{f^\sc_1}$ \\
	& ($\setHB$, $\setMO$, $\setRF$. $\setFR$ between \sc event and relaxed 
		event forms $\setSO$ with assistance of an appropriately placed 
		fence) \\
	
	\hl{soFE}: & $\forall f^\sc_1 \in \ordfences{\sc}_\tau$, $wr^\sc_2 \in
	\ordevents{\sc}_\tau$ if $\exists wr_1 \in \events_\tau$ \st 
	($\mo{\tau}{wr_1}{wr^\sc_2}$ $\v$ $\hb{\tau}{wr_1}{wr^\sc_2}$
	$\v$ $\rf{\tau}{wr_1}{wr^\sc_2}$) $\^$ $\seqb{\tau}{f^\sc_1}{wr_1}$ 
	then $\so{\tau}{f^\sc_1}{wr^\sc_2}$ \\
	& ($\setHB$, $\setMO$, $\setRF$. $\setFR$ between relaxed event and
	\sc event forms $\setSO$ with assistance of an appropriately placed 
	fence) \\
	
	\hl{soFF}: & $\forall f^{\sc}_1, f^{\sc}_2 \in \ordfences{\sc}_\tau$, 
	if $\exists wr_1, wr_2 \in \events_\tau$ \st 
	($\mo{\tau}{wr_1}{wr_2}$ $\v$ $\hb{\tau}{wr_1}{wr_2}$
	$\v$ $\rf{\tau}{wr_1}{wr_2}$) $\^$
	($\seqb{\tau}{f^{\sc}_1}{wr_1}$, $\seqb{\tau}{wr_2}{f^{\sc}_2}$) 
	then $\so{\tau}{f^{\sc}_1}{f^{\sc}_2}$ \\
	& ($\setHB$, $\setMO$, $\setRF$. $\setFR$ between relaxed events 
		forms $\setSO$ with assistance of appropriately placed fences) \\
	\hline
\end{longtable}

Consider a \cc trace $\tau$ and a transformation
$\inv{\tau}$ of $\tau$ \st $\events_{\inv{\tau}}$ = $\events_\tau$
$\union$ set of synthesized fences, then;
%

\begin{theorem} \label{thm:to-so}
	$\to{\inv{\tau}}{}{}$ = (transitive closure of $\so{\inv{\tau}}{}{}$).
\end{theorem}
%
Thus, using Theorem~\ref{thm:to-so} we can state that,
a cyclic $\so{\inv{\tau}}{}{}$ implies that there does not exist a 
total order on \sc ordered events of $\inv{\tau}$. The trace $\inv{\tau}$, 
then, is not a valid \cc trace and the buggy trace $\tau$ 
has been invalidated.
%
Thus, the aim of \sfence is to synthesis fences in the input program $P$
at appropriate locations and force a cyclic $\setSO$ order in the \sc 
ordered events of a buggy trace $\tau$ of $P$ and invalidating the trace 
for the transformed program $\fx{P}$.

\snj{sc fences alone ensures optimality inno of fences, str 2 needed for opt in type of fences.}
