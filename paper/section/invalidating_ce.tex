As discussed in Section~\ref{sec:c11}, in a valid trace of the
input program $P$ (including a buggy trace), 
the program events must satisfy the \hlref{coherence conditions}.
%
Contrarily, if we synthesize \cc fences in the input program 
such that the irreflexivity of any of the coherence conditions 
or the total order on \sc events in the buggy trace gets violated 
then we can invalidate the trace and stop the program behavior.

Assume a set of synthesized \cc fences $\overline{\fences_{\tau'}}$
in a transformation $\tau'$ of a buggy trace $\tau$. 
%
The synthesized fences inflate the 
$\setSB$, $\setSW$ and $\setDOB$ relation sets by adding 
relations between program events ($\events_\tau$) and newly 
added fences ($\sfences_{\tau'}$) to form corresponding 
$\seqb{\tau'}{}{}$, $\sw{\tau'}{}{}$ and $\dob{\tau'}{}{}$ 
relations (as explained in Section~\ref{sec:c11}). 
%
Further, the $\ithb{\tau'}{}{}$ relation contains  
program event pairs as a consequence of the freshly formed 
$\sw{\tau'}{}{}$ and $\dob{\tau'}{}{}$ relations in addition
to the event pairs in $\setITHB$.
%
The $\mo{\tau'}{}{}$, $\rf{\tau'}{}{}$ and $\fr{\tau'}{}{}$ 
relations remain the same as the corresponding relations of 
$\tau$, as fences do not contribute to these relations.

We propose two strategies, namely \stfence and \wkfence, to 
detect invalidation of the program trace $\tau$. 
%
The \wkfence strategy detects violation of a coherence condition.
If such a case exists then we can stop the behavior by
synthesizing weaker fences (memory orders \rel, \acq, \acqrel). 
%
However, if a coherence condition is not violated then we move
to \stfence strategy to break the total order requirement on
\sc ordered events and stop the behavior by synthesizing 
strong (\sc ordered) fences.
%
The strategies are discussed below.\newline

\begin{figure}[t]
	\begin{tabular}{|c|c|c|}
		\hline
		\resizebox{0.33\textwidth}{!}{\tikzset{every picture/.style={line width=0.75pt}} %set default line width to 0.75pt        
\begin{tikzpicture}[x=1em,y=1em,yscale=-1,xscale=-1]
	\tikzstyle{every node}=[font=\normalfont]
	
	\node (ifl1) [inner sep=2pt,color=Brown] {$\mathbb{I}(Flag_1,0)$};
	\node (ifl2) [right=30pt,inner sep=2pt,color=Brown] {$\mathbb{I}(Flag_2,0)$};
	
	\node (f11) [below left=10pt and -30pt of ifl1,inner sep=2pt,color=White] {$\mathbb{F}^\acqrel_{11}$};
	\node (fl1) [below=10pt of f11, inner sep=2pt] {$ W^\sc(Flag_1,1) $};
	\node (f12) [below=10pt of fl1, inner sep=2pt,color=White] {$\mathbb{F}^\acqrel_{12}$};
	\node (rfl2) [below=10pt of f12, inner sep=2pt] {if $R^\rlx(Flag_2,0)$:};
	\node (cs11) [below right=-1pt and -40pt of rfl2] {(critical};
	\node (cs12) [below right=-4pt and -40pt of cs11] {section1)};
	\node (f13) [below=35pt of rfl2, inner sep=2pt,color=White] {$\mathbb{F}^\acqrel_{13}$};
	
	\node (f21) [right=40pt of f11, inner sep=2pt,color=White] {$\mathbb{F}^\acqrel_{21}$};
	\node (fl2) [below=10pt of f21, inner sep=2pt] {$W^\sc(Flag_2,1)$};
	\node (f22) [below=10pt of fl2, inner sep=2pt,color=White] {$\mathbb{F}^\acqrel_{22}$};
	\node (rfl1) [below=10pt of f22, inner sep=2pt] {if $R^\rlx(Flag_1,0)$:};
	\node (cs21) [below right=-1pt and -40pt of rfl1] {(critical};
	\node (cs22) [below right=-4pt and -40pt of cs21] {section2)};
	\node (f23) [below=35pt of rfl1, inner sep=2pt,color=White] {$\mathbb{F}^\acqrel_{23}$};
	%
	\draw [->,>=stealth,color=RedOrange] ($ (ifl1.south east)+(1,-5pt) $) -- node[pos=0.3,left=-2pt,font=\scriptsize,color=black] { $\lmo$ } ($ (fl1.north east)+(0.5,-5pt) $);
	\draw [->,>=stealth,color=RedOrange] ($ (ifl2.south west)+(-0.9,-5pt) $) -- node[pos=0.3,right=-2pt,font=\scriptsize,color=black] { $\lmo$ } ($ (fl2.north west)+(-0.7,-5pt) $);
	
	\draw [->,>=stealth,color=CarnationPink] (fl1) -- node[midway,right=-2pt,font=\scriptsize,color=black] { $\lsb$ } (rfl2);
	\draw [->,>=stealth,color=CarnationPink] (fl2) -- node[midway,right=-2pt,font=\scriptsize,color=black] { $\lsb$ } (rfl1);
	
	\draw [->,>=stealth,color=PineGreen] ($ (ifl1.south east)+(1,-5pt) $) -- node[pos=0.8,right=-2pt,font=\scriptsize,color=black] { $\lrf$ } ($ (rfl1.north west)+(-1.2,-5pt) $);
	\draw [->,>=stealth,color=PineGreen] ($ (ifl2.south west)+(-0.9,-5pt) $) -- node[pos=0.8,left=-2pt,font=\scriptsize,color=black] { $\lrf$ } ($ (rfl2.north east)+(1.2,-5pt) $);
	
\end{tikzpicture}
} &
		\resizebox{0.33\textwidth}{!}{\tikzset{every picture/.style={line width=0.75pt}} %set default line width to 0.75pt        
\begin{tikzpicture}[x=1em,y=1em,yscale=-1,xscale=-1]
\tikzstyle{every node}=[font=\normalfont]

\node (ifl1) [inner sep=2pt,color=Brown] {$\mathbb{I}(Flag_1,0)$};
\node (ifl2) [right=30pt,inner sep=2pt,color=Brown] {$\mathbb{I}(Flag_2,0)$};

\node (f11) [below left=10pt and -30pt of ifl1,inner sep=2pt,color=Blue] {$\mathbb{F}^\acqrel_{11}$};
\node (fl1) [below=10pt of f11, inner sep=2pt] {$ W^\sc(Flag_1,1) $};
\node (f12) [below=10pt of fl1, inner sep=2pt,color=Blue] {$\mathbb{F}^\acqrel_{12}$};
\node (rfl2) [below=10pt of f12, inner sep=2pt] {if $R^\rlx(Flag_2,0)$:};
\node (cs11) [below right=-1pt and -40pt of rfl2] {(critical};
\node (cs12) [below right=-4pt and -40pt of cs11] {section1)};
\node (f13) [below=35pt of rfl2, inner sep=2pt,color=Blue] {$\mathbb{F}^\acqrel_{13}$};

\node (f21) [right=40pt of f11, inner sep=2pt,color=Blue] {$\mathbb{F}^\acqrel_{21}$};
\node (fl2) [below=10pt of f21, inner sep=2pt] {$W^\sc(Flag_2,1)$};
\node (f22) [below=10pt of fl2, inner sep=2pt,color=Blue] {$\mathbb{F}^\acqrel_{22}$};
\node (rfl1) [below=10pt of f22, inner sep=2pt] {if $R^\rlx(Flag_1,0)$:};
\node (cs21) [below right=-1pt and -40pt of rfl1] {(critical};
\node (cs22) [below right=-4pt and -40pt of cs21] {section2)};
\node (f23) [below=35pt of rfl1, inner sep=2pt,color=Blue] {$\mathbb{F}^\acqrel_{23}$};
%
\draw [->,>=stealth,color=RedOrange] ($ (ifl1.south east)+(1,-5pt) $) -- node[pos=0.3,left=-2pt,font=\scriptsize,color=black] { $\lmo$ } ($ (fl1.north east)+(0.5,-5pt) $);
\draw [->,>=stealth,color=RedOrange] ($ (ifl2.south west)+(-0.9,-5pt) $) -- node[pos=0.3,right=-2pt,font=\scriptsize,color=black] { $\lmo$ } ($ (fl2.north west)+(-0.7,-5pt) $);

\draw [->,>=stealth,color=CarnationPink] (f11) -- node[midway,right=-2pt,font=\scriptsize,color=black] { $\lsb$ } (fl1);
\draw [->,>=stealth,color=CarnationPink] (fl1) -- node[midway,right=-2pt,font=\scriptsize,color=black] { $\lsb$ } (f12);
\draw [->,>=stealth,color=CarnationPink] (f12) -- node[midway,right=-2pt,font=\scriptsize,color=black] { $\lsb$ } (rfl2);
\draw [->,>=stealth,color=CarnationPink] ($ (cs12.south west)+(-0.55,-5pt) $) -- node[midway,right=-2pt,font=\scriptsize,color=black] { $\lsb$ } (f13);

\draw [->,>=stealth,color=CarnationPink] (f21) -- node[midway,right=-2pt,font=\scriptsize,color=black] { $\lsb$ } (fl2);
\draw [->,>=stealth,color=CarnationPink] (fl2) -- node[midway,right=-2pt,font=\scriptsize,color=black] { $\lsb$ } (f22);
\draw [->,>=stealth,color=CarnationPink] (f22) -- node[midway,right=-2pt,font=\scriptsize,color=black] { $\lsb$ } (rfl1);
\draw [->,>=stealth,color=CarnationPink] ($ (cs22.south west)+(-0.55,-5pt) $) -- node[midway,right=-2pt,font=\scriptsize,color=black] { $\lsb$ } (f23);

\draw [->,>=stealth,color=PineGreen] ($ (ifl1.south east)+(1,-5pt) $) -- node[pos=0.8,right=-2pt,font=\scriptsize,color=black] { $\lrf$ } ($ (rfl1.north west)+(-1.2,-5pt) $);
\draw [->,>=stealth,color=PineGreen] ($ (ifl2.south west)+(-0.9,-5pt) $) -- node[pos=0.8,left=-2pt,font=\scriptsize,color=black] { $\lrf$ } ($ (rfl2.north east)+(1.2,-5pt) $);

\end{tikzpicture}
} &
		\resizebox{0.33\textwidth}{!}{\tikzset{every picture/.style={line width=0.75pt}} %set default line width to 0.75pt        
\begin{tikzpicture}[x=1em,y=1em,yscale=-1,xscale=-1]
	\tikzstyle{every node}=[font=\normalfont]
	
	\node (ifl1) [inner sep=2pt,color=Brown] {$\mathbb{I}(Flag_1,0)$};
	\node (ifl2) [right=30pt,inner sep=2pt,color=Brown] {$\mathbb{I}(Flag_2,0)$};
	
	\node (f11) [below left=10pt and -30pt of ifl1,inner sep=2pt,color=White] {$\mathbb{F}^\acqrel_{11}$};
	\node (fl1) [below=10pt of f11, inner sep=2pt] {$ W^\sc(Flag_1,1) $};
	\node (f12) [below=10pt of fl1, inner sep=2pt,color=Blue] {$\mathbb{F}^\sc_{1}$};
	\node (rfl2) [below=10pt of f12, inner sep=2pt] {if $R^\rlx(Flag_2,0)$:};
	\node (cs11) [below right=-1pt and -40pt of rfl2] {(critical};
	\node (cs12) [below right=-4pt and -40pt of cs11] {section1)};
	\node (f13) [below=35pt of rfl2, inner sep=2pt,color=White] {$\mathbb{F}^\acqrel_{13}$};
	
	\node (f21) [right=40pt of f11, inner sep=2pt,color=White] {$\mathbb{F}^\acqrel_{21}$};
	\node (fl2) [below=10pt of f21, inner sep=2pt] {$W^\sc(Flag_2,1)$};
	\node (f22) [below=10pt of fl2, inner sep=2pt,color=Blue] {$\mathbb{F}^\sc_{2}$};
	\node (rfl1) [below=10pt of f22, inner sep=2pt] {if $R^\rlx(Flag_1,0)$:};
	\node (cs21) [below right=-1pt and -40pt of rfl1] {(critical};
	\node (cs22) [below right=-4pt and -40pt of cs21] {section2)};
	\node (f23) [below=35pt of rfl1, inner sep=2pt,color=White] {$\mathbb{F}^\acqrel_{23}$};
	%
	\draw [->,>=stealth,color=RedOrange,opacity=0.3] ($ (ifl1.south east)+(1,-5pt) $) -- node[pos=0.3,left=-2pt,font=\scriptsize,color=black] { $\lmo$ } ($ (fl1.north east)+(0.5,-5pt) $);
	\draw [->,>=stealth,color=RedOrange,opacity=0.3] ($ (ifl2.south west)+(-0.9,-5pt) $) -- node[pos=0.3,right=-2pt,font=\scriptsize,color=black] { $\lmo$ } ($ (fl2.north west)+(-0.7,-5pt) $);
	
	\draw [->,>=stealth,color=Mahogany] (fl1) -- node[midway,right=-2pt,font=\scriptsize,color=black] { $\lso$ } (f12);
	\draw [->,>=stealth,color=CarnationPink,opacity=0.3] (f12) -- node[midway,right=-2pt,font=\scriptsize,color=black] { $\lsb$ } (rfl2);
	
	\draw [->,>=stealth,color=Mahogany] (fl2) -- node[midway,right=-2pt,font=\scriptsize,color=black] { $\lso$ } (f22);
	\draw [->,>=stealth,color=CarnationPink,opacity=0.3] (f22) -- node[midway,right=-2pt,font=\scriptsize,color=black] { $\lsb$ } (rfl1);
	
	\draw [->,>=stealth,color=Mahogany] (f12) -- node[pos=0.2,below=-2pt,font=\scriptsize,color=black] { $\lso$ } (fl2);
	\draw [->,>=stealth,color=Mahogany] (f22) -- node[pos=0.2,below=-2pt,font=\scriptsize,color=black] { $\lso$ } (fl1);
	
	\draw [->,>=stealth,color=PineGreen,opacity=0.3] ($ (ifl1.south east)+(1,-5pt) $) -- node[pos=0.8,right=-2pt,font=\scriptsize,color=black] { $\lrf$ } ($ (rfl1.north west)+(-1.2,-5pt) $);
	\draw [->,>=stealth,color=PineGreen,opacity=0.3] ($ (ifl2.south west)+(-0.9,-5pt) $) -- node[pos=0.8,left=-2pt,font=\scriptsize,color=black] { $\lrf$ } ($ (rfl2.north east)+(1.2,-5pt) $);
	
\end{tikzpicture}
} \\
		\hline
		\multicolumn{1}{c}{\hl{mutex}} &
		\multicolumn{1}{c}{\hl{mutex-fences}} &
		\multicolumn{1}{c}{\hl{mutex-invalidated}} 
	\end{tabular}
\end{figure}

\noindent
{\bf Invalidate by violating coherence condition (\wkfence)}\newline
The \wkfence strategy simply attempts to detect a cycle in 
one of the \hlref{coherence conditions} listed in 
Section~\ref{sec:c11}.
%
Consider the example \hlref{WRIR}. Since, the trace does not 
violate any coherence condition it is a valid trace under \cc. 
Assume that the trace shown in the figure is a counter example.
%
Synthesizing release fence $\mathbb{F^\rel}$ and acquire fence 
$\mathbb{F^\acq}$ as shown in \hlref{WRIR-invalidated} forms
a cycle in $\setMO;\setRF;\setHB;\setRF^{-1}$ thus violating the
coherence of the trace.
%
\begin{figure}[h]
	\begin{tabular}{|c|c|}
		\hline
		\resizebox{0.5\textwidth}{!}{\tikzset{every picture/.style={line width=0.75pt}} %set default line width to 0.75pt        
\begin{tikzpicture}[x=1em,y=1em,yscale=-1,xscale=-1]
\tikzstyle{every node}=[font=\normalfont]
\node (ix) [inner sep=2pt,color=Brown] {$\mathbb{I}(x,0)$};
\node (iy) [right=30pt of ix, inner sep=2pt,color=Brown] {$\mathbb{I}(y,0)$};
\node (wx) [below left=15pt and -5pt of ix, inner sep=2pt] {$ W^\rlx(x,1)$};
\node (rx1) [right=20pt of wx, inner sep=2pt] {$R^\rlx(x,1)$};
\node (wy) [below=25pt of rx1, inner sep=2pt] {$W^\rlx(y,1)$};
\node (ry) [right=20pt of rx1, inner sep=2pt] {$R^\rlx(y,1)$};
\node (rx0) [below=25pt of ry, inner sep=2pt] {$R^\rlx(x,0)$};
%
\draw [->,>=stealth,color=RedOrange] (ix) -- node[pos=0.3,left=-1pt,font=\scriptsize,color=black] { $\lmo$ } (wx);
\draw [->,>=stealth,color=RedOrange] (iy.south) to[in=-135,out=90] node[pos=0.1,left=-2pt,font=\scriptsize,color=black] { $\lmo$ } (wy);
\draw [->,>=stealth,color=PineGreen] (wx) -- node[midway,above=-2pt,font=\scriptsize,color=black] { $\lrf$ } (rx1);
\draw [->,>=stealth,color=CarnationPink] (rx1) -- node[midway,left=-2pt,font=\scriptsize,color=black] { $\lsb$ } (wy);
\draw [->,>=stealth,color=PineGreen] (wy) -- node[midway,above=-2pt,font=\scriptsize,color=black] { $\lrf$ } (ry);
\draw [->,>=stealth,color=CarnationPink] (ry) -- node[midway,left=-2pt,font=\scriptsize,color=black] { $\lsb$ } (rx0);

\draw [opacity=0.3] (-0.1,1.5) -- (-0.1,7);
\draw [opacity=0.3] (0.1,1.5) -- (0.1,7);

\draw [opacity=0.3] (-7.3,1.5) -- (-7.3,7);
\draw [opacity=0.3] (-7.5,1.5) -- (-7.5,7);

\end{tikzpicture}
} &
		\resizebox{0.5\textwidth}{!}{\tikzset{every picture/.style={line width=0.75pt}} %set default line width to 0.75pt        
\begin{tikzpicture}[x=1em,y=1em,yscale=-1,xscale=-1]
\tikzstyle{every node}=[font=\normalfont]

\node (ix) [inner sep=2pt,color=Brown] {$\mathbb{I}(x,0)$};
\node (iy) [right=30pt of ix, inner sep=2pt,color=Brown] {$\mathbb{I}(y,0)$};
\node (wx) [below left=10pt and -5pt of ix, inner sep=2pt] {$ W^\rlx(x,1)$};
\node (rx1) [right=20pt of wx, inner sep=2pt] {$R^\rlx(x,1)$};
\node (frel) [below=10pt of rx1, inner sep=2pt,color=Blue] {$\mathbb{F}^\rel$};
\node (wy) [below=10pt of frel, inner sep=2pt] {$W^\rlx(y,1)$};
\node (ry) [right=20pt of rx1, inner sep=2pt] {$R^\rlx(y,1)$};
\node (facq) [below=10pt of ry, inner sep=2pt,color=Blue] {$\mathbb{F}^\acq$};
\node (rx0) [below=10pt of facq, inner sep=2pt] {$R^\rlx(x,0)$};
%
\draw [->,>=stealth,color=RedOrange] (ix) -- node[pos=0.3,left=-1pt,font=\scriptsize,color=black] { $\lmo$ } (wx);
\draw [->,>=stealth,color=RedOrange,opacity=0.4] (iy.south) to[in=-135,out=90] node[pos=0.1,left=-2pt,font=\scriptsize,color=black] { $\lmo$ } (wy);
\draw [->,>=stealth,color=PineGreen] (wx) -- node[midway,above=-2pt,font=\scriptsize,color=black] { $\lrf$ } (rx1);
\draw [->,>=stealth,color=CarnationPink] (rx1) -- node[pos=0.3,left=-2pt,font=\scriptsize,color=black] { $\lsb$ } (frel);
\draw [->,>=stealth,color=CarnationPink,opacity=0.4] (frel) -- node[pos=0.3,left=-2pt,font=\scriptsize,color=black] { $\lsb$ } (wy);
\draw [->,>=stealth,color=PineGreen,opacity=0.4] (wy) -- node[pos=0.8,above=-2pt,font=\scriptsize,color=black] { $\lrf$ } (ry);
\draw [->,>=stealth,color=CarnationPink,opacity=0.4] (ry) -- node[pos=0.3,left=-2pt,font=\scriptsize,color=black] { $\lsb$ } (facq);
\draw [->,>=stealth,color=CarnationPink] (facq) -- node[pos=0.3,left=-2pt,font=\scriptsize,color=black] { $\lsb$ } (rx0);
\draw [->,>=stealth,color=Magenta] (frel) -- node[midway,above=-2pt,font=\scriptsize,color=black] { \lsw } (facq);
\draw [color=PineGreen] (-10.2,8.3) -- (-10.2,8.7);
\draw [color=PineGreen] (6.5,8.7) -- (-10.2,8.7);
\draw [color=PineGreen] (6.5,0) -- (6.5,8.7);
\draw [->,>=stealth,color=PineGreen] (6.5,0) -- node[midway,below=-2pt,font=\scriptsize,color=black] { $\lrf^{-1}$ } (2,0);

\draw [opacity=0.3] (-0.1,2) -- (-0.1,8);
\draw [opacity=0.3] (0.1,2) -- (0.1,8);

\draw [opacity=0.3] (-7.3,2) -- (-7.3,8);
\draw [opacity=0.3] (-7.5,2) -- (-7.5,8);

\end{tikzpicture}
} \\
		\hline
		\multicolumn{1}{c}{\hl{WRIR}} &
		\multicolumn{1}{c}{\hl{WRIR-invalidated}}
	\end{tabular}
\end{figure}

However, inserting fences in a buggy trace may not be sufficient for 
violating a coherence condition. Consider example \hlref{mutex} and
its transformation \hlref{mutex-fences} with synthesized fences. 
%
As can be seen from \hlref{mutex-fences}, even after inserting fences 
at every location in the buggy trace, the trace still satisfies all 
coherence conditions and is thus valid.
\newline

\noindent
{\bf Invalidating by violating \sc total-order (\stfence)}\newline
If the \wkfence strategy fails to invalidate the buggy trace then
we attempt to violate the total-order on \sc events.
%
Since the coherence condition did on violate on the events of $\tau'$ 
it implies that conditions do not violate on the \sc ordered events
of $\tau'$ either. 
%
As a result in the \stfence strategy we attempt to violate the 
irreflexivity condition on 
$(\onsc{\hb{\tau'}{}{}}$ $\union$ $\onsc{\mo{\tau'}{}{}}$ $\union$ 
$\onsc{\rf{\tau'}{}{}}$ $\union$ $\onsc{\fr{\tau'}{}{}})^+$. 

We introduce a possibly reflexive relation on \sc-ordered
events of $\tau'$, called {\em \sc-order} $(\so{\tau'}{}{})$ to 
capture the ordering between the \sc events. 
%
\begin{definition}{\bf \sc-order ($\so{\tau'}{}{}$)}\newline
	$\forall e_1, e_2 \in \events_\tau$ \st 
	$(e_1,e_2) \in$ $\setHB$ $\union$ $\setMO$ $\union$ $\setRF$ 
	$\union$ $\setFR$
	
	if
	$e_1, e_2 \in \ordevents{\sc}_\tau$ then 
	$\so{\tau'}{e_1}{e_2}$;
	
	if
	$e_1 \in \ordevents{\sc}_\tau$, 
	$\exists \mathbb{F}^\sc \in \ordsfences{\sc}_{\tau'}$ where
	$\seqb{\tau'}{e_2}{\mathbb{F}^\sc}$ then
	$\so{\tau'}{e_1}{\mathbb{F}^\sc}$;
	
	if
	$e_2 \in \ordevents{\sc}_\tau$, 
	$\exists \mathbb{F}^\sc \in \ordsfences{\sc}_{\tau'}$ where
	$\seqb{\tau'}{\mathbb{F}^\sc}{e_1}$ then
	$\so{\tau'}{\mathbb{F}^\sc}{e_2}$;
	
	if
	$\exists \mathbb{F}^\sc_1$, $\mathbb{F}^\sc_2$ 
	$\in \ordsfences{\sc}_{\tau'}$ where
	$\seqb{\tau'}{\mathbb{F}^\sc_1}{e_1}$ and 
	$\seqb{\tau'}{e_2}{\mathbb{F}^\sc_2}$ then
	$\so{\tau'}{\mathbb{F}^\sc_1}{\mathbb{F}^\sc_1}$.
\end{definition}
%
Note that, $\setSO \subseteq \setTO$ for any trace $\tau$. It
does not contain pairs of \sc events that do not have a definite
order. 
%
Consider the example \hlref{WRWR}, $\to{}{W^\sc(x,1)}{W^\sc(y,1)}$
and $\to{}{W^\sc(y,1)}{W^\sc(x,1)}$ are both valid total-orders
on the \sc events of the trace.
%
The set $\setSO$ does not contain either of the two cases and 
would be empty for this example.
%
Such a candidate pair of events cannot contribute to the 
reflexivity of $\setSO$ and can be safely ignored for the
purpose of this work. 
\setlength{\textfloatsep}{0pt}
%\begin{wrapfigure}{l}{0.3\textwidth}
%	\vspace{-2em}
%	\begin{tabular}{|c|}
%		\hline
%		\resizebox{0.3\textwidth}{!}{\tikzset{every picture/.style={line width=0.75pt}} %set default line width to 0.75pt        
\begin{tikzpicture}[x=1em,y=1em,yscale=-1,xscale=-1]
\tikzstyle{every node}=[font=\normalfont]
\node (fl1) [inner sep=2pt] {$ W^\sc(x,1) 1 $};
\node (fl2) [right=35pt of fl1, inner sep=2pt] {$W^\sc(y,1)$};
\node (rfl2) [below=22pt of fl1, inner sep=2pt] {if $R^\rlx(y,0)$:};
\node (rfl1) [below=22pt of fl2, inner sep=2pt] {if $R^\rlx(x,0)$:};
%
\draw [->,>=stealth,color=CarnationPink] (fl1) -- node[midway,right=-2pt,font=\scriptsize,color=black] { $\lsb$ } (rfl2);
\draw [->,>=stealth,color=CarnationPink] (fl2) -- node[midway,left=-2pt,font=\scriptsize,color=black] { $\lsb$ } (rfl1);

\end{tikzpicture}
} \\
%		\hline
%		\multicolumn{1}{c}{\hl{WRWR}}
%	\end{tabular}
%\end{wrapfigure}
\begin{figure}[h]
	\begin{tabular}{|c|}
		\hline
		\resizebox{0.3\textwidth}{!}{\tikzset{every picture/.style={line width=0.75pt}} %set default line width to 0.75pt        
\begin{tikzpicture}[x=1em,y=1em,yscale=-1,xscale=-1]
\tikzstyle{every node}=[font=\normalfont]
\node (fl1) [inner sep=2pt] {$ W^\sc(x,1) 1 $};
\node (fl2) [right=35pt of fl1, inner sep=2pt] {$W^\sc(y,1)$};
\node (rfl2) [below=22pt of fl1, inner sep=2pt] {if $R^\rlx(y,0)$:};
\node (rfl1) [below=22pt of fl2, inner sep=2pt] {if $R^\rlx(x,0)$:};
%
\draw [->,>=stealth,color=CarnationPink] (fl1) -- node[midway,right=-2pt,font=\scriptsize,color=black] { $\lsb$ } (rfl2);
\draw [->,>=stealth,color=CarnationPink] (fl2) -- node[midway,left=-2pt,font=\scriptsize,color=black] { $\lsb$ } (rfl1);

\end{tikzpicture}
} \\
		\hline
		\multicolumn{1}{c}{\hl{WRWR}}
	\end{tabular}
\end{figure}

Consider again the example \hlref{mutex} without synthesized fences.
Upon synthesizing \sc fences $\mathbb{F}^\sc_1$ and 
$\mathbb{F}^\sc_2$ as shown in \hlref{mutex-invalidated} the
transformed trace forms the following $\so{}{}{}$ relations,

$\seqb{}{W^\sc(Flag_1,1)}{\mathbb{F}^\sc_1}$ $\implies$ $\so{}{W^\sc(Flag_1,1)}{\mathbb{F}^\sc_1}$

$\seqb{}{W^\sc(Flag_2,1)}{\mathbb{F}^\sc_2}$ $\implies$ $\so{}{W^\sc(Flag_2,1)}{\mathbb{F}^\sc_2}$

$\fr{}{R^\rlx(Flag_1,0)}{W^\sc(Flag_1,1)}$ $\implies$ $\so{}{\mathbb{F}^\sc_2}{W^\sc(Flag_1,1)}$

$\fr{}{R^\rlx(Flag_2,0)}{W^\sc(Flag_2,1)}$ $\implies$ $\so{}{\mathbb{F}^\sc_1}{W^\sc(Flag_2,1)}$,

thus violating the total-order requirement on \sc events and
invalidating the trace with strong fences.

\ourtechnique, thus, attempts to stop a buggy trace from
manifesting as an execution by synthesizing fences to 
invalidate the trace, by using either the \wkfence or 
\stfence strategy.
%
However, 
as discussed in Section~\ref{sec:c11} \cc fences are weak
fences.
%
It is noteworthy that introducing \cc fences in a program 
with weakly ordered events may not impose strong enough
reordering restrictions as strongly ordered events would.
%
As a consequence, it may be possible to stop a buggy 
trace by modifying the memory orders of program events
to stricter orders but fail to stop the same by 
inserting even the strictest \cc fences.

\begin{figure}[h]
	\begin{tabular}{|c|c|}
		\hline
		\resizebox{0.49\textwidth}{!}{\tikzset{every picture/.style={line width=0.75pt}} %set default line width to 0.75pt        
\begin{tikzpicture}[x=1em,y=1em,yscale=1,xscale=1]
\tikzstyle{every node}=[font=\small]

% other nodes
\node (init) {$Initially$, $x=0$, $y=0$};
\node (f11) [below left=0pt and 20pt of init,color=White] { $F^\sc_{11}$ };
\node (wx1) [below=8pt of f11] {$ W^\sc(x,1) $};
\node (f12) [below=8pt of wx1,color=White] { $F^\sc_{12}$ };

\node (f21) [right=30pt of f11,color=White] { $F^\sc_{21}$ };
\node (rx1) [below=8pt of f21] {if $ R^\sc(x,1) $};
\node (f22) [below=8pt of rx1,color=White] { $F^\sc_{22}$ };
\node (ry0) [below=8pt of f22] {\tab$ R^\sc(y,0) $};
\node (f23) [below=8pt of ry0,color=White] { $F^\sc_{23}$ };

\node (f31) [right=30pt of f21,color=White] { $F^\sc_{31}$ };
\node (wy1) [below=8pt of f31] {$ W^\sc(y,1) $};
\node (f32) [below=8pt of wy1,color=White] { $F^\sc_{32}$ };

\node (f41) [right=30pt of f31,color=White] { $F^\sc_{41}$ };
\node (ry1) [below=8pt of f41] {if $ R^\sc(y,1) $};
\node (f42) [below=8pt of ry1,color=White] { $F^\sc_{42}$ };
\node (rx0) [below=8pt of f42] {\tab$ R^\sc(x,0) $};
\node (f43) [below=8pt of rx0,color=White] { $F^\sc_{43}$ };

\draw [->,>=stealth,color=Brown] (rx1) -- (ry0);
\draw [->,>=stealth,color=Brown] (ry1) -- (rx0);

\draw [->,>=stealth,color=Brown] ($ (wx1.east)+(-3pt,0) $) -- ($ (rx1.west)+(3pt,0) $);
\draw [->,>=stealth,color=Brown] (ry0) -- (wy1);
\draw [->,>=stealth,color=Brown] ($ (wy1.east)+(-3pt,0) $) -- ($ (ry1.west)+(3pt,0) $);
\draw [color=Brown] (rx0.south)+(0,3pt) -- ($ (rx0.south)+(0,-10pt) $);
\draw [color=Brown] ($ (rx0.south)+(0,-10pt) $) -- ($ (wx1.south)+(0,-56.85pt) $);
\draw [->,>=stealth,color=Brown] ($ (wx1.south)+(0,-56.85pt) $) -- (wx1);

\draw [gray,opacity=0.5] (-6.2,-1.0) -- (-6.2,-12.5);
\draw [gray,opacity=0.5] (-6.0,-1.0) -- (-6.0,-12.5);

\draw [gray,opacity=0.5] (-0.1,-1.0) -- (-0.1,-12.5);
\draw [gray,opacity=0.5] (0.1,-1.0) -- (0.1,-12.5);

\draw [gray,opacity=0.5] (5.0,-1.0) -- (5.0,-12.5);
\draw [gray,opacity=0.5] (5.2,-1.0) -- (5.2,-12.5);

\end{tikzpicture}
} &
		\resizebox{0.49\textwidth}{!}{\tikzset{every picture/.style={line width=0.75pt}} %set default line width to 0.75pt        
\begin{tikzpicture}[x=1em,y=1em,yscale=1,xscale=1]
\tikzstyle{every node}=[font=\small]

% other nodes
\node (init) {$Initially$, $x=0$, $y=0$};
\node (f11) [below left=0pt and 20pt of init,color=Blue] { $\mathbb{F}^\sc_{11}$ };
\node (wx1) [below=8pt of f11] {$ W^\rlx(x,1) $};
\node (f12) [below=8pt of wx1,color=Blue] { $\mathbb{F}^\sc_{12}$ };

\node (f21) [right=30pt of f11,color=Blue] { $\mathbb{F}^\sc_{21}$ };
\node (rx1) [below=8pt of f21] {if$ R^\rlx(x,1) $};
\node (f22) [below=8pt of rx1,color=Blue] { $\mathbb{F}^\sc_{22}$ };
\node (ry0) [below=8pt of f22] {\tab$ R^\rlx(y,0) $};
\node (f23) [below=8pt of ry0,color=Blue] { $\mathbb{F}^\sc_{23}$ };

\node (f31) [right=30pt of f21,color=Blue] { $\mathbb{F}^\sc_{31}$ };
\node (wy1) [below=8pt of f31] {$ W^\rlx(y,1) $};
\node (f32) [below=8pt of wy1,color=Blue] { $\mathbb{F}^\sc_{32}$ };

\node (f41) [right=30pt of f31,color=Blue] { $\mathbb{F}^\sc_{41}$ };
\node (ry1) [below=8pt of f41] {if$ R^\rlx(y,1) $};
\node (f42) [below=8pt of ry1,color=Blue] { $\mathbb{F}^\sc_{42}$ };
\node (rx0) [below=8pt of f42] {\tab$ R^\rlx(x,0) $};
\node (f43) [below=8pt of rx0,color=Blue] { $\mathbb{F}^\sc_{43}$ };

\draw [->,>=stealth,color=Brown] (f11) -- (wx1);
\draw [->,>=stealth,color=Brown] (wx1) -- (f12);

\draw [->,>=stealth,color=Brown] (f21) -- (rx1);
\draw [->,>=stealth,color=Brown] (rx1) -- (f22);
\draw [->,>=stealth,color=Brown] (f22) -- (ry0);
\draw [->,>=stealth,color=Brown] (ry0) -- (f23);

\draw [->,>=stealth,color=Brown] (f31) -- (wy1);
\draw [->,>=stealth,color=Brown] (wy1) -- (f32);

\draw [->,>=stealth,color=Brown] (f41) -- (ry1);
\draw [->,>=stealth,color=Brown] (ry1) -- (f42);
\draw [->,>=stealth,color=Brown] (f42) -- (rx0);
\draw [->,>=stealth,color=Brown] (rx0) -- (f43);

\draw [->,>=stealth,color=Brown] (f11) -- (f22);
\draw [->,>=stealth,color=Brown] (f22) -- (f32);
\draw [->,>=stealth,color=Brown] (f31) -- (f42);
\draw [color=Brown] (f42.east)+(0,0) -- ($ (f42.east)+(20pt,0) $);
\draw [color=Brown] (f42.east)+(20pt,0) -- ($ (f43.east)+(20pt,-7pt) $);
\draw [color=Brown] ($ (f43.east)+(20pt,-7pt) $) -- ($ (f12.south)+(0,-46.5pt) $);
\draw [->,>=stealth,color=Brown] ($ (f12.south)+(0,-46.5pt) $) -- (f12);

\draw [gray,opacity=0.5] (-6.2,-1.0) -- (-6.2,-12.5);
\draw [gray,opacity=0.5] (-6.0,-1.0) -- (-6.0,-12.5);

\draw [gray,opacity=0.5] (-0.1,-1.0) -- (-0.1,-12.5);
\draw [gray,opacity=0.5] (0.1,-1.0) -- (0.1,-12.5);

\draw [gray,opacity=0.5] (5.0,-1.0) -- (5.0,-12.5);
\draw [gray,opacity=0.5] (5.2,-1.0) -- (5.2,-12.5);

\end{tikzpicture}
} \\
		\hline
		\multicolumn{1}{c}{\hl{iriw-invalid}} & 
		\multicolumn{1}{c}{\hl{iriw-valid}}
	\end{tabular}
\end{figure}
The examples \hlref{iriw-invalid} and \hlref{iriw-valid} 
highlight the difference. As we can see in \hlref{iriw-valid},
the trace cannot be invalidated by \cc fences but the same 
can be achieved by changing the memory order of read and 
write events as shown in \hlref{iriw-invalid}.
%
%Consider the example \hlref{iriw-invalid}, the trace shown in
%figure is not a valid \cc trace as we cannot form a total 
%order on the \sc program events that agrees with the other event
%relations. The Figure~\hlref{iriw-valid} shows a version 
%of the example where the read and write events are \rlx ordered,
%under such a setting there is no possible placement of 
%program fences that can stop the trace.
%
\ourtechnique attempts to invalidate traces by synthesizing 
\cc fences, thus, buggy
traces such as \hlref{iriw-valid}, without the fences shown 
in the figure, cannot be stopped by \ourtechnique.