\par
The tool is created using the python programming language. Extra libraries used:
\begin{itemize}
	\item networkx - for graphing related queries, such as find cycles
	\item ast - for evaluating values from store
	\item operator
	\item shlex
	\item argparse
	\item os
\end{itemize}

\subsection{Translating model checker output}
The tool begins by translating model checker output that it obtains from the terminal process after running the required command. From CDS Checker output, it extracts the following information for each instruction in the trace:
\begin{enumerate}
	\item thread number of the instruction
	\item instruction type (thread create, thread start, init, atomic read, atomic write, atomic rmw)
	\item memory order (relaxed, acquire, release, acq\textunderscore rel, seq\textunderscore cst)
	\item memory address (variable name in the system)
	\item value
	\item rf value (in case of read/rmw instruction)
\end{enumerate}

\par 
Other required information is extracted from the C++ source program. These include:
\begin{enumerate}[7.]
	\item variable name for each instruction
	\item line number for each instruction
\end{enumerate}

\par
For any of the above meta-data, if a field does not apply for a certain instruction, it is given the value "NA".

\subsubsection{Extracting line numbers}
Line numbers are extracted by pattern-matching the instruction meta-data in the above mentioned fields, creating the instruction out of them and matching them to the source thread. For example, the information extracted for a single instruction is: thread number 2, instruction type read, memory order seq\_ cst, memory address \textit{x}. Then the instruction would be created as \texttt{x.load(memory\_ order\_ seq\_ cst)} and be matched inside the function defined for thread 2. There are also contingencies for this procedure, so certain rules need to be followed in the input program.

\par
Since line numbers are required to be extracted from the source program in order to obtain the locations where the fences need to be inserted, the source program needs to follow a certain structure for the tool to work. The rules are:
\begin{enumerate}
	\item Each atomic operation needs to be on a separate line.\\
	\textit{Example:} \texttt{a.store(b.load(memory\_ order\_ relaxed), memory\_ order\_ relaxed));} needs to be re-written as the following snippet:\\
	\texttt{%
	int temp = b.load(memory\_ order\_ relaxed);\\
	a.store(temp, memory\_ order\_ relaxed));%
	}
	unless it is an argument inside (say) an if statement with boolean operators like \texttt{if(a.load(memory\_ order\_ relaxed) || b.load(memory\_ order\_ relaxed)) }\\
	This is because each atomic access is treated as a separate entity in the model checker's output.
	
	\item Each atomic variable used in the program needs to be initialized in the main/parent thread. The tool maps the variable name to its memory address through the initialized values.
	
	\item While writing a "store" operation, there needs to be a space character after the comma after the value is written.\\
	\textit{Example:} \texttt{a.store(0, memory\_ order\_ relaxed)} and not \texttt{a.store(0,memory\_ order\_ relaxed)}
	
	\item The thread functions need to be defined in the same order as they were called in the main/parent thread before the main thread and there can be no other extra functions apart from the thread functions and the main function.
	
	\item The starting bracket for each function definition, loop call, or if-else statement needs to be in the same line as the function name.\\
	\textit{Example:} \texttt{static void t1() \{} and not\\ \texttt{static void t1()\\ \{}
	
	\item \textbf{\textit{Handing RMW instructions:}} since RMW instructions can be of any type (such as fetch\_ add, fetch\_ sub, etc) and still be just referred to as "atomic rmw" in the model checker, it can be hard to pattern-match these lines. Therefore another technique is used for pattern matching here. A "dummy variable" is created and a "store" operation is added to the line right above the RMW instruction. The value that it is storing needs to differ each time it is called in order to maintain uniqueness. Then the tool simply pattern-matches the instruction right above the RMW in the given thread and finds the line by adding "1" to the line number of the dummy variable store operation.\\
	\textit{Example:}\\ \texttt{[14] dummy.store(2, memory\_ order\_ relaxed);}\\ \texttt{[15] x.fetch\_ xor(1, memory\_ order\_ acq\_ rel);}\\
	This method also does not hinder the flow of the program since the operation on the dummy variable is a store operation which is never read, hence never forming an rf relation. The memory order will also never be "seq\_ cst", so it will never be considered while calculating TO relations.
	
	\item The above method is also to be used every time there is a repeated instruction in the same thread. For example if the instruction \texttt{b.load(memory\_ order\_ relaxed);} is called twice inside a single thread, pattern-matching this line would not be able to point to the instruction which called it at that point. Therefore, a dummy store operation needs to be inserted above every one of these repeating operations.\\\textit{Note:} only if the exact instruction matches does the dummy method need to be applied. If the memory orders or store values are different, then the instructions become different.
	
	% also add dummy variable store operation above a store instruction whose valule is another variable like a.store(temp, mo_relaxed); since "temp" will not be given by CDS Checker and only the value can be matched.
	
	\item "thread create" statements inside loops need to be unravelled and their functions need to be re-written as many times as they are called, with different names. This is because if there are multiple thread spawns from the a function using a loop, the number of threads will differ for each trace in the model checker. The model checker will treat them as different thread functions instead of the same one being called. Moreover, there might be cases where fences need to be inserted in the thread in one of the iterations of the loop but not the other.\\
	\textit{Example:}\\ \texttt{%
	static void thread1();\\
	int main() \{\\
	\hspace{10mm} for(3 iterations) \{\\
	thread\_ create(thread1);\\
	\}\}%
	}\\ needs to be re-written as:\\
	\texttt{%
	static void thread1();\\
	static void thread2();\\
	static void thread3();\\
	int main() \{\\
	thread\_ create(thread1);\\
	thread\_ create(thread2);\\
	thread\_ create(thread3);\\
	\}%
	}
	
	\item 'IF' and 'FOR' statements always need to be bracketed in case there is a fence insertion taking place inside.
\end{enumerate}
