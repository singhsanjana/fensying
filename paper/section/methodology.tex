Given a buggy input program $P$, \ourtechnique attempts to 
stop the buggy traces or counter examples (traces with 
assert statement violations) of $P$ by inserting \cc fences.
%
To do so the technique must accomplish three objectives
(O1) determine whether the buggy trace can be stopped by 
synthesizing \cc fences,
(O2) determine the placement of optimal number of synthesized 
fences (\ie the least number of program locations where fences 
must be synthesized that is sufficient to stop the trace), 
and
(O3) determine the optimal memory order of the synthesized 
fences (\ie the weakest memory order of synthesized fences 
that is sufficient to stop the trace).
%
We present the \ourtechnique-algorithm (Algorithm
\ref{alg:main algo}) that realizes the three objectives.
The algorithm takes a \cc program as input
and determines the optimal fence placement that can stop
the buggy traces of the input program or determines that
the program cannot be made bug-free with \cc fences.

\begin{algorithm}[h]
	\caption{Fence Synthesis}
	\label{alg:main algo}
	\DontPrintSemicolon
	\SetAlgoLined
	
	\SetKwFunction{Fmain}{\ourtechnique}
	\SetKwFunction{Fceg}{generateCounterExamples}
	\SetKwFunction{FcandidateF}{candidateFences}
	\SetKwFunction{Frel}{computeRelations}
	\SetKwFunction{Fwk}{weakFensying}
	\SetKwFunction{Fst}{strongFensying}
	\SetKwFunction{Fmin}{minModel}
	\SetKwFunction{Fmo}{assignMO}
	
	\SetKwData{satquery}{$\Phi$}
	\SetKwData{minmodel}{min$\Phi$}
	\SetKwData{cyclesall}{$\mathcal{C}$}
	\SetKwData{ce}{CE}
	\SetKwData{wk}{weakCycles}
	\SetKwData{st}{strongCycles}
	
	\SetKwProg{Fn}{Function}{:}{}
	
	\Fn{\Fmain{input program $P$}}{		
		\satquery $:=$ $\top$; \cyclesall $:= \emptyset$ \;
		\ce $:=$ \Fceg{$P$}\;
		\ForAll(\tcc*[f]{$\tau = \langle \events_\tau, \setHB, \setMO, \setRF \rangle$}) {$\tau \in$ \ce}{
			$\events_{\imm{\tau}}$ $:=$ $\events_\tau$ $\union$ \FcandidateF{$\tau$}\;
			$(\hb{\imm{\tau}}{}{}, \mo{\imm{\tau}}{}{}, \rf{\imm{\tau}}{}{}, \rfinv{\imm{\tau}}{}{}, \fr{\imm{\tau}}{}{})$ 
				$:=$ \Frel{$\tau,\events_{\imm{\tau}}$}\;
			\wk $:=$ \Fwk{$\imm{\tau}$}\;
			\st $:=$ \Fst{$\imm{\tau}$}\;
			\If{\wk $= \emptyset$ $\^$ \st $= \emptyset$} {
				\KwRet $\emptyset$
				\tcc*{{\tt ABORT}: cannot stop $\tau$ with \cc fences}
			}
			\satquery $:=$ \satquery $\^$ $\formula{$\wk $\v$ \st$}$\;
			\cyclesall $:=$ \cyclesall $\union$ \wk $\union$ \st \;
		}
	
		\minmodel $:=$ \Fmin{\satquery}\;
		\KwRet \Fmo{\minmodel, \cyclesall}
	}
		
		
%		\State 
%		\State $ \seqb{\imm{\tau}}{}{} := $ computeSB($\setSB, \events_{\imm{\tau}}$) \State $ \so{\tau^{\mathtt{im}}}{}{} := $ computeSO($\events_{\imm{\tau}}, \setHB, \setMO, \setRF, \seqb{\imm{\tau}}{}{}$)
%		\State cycles := computeCycles($ \so{\imm{\tau}}{}{} $)
%		\If {cycles == $ \emptyset $}
%		\State \texttt{Abort} (``This trace can't be stopped using \cc fences.'')
%		\State \Return
%		\EndIf
%		\State $\phi := \phi\ \^ \formula{\so{\imm{\tau}}{}{}} $ 
%		%			\State $ \phi := \phi_\tau $
%		\EndFor
%		\State F:= MinModel($ \phi $)
%		\State \Return F



%		% no enabled events left ie maximal sequence explored
%		\lIf{\FunexploredEv{$\tau$} = $\emptyset$}{\KwRet 
%			\tcc*[f]{maximal sequence explored}}
%		
%		% if there is no sequence and no constraint sequence
%		\If(\tcc*[f]{find next event to explore}){$S = \emptysequence$}{
%			\If(\tcc*[f]{multiple leads possible}){
%				$\exists (e_r, e_w) \in$ \FunexploredRW{$\tau, F$}}{
%				\lForAll{$e_w' \in \ui{\tau}{F}{e_r}$}{
%					\Fupdate($\tau, e_w', F$)
%				}
%				\nexte := $e_r$
%			}
%			\lElse{
%				\nexte := any event $\in$ \FunexploredEv{$\tau$};
%			}
%			
%			% updateLeads wrt to selected event
%			\Fupdate($\tau$, \nexte, $F$)
%		}
%		
%		% there is a sequence to be explored
%		\lElse{
%			\nexte := $\hd{S}$
%		}
%		
%		%		\lIf(\tcc*[f]{if no branch explore \nexte})
%		\lIf
%		{$S = \emptysequence \^$ \FunexploredLd{$\tau$} = $\emptyset$}{
%			$\ld{\s{\tau}} \cunioneq (\emptysequence,\ \seq{${\nexte}$},\ F)$
%		}
%		
%		\lElseIf(\tcc*[f]{explore next event in $S$}){$S \neq \emptysequence$}{
%			$\ld{\s{\tau}} \cunioneq (\emptysequence,\ S,\ F)$
%		}
%		
%		\While(\tcc*[f]{explore all leads})
%		{$\exists l \in$ \FunexploredLd{$\tau$}}{
%			\nextseq := $l^s \cmerge l^c$\;
%			\Dprime := $\{\tau' \| \hd{${\nextseq}$}.\tau' \in Dn(\s{\tau})\}$\;
%			\Fexplore{$\tau.\hd{${\nextseq}$},\ \tl{${\nextseq}$}$, \Dprime, $l^F$}\;
%			$\dn{\s{\tau}} \unioneq$ \nextseq
%		}
%	}
\end{algorithm}
\divComment{Can we give termination guarantee?}

\noindent
{\bf Algorithm~\ref{alg:main algo} overview:} 
The algorithm assumes the knowledge of the set of counter
examples in the form of traces (\ie a set of events and 
the sets of relations on the events, as defined in Section
\ref{sec:preliminaries}).
%
Broadly the algorithm places candidate fences before and
after every program event then works towards eliminating 
candidate
fences that do not contribute to the optimal solution.
%
The elimination is a two-phase process where in the first
phase the algorithm discards candidate fences that do not 
contribute to the violation of either a coherence condition 
or the \sc total order. 
It may be the case that we detect more than one violations
 in a buggy trace.
Thus, further, in the second phase the algorithm reduces the 
remaining candidate fences to the optimal number with
the optimal memory order.

The algorithm takes a \cc program as input and relies on a 
counter example generator to return the set of counter 
examples or buggy traces of the input program (line 3).
It then transforms the buggy traces $\tau$ to an intermediate 
trace $\imm{\tau}$ by synthesizing candidate 
fences (lines 5,6).
%
The algorithm iterates over each counter example to 
collect cycles in coherence conditions or \sc total order
(lines 7,8) and aborts the process if for any buggy trace
the set of cycles is empty, indicating that the trace cannot 
be stopped by synthesizing \cc fences (lines 9,10).
This step constitutes the phase one where any fence not
involved in a cycle is discarded.
%
On the fences involved in the discovered cycles, we use a
SAT solver to compute the minimum number of fences
(lines 11,13). 
%
The fences that contribute to the optimal (in number of
fences) set of of fences are then mapped back to their 
corresponding cycle to ascertain the memory order of 
the fence (lines 12,14).
%
This step along with the previous step using a SAT solver
performs the phase two of elimination of candidate fences.
%
The final form of the buggy trace (with the optimal 
synthesized fences) renders the trace invalidated,
represented as $\inv{\tau}$, ensuring that the trace 
does not belong to the set of traces of the transformed
(fixed) program $\fx{P}$. 
%
We discuss the details of each step below.

\noindent
{\bf Counter examples and candidate fences:}
\ourtechnique is a fence synthesis technique to stop
buggy traces that requires the set of buggy traces to 
perform its analysis. We thus rely on an external counter
example generator that takes the input program $P$ and
returns the set of buggy traces (line 3) where each buggy 
trace is a tuple $\langle \events_\tau, \setHB, \setMO, 
\setRF \rangle$.
%
Consider the \hlref{mutex-input-program} where two 
threads are racing to mutually exclusively update the
value of $x$. The program under \cc violates the 
mutual exclusion property and a counter example generator
returns two buggy traces diagrammatically represented in
\hlref{mutex-bt1} and \hlref{mutex-bt2}.

\begin{figure}[!htb]
	\begin{center}
		\setlength{\tabcolsep}{5pt}
		\begin{tabular}{|l||l|}
			\hline
			\multicolumn{2}{|c|}{Initially: $Flag_1=0, Flag_2 = 0, x=0$} \\
			
			$ Flag_1 :=_\rlx 1 $ & $ Flag_2 :=_\rlx 1  $ \\
			\textbf{if} $ (Flag_2 =_\rlx 0) $ & \textbf{if} $ (Flag_1 =_\rlx 0) $ \\
			\quad $ x :=_\rlx 1 $ & \quad $ x :=_\rlx 2 $ \\
			\quad assert($ x =_\rlx 1 $) & \quad assert($ x =_\rlx 2 $) \\
			\hline
			
			\multicolumn{2}{c}{\hl{mutex-input-program}}
		\end{tabular} 
	\end{center}
\end{figure}

\begin{figure}[!h]
	\begin{tabular}{|c|c|c|c|}
		\hline
		\resizebox{0.24\textwidth}{!}{\input{figures/egMutex_ce1.tex}} &
		\resizebox{0.24\textwidth}{!}{\input{figures/egMutex_ce1_fences.tex}} &
		\resizebox{0.24\textwidth}{!}{\input{figures/egMutex_ce2.tex}} &
		\resizebox{0.24\textwidth}{!}{\input{figures/egMutex_ce2_fences.tex}} \\
		\hline
		
		\multicolumn{1}{c}{\hl{mutex-bt1}} &
		\multicolumn{1}{c}{$\imm{\hl{mutex-bt1}}$}  &
		\multicolumn{1}{c}{\hl{mutex-bt2}} &
		\multicolumn{1}{c}{$\imm{\hl{mutex-bt2}}$} \\
		
		\multicolumn{1}{c}{(buggy trace 1)} &
		\multicolumn{1}{c}{(intermediate-} &
		\multicolumn{1}{c}{(buggy trace 2)} &
		\multicolumn{1}{c}{(intermediate-} \\
	
		\multicolumn{1}{c}{} &
		\multicolumn{1}{c}{buggy trace 1)} &
		\multicolumn{1}{c}{} &
		\multicolumn{1}{c}{buggy trace 2)} \\
	\end{tabular}
\end{figure}

Algorithm~\ref{alg:main algo} iterates over each buggy trace
$\tau$ (line 4) and transforms the trace to an intermediate 
trace $\imm{\tau}$ (line 5). The algorithm further updates the 
event relations accordingly (line 6), as discussed in Section
\ref{sec:invalidating ce} a change is witnessed in the 
$\setHB$ relation. The transformed intermediate traces 
corresponding to buggy traces \hlref{mutex-bt1} and 
\hlref{mutex-bt2}, along with the updated event relations are 
represented in $\imm{\hlref{mutex-bt1}}$ and 
$\imm{\hlref{mutex-bt2}}$ respectively.

\noindent
{\bf Detecting cyclic relations indicating violation of 
	trace coherence:}
In each intermediate buggy trace, the algorithm proceeds 
to perform \wkfence (line 7) and return cycles in compositions 
of event relations that define \hlref{coherence conditions} 
(Section~\ref{sec:c11}). 
%
Consider an intermediate buggy trace $\imm{\tau}$ that 
contains the following event relation: 
$\rf{\imm{\tau}}{e_1}{e_2}$, $\hb{\imm{\tau}}{e_2}{\mathbb{F}_1}$,
$\hb{\imm{\tau}}{\mathbb{F}_1}{e_1}$, where $e_i \in 
\events_\tau$ and $\mathbb{F}_1$ is a candidate fence.
The intermediate trace contains a weak cycle violating
the $\rf{\imm{\tau}}{}{}$;$\hb{\imm{\tau}}{}{}$ 
irreflexivity. The cycle is represented by the set of events
$\{e_1, e_2, \mathbb{F}_1\}$ and would belong the set
{\tt weakCycles} for $\imm{\tau}$.
%
For the intermediate traces 
$\imm{\hlref{mutex-bt1}}$ and $\imm{\hlref{mutex-bt2}}$
\wkfence would return an empty set and thus 
{\tt weakCycles} = $\emptyset$.

The algorithm then proceeds to perform \stfence (line 8) that 
computes the $\so{\imm{\tau}}{}{}$ relation on \sc 
events and returns the cycles in $\so{\imm{\tau}}{}{}$.
The cyclic $\so{\imm{\tau}}{}{}$ relations in the intermediate
traces $\imm{\hlref{mutex-bt1}}$ and $\imm{\hlref{mutex-bt2}}$ 
are shown in $\imm{\hlref{mutex-bt1-so}}$ and 
$\imm{\hlref{mutex-bt2-so}}$. 
%
Note that, $\so{\imm{\tau}}{}{}$ 
would also be formed between every fence before 
$\mathbb{F}^\sc_{13}$ and every fence including and after 
$\mathbb{F}^\sc_{24}$ in $\imm{\hlref{mutex-bt1-so}}$ 
(correspondingly in $\imm{\hlref{mutex-bt1-so}}$), however, we 
have skipped the edges in the figures for better readability.
%
As we can see from the figures, the $\so{\imm{\tau}}{}{}$ 
relation contains a cycle $\{\mathbb{F}^\sc_{12}, 
\mathbb{F}^\sc_{22}\}$. The algorithm discards all candidate 
fences, sparing $\mathbb{F}^\sc_{12}$ and $\mathbb{F}^\sc_{22}$, 
from the solution and determines that the buggy traces 
can be stopped with fences as the set of cycles for each of the 
two buggy traces in not empty (lines 9,10).
%
The algorithm collects the cycles as solutions (line 12) and 
proceeds to phase two for reducing the solution to the optimal 
solution.

\begin{figure}[!h]
	\begin{tabular}{|c|c|}
		\hline
		\resizebox{0.24\textwidth}{!}{\input{figures/egMutex_ce1_fences_so.tex}} &
		\resizebox{0.24\textwidth}{!}{\input{figures/egMutex_ce2_fences_so.tex}} \\
		\hline
		
		\multicolumn{1}{c}{$\imm{\hl{mutex-bt1-so}}$}  &
		\multicolumn{1}{c}{$\imm{\hl{mutex-bt2-so}}$-} \\
		
		\multicolumn{1}{c}{intermediate} &
		\multicolumn{1}{c}{intermediate} \\
		
		\multicolumn{1}{c}{buggy trace 1} &
		\multicolumn{1}{c}{buggy trace 2} \\
	\end{tabular}
\end{figure}

\noindent
{\bf Reducing to optimal number of synthesized fences 
	using SAT solvers:}
While iterating over the set of buggy traces, Algorithm
\ref{alg:main algo} collects a set of cycles in coherent 
conditions or \sc total order from each buggy trace. 
To stop the buggy traces at least one cycle from each 
buggy trace must be present in the optimal solution.
%
The algorithm uses a SAT query to determine the minimal
number of candidate fences to form an optimal
solution. 
%
To form the SAT query, from each cycle of a buggy trace 
we take a conjunction of the candidate fences that belong 
to the cycle. Further, to retain at least one cycle 
corresponding to each buggy trace we take a disjunction
of the product formula of each cycle.
The disjunction represents the SAT query corresponding to 
each buggy trace computed as $\formula{{\tt weakCycles} \v 
{\tt strongCycles}}$ (line 11).
%
Finally, we take the SAT queries of each buggy trace and 
conjunct them to form the SAT query representing a 
non-optimal solution to stop all buggy traces of $P$ 
(line 11).
The SAT query is represented in Equation~\ref{eq:sat query},
where ${\tt \bf W_\tau}$ and ${\tt \bf S_\tau}$ represent
the {\tt weakCycles} and {\tt strongCycles} in the iteration
corresponding to $\tau \in$ {\tt CE}.
\begin{equation}\label{eq:sat query}
	\Phi = \bigwedge\limits_{\tau \in {\tt CE}}
		(
		(\bigvee\limits_{{\tt W} \in {\tt \bf W_\tau}}
			\bigwedge\limits_{\mathbb{F}_{\tt w} \in {\tt W}} 
			\mathbb{F}_{\tt w})
		\v
		(\bigvee\limits_{{\tt S} \in {\tt \bf S_\tau}}
			\bigwedge\limits_{\mathbb{F}_{\tt s} \in {\tt S}} 
			\mathbb{F}_{\tt s})
		)
\end{equation}
We use a SAT solver to compute the {\em min-model} of the 
SAT query. The min-model returns the smallest set of candidate 
fences sufficient to satisfy the query \ie the smallest set
of candidate fences sufficient to stop all buggy traces of 
$P$ (line 13).
%
Consider the buggy traces \hlref{mutex-bt1} and \hlref{mutex-bt2},
for both traces the {\tt weakCycles} = $\emptyset$ and
{\tt strongCycles} = $\{\mathbb{F}^\sc_{12}, \mathbb{F}^\sc_{22}\}$.
The sat query for \hlref{mutex-input-program}, $\Phi$ = 
($\mathbb{F}^\sc_{12}$ $\^$ $\mathbb{F}^\sc_{22}$) $\^$ 
($\mathbb{F}^\sc_{12}$ $\^$ $\mathbb{F}^\sc_{22}$) and 
{\tt min}$\Phi$ = $\{\mathbb{F}^\sc_{12}, \mathbb{F}^\sc_{22}\}$.
\noindent
{\bf Determining optimal memory orders of fences:}
The smallest set of fences returned by the SAT solver gives us 
a sound solution to stop the buggy traces that is optimal in the
number of fences.
To achieve optimality in the memory order of fences, the 
minimal set of fences must be assigned the weakest memory
order that can stop the buggy traces.
Consider {\tt min-cycles} as the set of weak and strong cycles 
of all buggy traces \st every  fence in the cycle belongs 
to {\tt min$\Phi$}. 
%
\ourtechnique starts by assigning a memory order to fences of 
{\tt min-cycles} locally and computes a weight of the cycle as 
follows: 

We know that since introducing candidate fences only modifies
the relations $\setSB$, $\setSW$ and $\setDOB$ of a buggy trace
$\tau$ to form $\seqb{\imm{\tau}}{}{}$, $\sw{\imm{\tau}}{}{}$ 
and $\dob{\imm{\tau}}{}{}$ respectively, thus if a weak cycle
involving candidate fences is detected then the fence must form
a $\sw{\imm{\tau}}{}{}$ or $\dob{\imm{\tau}}{}{}$ relation with
an event of $\imm{\tau}$. 
%
Let $R$ = $\{ \sw{\imm{\tau}}{}{}$, $\dob{\imm{\tau}}{}{} \}$. 
%
If a fence $\mathbb{F}$ in a weak cycle $c \in$ {\tt min-cycles} 
is related to an event $e$ of the cycle $c$ as $eR\mathbb{F}$ 
then we assign the memory order \acq to $\mathbb{F}$.
%
Similarly, if an event $e$ of $c$ is related to $\mathbb{F}$ as 
$\mathbb{F}Re$ then we assign memory order \rel to $\mathbb{F}$ 
and if events $e,e'$ of $c$ are related as $eR\mathbb{F}Re'$ then 
we assign the memory order \acqrel to $\mathbb{F}$.
%
Further, all fences in a strong cycle are assigned the memory
order \sc.

Accordingly, we assign weights to the fences where stronger 
fences weigh more than weaker fences. We use the following
numeric weighing scheme: \rel or \acq ordered fence is assigned
weight 1, \acqrel fence is assigned weight 2 and \sc fence is
assigned weight 3. The weight of a cycle is the sum of weights
of the fences in the cycle.
%
Consider a weak cycle
$\seqb{\imm{\tau}}{e}{\sw{\imm{\tau}}{\mathbb{F}_1}{\sw{\imm{\tau}}{\mathbb{F}_2}{\seqb{\imm{\tau}}{\mathbb{F}_3}{\rf{\imm{\tau}}{e'}{e}}}}}$
representing a $\rf{\imm{\tau}}{}{};\hb{\imm{\tau}}{}{}$ 
violation. According to the scheme discussed above the fences
$\mathbb{F}_1$, $\mathbb{F}_2$ and $\mathbb{F}_3$ are assigned 
the memory orders \rel, \acqrel and \acq and weights 1, 2 and 
1 respectively. The weight of the cycle is computed as 4.

The locally assigned memory orders represent the weakest 
memory order that can stop the corresponding trace. 
\ourtechnique further iterates over all buggy traces and 
detects the weakest memory order for each fence that can stop 
all the buggy traces as follows: 
%
Consider a cycle $c_1$ that stops a buggy trace $\tau_1$ and a 
cycle $c_2$ that stops a buggy trace $\tau_2$. We coalesce the
two to form an optimal solution that stops both $\tau_1$ and
$\tau_2$ by taking a union of the fences and choosing the
stronger memory order for each fence present in both cycles.
%
Further, between two candidate solutions with the same set of
fences we select the one with the lower weight.

\begin{wrapfigure}{l}{0.45\textwidth}
	\vspace{-2.5em}
	\centering
	\setlength{\tabcolsep}{5pt}
	\begin{tabular}{|l|}
		\hline
		$\tau_1c_1$(4): $\mathbb{F}^\acqrel_1$ $\^$ $\mathbb{F}^\acqrel_2$ \\
		$\tau_1c_2$(4): $\mathbb{F}^\rel_1$ $\^$ $\mathbb{F}^\acq_2$ $\^$ $\mathbb{F}^\acqrel_3$ \\
		\multicolumn{1}{|c|}{cycles of $\tau_1$} \\
		\hline
		
		$\tau_2c_1$(3): $\mathbb{F}^\rel_1$ $\^$ $\mathbb{F}^\acq_2$ $\^$ $\mathbb{F}^\acq_3$ \\
		\multicolumn{1}{|c|}{cycle of $\tau_2$} \\
		\hline
		
		$\tau_{12}c_{11}$(5): $\mathbb{F}^\acqrel_1$ $\^$ $\mathbb{F}^\acqrel_2$ $\^$ $\mathbb{F}^\acq_3$ \\
		$\tau_{12}c_{21}$(4): $\mathbb{F}^\rel_1$ $\^$ $\mathbb{F}^\acq_2$ $\^$ $\mathbb{F}^\acqrel_3$ \\
		\multicolumn{1}{|c|}{candidate solutions} \\
		\hline
	\end{tabular}
	\vspace{-2.5em}
\end{wrapfigure}

Consider the cycles of buggy traces $\tau_1$ and $\tau_2$
shown on the left.
Let {\tt min$\Phi$} = $\{\mathbb{F}_1, \mathbb{F}_2, 
\mathbb{F}_3\}$.
The locally computed memory orders of the fences are shown 
as superscripts and the weights of the cycle $\tau_1c_1$,
$\tau_1c_2$ and $\tau_2c_1$ are 4, 4, 3 respectively 
written against the name of the cycles in the figure.
%
Assuming \ourtechnique starts with $\tau_1$, it detects 
two candidate solutions ($\tau_1c_1$ and $\tau_1c_2$) that 
can stop $\tau_1$. 
%
The candidate solutions are coalesced with $\tau_2c_1$
to form $\tau_{12}c_{11}$ and $\tau_{12}c_{21}$ of
weights 5 and 4 respectively. \ourtechnique discards
$\tau_{12}c_{11}$ and assigns the optimal memory orders
\rel, \acq and \acqrel to fences $\mathbb{F}_1$,
$\mathbb{F}_2$ and $\mathbb{F}_3$ respectively.

The candidate fences inserted by \ourtechnique that 
are contained in {\tt min$\Phi$} are synthesized in
the input program at their respective program 
locations. 
%
The set {\tt min$\Phi$} may contain fences that belong
to $\events$ \ie fences in the events of the input 
program, not inserted by \ourtechnique. The memory
order of the program fence may differ from the memory
order computed by \ourtechnique.
%
A stronger original memory order than the one computed
is sound, however, if the original memory order is
weaker then we synthesize a fresh fence at the 
respective program location with a complementary
memory order. 
%
For example, assume the input program contains a fence 
($\mathbb{F}$) with memory order \rel and 
\ourtechnique computes that $\mathbb{F}$
$\in$ {\tt min$\Phi$} and must have the memory order
\acqrel, in such a case we synthesize a fence 
$\mathbb{F}'$ with memory order \acq immediately 
program-ordered before $\mathbb{F}$.
%
The set of events $\events$ $\union$ the set of 
synthesized fences form the events of $\fx{P}$.