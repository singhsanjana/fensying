Future work:
The theory and this tool can be further worked upon
by introducing many new features and optimizations. A feature in the
theory which can provide substantial optimization is to consider
\moar fences along with \mosc fences. \mosc fences are the strongest 
kinds of fences, hence introducing a much stricter model as compared with
other fences. On the other hand, \moar fences
being of a weaker nature can be mixed in along with the \mosc
fences. A way to implement this is by exploring \setSW 
paths in the input program after introducing
pseudo fences and then observing their semantics. Introducing \moar fences
along with \mosc fences (if required at all) would result in a
weaker output model than the one described in this tool, yet retaining
the basic idea of preventing buggy executions.

Another method to prevent buggy outputs without having to necessarily
introduce \mosc fences at all possible places is by changing the 
memory orders of the instructions themselves. For instance, a \morlx 
instructions can be switched to a \morel/\moacq, \moar, or a \mosc memory order.
\ishComment{Would this necessarily result in a weaker model? Or is it simply another method to obtain the same results?}

The tool can be further worked upon as well. Currently it does not have support
for locks. Functionality for detecting locks and using their semantics
in the theory for \setTO relations can be additionally explored. Another optimization
to consider can be the detection of loops in the program. A single fence
inserted inside the scope of a loop in actuality is not a single fence, since
the contents of the loop will run multiple times. In this case, inserting extra fences
outside the loop may end up being a more optimized solution.