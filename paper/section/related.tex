The literature in the  fence synthesis is dominated by 
techniques targeting the x86-TSO and sparc-PSO memory models
such as \cite{abdulla2012counter,alglave2010fences,alglave2014don,linden2011verification,abdulla2012automatic,abdulla2015best,bender2015declarative} 
for x86-TSO,
\cite{abdulla2015precise,linden2013verification} 
for sparc-PSO and \cite{liu2012dynamic,meshman2014synthesis,abdulla2013memorax,joshi2015property,kuperstein2012automatic}
for both TSO and PSO.
%
The technique \cite{bender2015declarative} additionally
provides fence synthesis for ARMv7 memory model and 
\cite{kuperstein2012automatic} additionally provides
fence synthesis for RMO memory model.
%
Apart from the techniques for TSO and PSO, a few 
fence synthesis techniques have been proposed for Power
memory model 
\cite{alglave2010fences,abdulla2015precise,fang2003automatic}, 
where \cite{fang2003automatic} also 
provides support for IA-32 memory model, and
\cite{joshi2015property} provides a technique applicable to
any memory model whose behaviors can be justified using
interleaving with reordering.
%
\ourtechnique is the first fence synthesis technique for 
\cc memory model.

Some of the earlier works  
\cite{alglave2010fences,abdulla2015precise,fang2003automatic} 
perform fence synthesis to reduce weak memory behaviors of an 
input program to those permitted under Sequential Consistency (SC)
or its variant \cite{abdulla2015best}. Such reduction would
make the input compatible with the existing verification 
techniques  for SC memory model.
%
Most fence synthesis techniques (including \ourtechnique)
\cite{meshman2014synthesis}\cite{abdulla2012counter}\cite{abdulla2013memorax}\cite{joshi2015property}\cite{abdulla2015precise}\cite{linden2011verification}\cite{kuperstein2012automatic}\cite{abdulla2012automatic}\cite{linden2013verification}, however, 
use safety property specifications (usually provided as assert
statements in the input program) and attempt to remove 
program traces that violate a safety property.

Some of the techniques discussed above
\cite{meshman2014synthesis}\cite{taheri2019polynomial}\cite{abdulla2012counter}\cite{abdulla2013memorax}\cite{joshi2015property}\cite{abdulla2015precise}\cite{kuperstein2012automatic}\cite{bender2015declarative} perform optimal fence synthesis, 
same as \ourtechnique, for their respective memory models.
%
The techniques \cite{linden2011verification}\cite{linden2013verification} 
extend their solution to cyclic programs.
%
The technique \cite{joshi2015property} proposes a bounded 
fence synthesis technique, where they bound the number of 
reordering to a constant $k$.