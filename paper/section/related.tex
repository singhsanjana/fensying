The literature in the  fence synthesis is dominated by 
techniques targeting the x86-TSO and sparc PSO memory models
such as \cite{abdulla2012counter}\cite{alglave2010fences}
\cite{alglave2014don}\cite{linden2011verification}
\cite{abdulla2012automatic}\cite{abdulla2015best}
\cite{bender2015declarative} for x86-TSO,
\cite{abdulla2015precise}\cite{linden2013verification} 
for sparc PSO and \cite{liu2012dynamic}
\cite{meshman2014synthesis}\cite{abdulla2013memorax}
\cite{joshi2015property}\cite{kuperstein2012automatic}
for both TSO and PSO.
%
The technique \cite{bender2015declarative} additionally
provides fence synthesis for ARMv7 memory model and 
\cite{kuperstein2012automatic} additionally provides
fence synthesis for RMO memory model.
%
Apart from the techniques for TSO and PSO, a few 
fence synthesis techniques have been proposed for Power
memory model \cite{alglave2010fences}\cite{abdulla2015precise}
\cite{fang2003automatic}; \cite{fang2003automatic} also 
provides support for IA-32 memory model.

Some of the earlier works in the area \cite{alglave2010fences}
\cite{abdulla2015precise}\cite{fang2003automatic} 
perform fence synthesis to reduce a weak memory behaviors of an 
input program to those permitted under Sequential Consistency (SC)
or its variant \cite{abdulla2015best}.
%
Most techniques \cite{