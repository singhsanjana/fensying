As discussed in Section~\ref{sec:preliminaries},
the \cc memory model defines a program trace by forming a set
of relations on the program events that follow \cc {\em coherence
rules}. 
%
\cc maintains {\em release sequences} in a trace that assist in
forming the event relations. A release sequence is the longest 
contiguous sequence of write (or rmw) events on an object $o$,
headed by a \rel or stricter write (or rmw) event, called a
{\em release head}. 
%
The sequence includes all events modification-ordered after the
release head and is broken by a weak write event from another thread.

The \cc memory model introduces an irreflexive and acyclic relation 
over the events of a trace $\tau$ called the {\em happens-before} 
relation ($\setHB$),
 \st $\setHB \subseteq \events_\tau {\times} \events_\tau$.
%
The happens-before relation is constructed by taking a union
of {\em sequenced-before} ($\setSB$) and {\em inter-thread-hb}
($\setITHB$). The components of happens-before are discussed
below.

\noindent
{\bf Sequenced-before($\setSB$)}: Events of a thread $P_i$  are 
	related by the {\em sequenced-before} relation ($\setSB$) in 
	their order of occurrence in $P_i$.
	
\noindent
{\bf Synchronizes-with($\setSW$)}: When a strict write $e_w$ 
	(memory order \rel or stricter) and a strict read $e_r$ 
	(memory order \acq or stricter) from different threads are 
	related as $\rf{\tau}{e_w}{e_r}$, they are also 
	related by the synchronizes-with as
	$\sw{\tau}{e_w}{e_r}$.

\noindent
{\bf Dependency-ordered-before($\setDOB$)}: When a strict read 
	$e_r$ (memory order \acq or stricter) is related to a write
	$e_w$ as $\rf{\tau}{e_w}{e_r}$ where $e_w$ belongs to the
	release sequence of a strict write $e_w'$ (memory order \rel 
	or stricter), $e_w'$ and $e_r$ are also related by the
	dependency-ordered-before \ie $\dob{\tau}{e_w'}{e_r}$.
	
\noindent
{\bf Inter-thread-hb($\setITHB$)}: The $\setSW$ and $\setDOB$
	relations form an inter-thread-hb between their corresponding 
	threads \ie $\sw{\tau}{e_w}{e_r}$ or $\dob{\tau}{e_w}{e_r}$
	$\implies$ $\ithb{\tau}{e_w}{e_r}$. Further
	all events $e,e'$ \st $\seqb{\tau}{e}{e_w}$ and 
	$\seqb{\tau}{e_r}{e'}$ are also related as $\ithb{\tau}{e}{e'}$.

Thus, two events $e_1,e_2$ in a trace $\tau$ are happens-before 
related \ie
$\hb{\tau}{e_1}{e_2}$ if $\seqb{\tau}{e_1}{e_2}$ $\v$ 
$\ithb{\tau}{e_1}{e_2}$.
%
As discussed previously, in a trace $\tau$, all write events of 
an object $o$ are related by a total order called 
{\em modification-order} ($\setMO$).
%
The $\setMO$ order is constructed in compliance with $\setHB$ and 
$\setRF$ such that $\setMO$, $\setRF$ and $\setHB$ must not 
disagree.
%
To meet the requirement \cc introduces a set of
coherence rules listed below \cite{LahavVafeiadis-PLDI17}.
A valid \cc trace must satisfy the conjunction
of the coherence rules.
%
\newline $\setHB$ is irreflexive
\newline $\setRF;\setHB$ is irreflexive
\newline $\setMO;\setRF;\setHB$ is irreflexive
\newline $\setMO;\setHB$ is irreflexive
\newline $\setMO;\setHB;\setRF^{-1}$ is irreflexive
\newline $\setMO;\setRF;\setHB;\setRF^{-1}$ is irreflexive

Finally, all \sc ordered events in a trace $\tau$ must be related
by a total order ($\setTO$) that concurs with the coherence rules.
%
We introduce an irreflexive relation called {\em from-reads} $\setFR$ 
that relates read events with write events ordered after it.
%
The relation from-reads is typically used for stricter memory models 
that relate all events of a trace by a total order,
such as sequentially-consistent memory model or TSO memory model.
\cc model constitutes a similar requirement on the \sc ordered
events of a trace and we use $\setFR$ it to define the $\setTO$
relation.
%
\begin{definition}[{\em from-reads} $\setFR$]
	$\setFR$ = $\setRF^{-1}$;$\setMO$
\end{definition}
%
Thus, the $\setTO$ must be constructed such that,

if 
$\to{\tau}{e^\sc_1}{e^\sc_2}$ then 
$\nhb{\tau}{e^\sc_2}{e^\sc_1}$ $\^$
$\nmo{\tau}{e^\sc_2}{e^\sc_1}$ $\^$
$\nrf{\tau}{e^\sc_2}{e^\sc_1}$ $\^$
$\nfr{\tau}{e^\sc_2}{e^\sc_1}$.

\noindent
In our technique we attempt to break the irreflexivity of either
a coherence rules or $\setTO$ by strategically placing \cc fences in 
the input program, as discussed in Section~\ref{sec:so theory}.

\noindent
{\bf Brief introduction to \cc fences}: 
\cc fences provide additional reordering restrictions on program 
events. Note that \cc fences are not memory barriers and do not
provide support for flushing local write values to shared memory.
%
A fence can be associated with memory orders $\rel$, $\acq$, 
$\acqrel$ and $\sc$
providing varying degrees of reordering restrictions.
%
Similar to program events, $\setSW$ and $\setDOB$ relations can also 
be formed between \cc fences and program events
\cite{batty2011mathematizing}\cite{C11} where an appropriately
placed fence assists a relaxed event in forming the necessary 
synchronization, as shown in the figure below.
%
\begin{figure}[h]
	\begin{tabular}{|c|c|c|c||c|c|}
		\hline
		\resizebox{0.13\textwidth}{!}{\tikzset{every picture/.style={line width=0.75pt}} %set default line width to 0.75pt        
\begin{tikzpicture}[x=1em,y=1em,yscale=-1,xscale=-1]
\tikzstyle{every node}=[font=\normalfont]
\node (w1) {$ w^\rel_1 $};
\node (r1) [below=20pt of w1] {$ r^\acq_1 $};
\node (sw) [below=-5pt of r1] {\hlref{sw}};

\draw [->,>=stealth,color=Magenta] ($ (w1.south east)+(.3,-5pt) $) to[out=135,in=-135] node[midway,right=-2pt,font=\scriptsize] {\textcolor{black}{\lsw}} ($ (r1.south east)+(0.4,-7pt) $);
\draw [->,>=stealth,color=PineGreen] ($ (w1.south west)+(-.3,-5pt) $) to[out=45,in=-45] node[left=-2pt,pos=.25,font=\scriptsize] {\textcolor{black}{\lrf}}($ (r1.south west)+(-0.3,-7pt) $);

\end{tikzpicture}
} &
		\resizebox{0.14\textwidth}{!}{\tikzset{every picture/.style={line width=0.75pt}} %set default line width to 0.75pt        
\begin{tikzpicture}[x=1em,y=1em,yscale=-1,xscale=-1]
\tikzstyle{every node}=[font=\normalfont]
\node (w1) [inner sep=2pt] {$ w^\rel_1 $};
\node (r1) [right=25pt of w1,inner sep=2pt] {$ r_1 $};
\node (f1) [below left=21pt and -15pt of r1,inner sep=2pt] {$ f^\acq_1 $};
\node (swef) [below left=0pt and -10pt of f1] {\hlref{swEF}};

`\draw [->,>=stealth,color=Magenta,thin] (w1.south) -- node[midway,left=0pt,font=\scriptsize,color=black] { $\lsw$ } (f1.west);
\draw [->,>=stealth,color=PineGreen,thin] (w1) -- node[midway,above=-2pt,font=\scriptsize,color=black] { $ \lrf $ }  (r1);
\draw [->,>=stealth,color=CarnationPink,thin] (r1) -- node[midway,left=-2pt,font=\scriptsize,color=black] { $\lsb$ } (f1);

\end{tikzpicture}
} &
		\resizebox{0.14\textwidth}{!}{\tikzset{every picture/.style={line width=0.75pt}} %set default line width to 0.75pt        
\begin{tikzpicture}[x=1em,y=1em,yscale=-1,xscale=-1]
\tikzstyle{every node}=[font=\normalfont]
\node [inner sep=2pt] (f1) {$ f^\rel_1 $};
\node (w1) [below left=21pt and -15pt of f1,inner sep=2pt] {$ w_1 $};
\node (r1) [right=25pt of w1,inner sep=2pt] {$ r^\acq_1 $};
\node (swfe) [below right=0pt and -15pt of wr1] {\hlref{sw-dobFE}};

`\draw [->,>=stealth,color=Magenta,thin] (f1.east) -- node[midway,right=0pt,font=\scriptsize,color=black] { $\lsw$ } (r1.north);
\draw [->,>=stealth,color=PineGreen,thin] (w1) -- node[midway,above=-2pt,font=\scriptsize,color=black] { $ \lrf $ } (r1);
\draw [->,>=stealth,color=CarnationPink,thin] (f1) -- node[midway,left=-2pt,font=\scriptsize,color=black] { $\lsb$ } (w1);

\end{tikzpicture}
} &
		\resizebox{0.17\textwidth}{!}{\tikzset{every picture/.style={line width=0.75pt}} %set default line width to 0.75pt        
\begin{tikzpicture}[x=1em,y=1em,yscale=-1,xscale=-1]
\tikzstyle{every node}=[font=\normalfont]
\node (f1) [inner sep=2pt] {$ f^\rel_1 $};
\node (f2) [right=25pt of f1,inner sep=2pt] {$ f^\acq_2 $};
\node (w1) [below left=20pt and -15pt of f1,inner sep=2pt] {$ w_1 $};
\node (r1) [below left=20pt and -15pt of f2,inner sep=2pt] {$ r_1 $};
\node (swff) [below right=-2pt and -5pt of w1] {\hlref{swFF}};

`\draw [->,>=stealth,color=Magenta] (f1) -- node[midway,above=-2pt,font=\scriptsize,color=black] { $\lsw$ } (f2);
\draw [->,>=stealth,color=PineGreen] (w1) -- node[midway,above=-2pt,font=\scriptsize,color=black]{\lrf} (r1);
\draw [->,>=stealth,color=CarnationPink] (f1) -- node[midway,left=-2pt,font=\scriptsize,color=black] { $\lsb$ } (w1);
\draw [->,>=stealth,color=CarnationPink] (r1) -- node[midway,left=-2pt,font=\scriptsize,color=black] { $\lsb$ } (f2);

%\draw [->,>=stealth,color=orange] ($ (ew1.south east)+(.5,-5pt) $) to[out=135,in=-135] node[midway,right=-2pt,font=\scriptsize] {mo} ($ (ew2.south east)+(0.4,-5pt) $);
%\draw [->,>=stealth,color=red] ($ (ew1.south west)+(-.3,-5pt) $) to[out=45,in=-45] node[midway,left=-2pt,font=\scriptsize] {c::hb} ($ (ew2.south west)+(-0.3,-5pt) $);


\end{tikzpicture}
} &
		
		\resizebox{0.16\textwidth}{!}{\tikzset{every picture/.style={line width=0.75pt}} %set default line width to 0.75pt        
\begin{tikzpicture}[x=1em,y=1em,yscale=-1,xscale=-1]
\tikzstyle{every node}=[font=\normalfont]
\node (w1) [inner sep=2pt] {$ w^\rel_1 $};
\node (r1) [right=25pt of w1,inner sep=2pt] {$ r^\acq_1 $};
\node (w2) [below left=21pt and -15pt of w1,inner sep=2pt] {$ w_2 $};
\node (dob) [below right=0pt and -5pt of w2] {\hlref{dob}};

`\draw [->,>=stealth,color=Mulberry] (w1.east) -- node[midway,above=-2pt,font=\scriptsize,color=black] { $\ldob$ } (r1.west);
\draw [->,>=stealth,color=PineGreen] (w2) -- node[midway,below=-2pt,font=\scriptsize,color=black] { $ \lrf $ }  (r1);
\draw [->,>=stealth,color=CarnationPink] (w1) -- node[midway,left=-2pt,font=\scriptsize,color=black] { $\lsb$ } (w2);

\end{tikzpicture}
} &
		\resizebox{0.17\textwidth}{!}{\tikzset{every picture/.style={line width=0.75pt}} %set default line width to 0.75pt        
\begin{tikzpicture}[x=1em,y=1em,yscale=-1,xscale=-1]
\tikzstyle{every node}=[font=\normalfont]
\node (w1) [inner sep=2pt] {$ w^\rel_1 $};
\node (f1) [right=25pt of w1,inner sep=2pt] {$ f^\acq_1 $};
\node (w2) [below left=20pt and -15pt of w1,inner sep=2pt] {$ w_2 $};
\node (r1) [below left=20pt and -13pt of f1,inner sep=2pt] {$ r_1 $};
\node (dobEF) [below right=0pt and -5pt of w2] {\hlref{dobEF}};

`\draw [->,>=stealth,color=PineGreen,thin] (w2) -- node[midway,above=-2pt,font=\scriptsize,color=black] { $\lrf$ } (r1);
\draw [->,>=stealth,color=CarnationPink,thin] (w1) -- node[midway,left=-2pt,font=\scriptsize,color=black] { $ \lsb $ } (w2);
\draw [->,>=stealth,color=CarnationPink,thin] (r1) -- node[midway,left=-2pt,font=\scriptsize,color=black] { $\lsb$ } (f1);
\draw [->,>=stealth,color=Mulberry,thin] (w1) -- node[midway,above=-2pt,font=\scriptsize,color=black] { $\ldob$ } (f1);

\end{tikzpicture}
} \\
		\hline
		\multicolumn{4}{c}{(a) \lsw relation} &
		\multicolumn{2}{c}{(b) \ldob relation} \\
	\end{tabular}
	\label{fig:sw}
\end{figure}

\noindent
Formally, $\setSW$ relation is formed using fences as,

$\forall e_1, e_2 \in \events_\tau$ \st $\rf{\tau}{e_1}{e_2}$

if
$e_1 \in \ordwrites{\moge\rel}_\tau$, 
$\exists \mathbb{F}^\acq \in \ordfences{\moge\acq}_{\tau}$ where
$\seqb{\tau}{e_2}{\mathbb{F}^\acq}$ then
$\sw{\tau}{e_1}{\mathbb{F}^\acq}$;

if
$e_2 \in \ordreads{\moge\acq}_\tau$, 
$\exists \mathbb{F}^\rel \in \ordfences{\moge\rel}_{\tau}$ where
$\seqb{\tau}{\mathbb{F}^\rel}{e_1}$ then
$\sw{\tau}{\mathbb{F}^\rel}{e_2}$;

if
$\exists \mathbb{F}^\rel \in \ordfences{\moge\rel}_{\tau}$,
$\mathbb{F}^\acq \in \ordfences{\moge\acq}_{\tau}$ where
$\seqb{\tau}{\mathbb{F}^\rel}{e_1}$,
$\seqb{\tau}{e_2}{\mathbb{F}^\acq}$ 

then
$\sw{\tau}{\mathbb{F}^\rel}{\mathbb{F}^\acq}$.

\noindent
Similarly, $\setDOB$ relation is formed using fences as,

$\forall e_1, e_2 \in \events_\tau$, 
$e_1' \in \ordwrites{\moge\rel}_\tau$ \st $\rf{\tau}{e_1}{e_2}$
and $e_1 \in$ {\em release-sequence}($e_1'$)

if $\exists \mathbb{F}^\acq \in \ordfences{\moge\acq}_{\tau}$ 
where $\seqb{\tau}{e_2}{\mathbb{F}^\acq}$ then
$\dob{\tau}{e_1'}{\mathbb{F}^\acq}$.

\noindent
Further, as is the case with program events,
if $\sw{\tau}{e_1}{e_2}$ or $\dob{\tau}{e_1}{e_2}$ then
events $e_1',e_2'$ \st $\seqb{\tau}{e_1'}{e_1}$ and 
$\seqb{\tau}{e_2}{e_2'}$ are related as $\ithb{\tau}{e}{e'}$.