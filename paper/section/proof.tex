Let $\cycle{\tau}$ represent the cycles of a trace $\tau$ and 
$\scycles{\tau}$, $\rcycles{\tau}$ represent the strong and 
relaxed cycles of $\tau$ respectively.

{\definition Strong Cycle}
A strong cycle $s \in \scycles{\tau}$ of a trace $\tau$ contains a 
set of events $\events_{s}$ = $\{e_1, e_2 ... e_n\}$,
where $\events_{s} \subseteq \events_\tau$,
\st $\so{s}{e_1}{\so{s}{...}{\so{s}{e_n}{e_1}}} $.
{\definition Relaxed cycle}
A relaxed cycle $r \in \rcycles{\tau}$ of a trace $\tau$ contains a 
set of events $\events_{s}$ = $\{e_1, e_2 ... e_n\}$,
where $\events_{r} \subseteq \events_\tau$,
\st $\reln{R}{s}{e_1}{\reln{R}{s}{...}{\reln{R}{s}{e_n}{e_1}}} $,
where $R = \setHB \union \setWS$.\newline

{\theorem{\lso-rules are exhaustive \ie $\setTO$ = $(\setSO)^+$ }}
\begin{proof}
	
	\noindent
	{\case{$\setTO \implies (\setSO)^+$}}
	
	$\setTO$ = $(\onsc{\setHB}$ $\union$ $\onsc{\setMO}$ $\union$
		$\onsc{\setRF}$ $\union$ $\onsc{\setFR})^+$.
	
	\noindent	Further, let $\exists e_1, e_2 \in \ordevents{\sc}$ \st
	
	if $\hb{\tau}{e_1}{e_2}$ then $\so{\tau}{e_1}{e_2}$ 
	(using \hlref{sohb}, \hlref{soFE}, \hlref{soEF}, \hlref{soFF});
	
	if $\mo{\tau}{e_1}{e_2}$ then $\so{\tau}{e_1}{e_2}$ (using \hlref{somo});
	
	if $\rf{\tau}{e_1}{e_2}$ then $\so{\tau}{e_1}{e_2}$ (using \hlref{sorf});
	
	if $\fr{\tau}{e_1}{e_2}$ then $\so{\tau}{e_1}{e_2}$ (using \hlref{sofr}).
	
	\noindent
	Thus, if $\exists \to{\tau}{e_1}{e_2}$ then $(e_1,e_2) \in (\setSO)^+$.
	
	{\case{$\setTO \impliedby (\setSO)^+$}}
	
	Let $\exists \so{\tau}{e_1}{e_2}$ $\implies$ $e_1,e_2 \in \ordevents{\sc}$
		(follows from construction of $\setSO$).
	
	Further,
	
	\begin{itemize}
	\item if $\so{\tau}{e_1}{e_2}$ generated by rule \hlref{sohb} then
		$(e_1,e_2) \in \onsc{\setHB}$ $\implies$ $\to{\tau}{e_1}{e_2}$;
		
	\item if $\so{\tau}{e_1}{e_2}$ generated by rule \hlref{somo} then
		$(e_1,e_2) \in \onsc{\setMO}$ $\implies$ $\to{\tau}{e_1}{e_2}$;
		
	\item if $\so{\tau}{e_1}{e_2}$ generated by rule \hlref{sorf} then
		$(e_1,e_2) \in \onsc{\setRF}$ $\implies$ $\to{\tau}{e_1}{e_2}$;
		
	\item if $\so{\tau}{e_1}{e_2}$ generated by rule \hlref{sofr} then
		$(e_1,e_2) \in \onsc{\setFR}$ $\implies$ $\to{\tau}{e_1}{e_2}$;
		
	\item if $\so{\tau}{e_1}{e_2}$ generated by rule \hlref{soEF} then
	 $(e_1,e_2) \in \onsc{\setHB}$ (if events corresponding to $wr^\sc_1$, 
	 $wr_2$ of Fig~\ref{fig:so rules}\hlref{soEF} $\in$ $\setHB$ or $\setRF$)
	 and $\to{\tau}{e_1}{e_2}$ (if events corresponding to $wr^\sc_1$, 
	 $wr_2$ of Fig~\ref{fig:so rules}\hlref{soEF} $\in$ $\setMO$)
	 $\implies$ $\to{\tau}{e_1}{e_2}$;	
	 
	 \item if $\so{\tau}{e_1}{e_2}$ generated by rule \hlref{soFE} then
	 $(e_1,e_2) \in \onsc{\setHB}$ (if events corresponding to $wr_1$, 
	 $wr^\sc_2$ of Fig~\ref{fig:so rules}\hlref{soFE} $\in$ $\setHB$ or $\setRF$)
	 and $\to{\tau}{e_1}{e_2}$ (if events corresponding to $wr_1$, 
	 $wr^\sc_2$ of Fig~\ref{fig:so rules}\hlref{soFE} $\in$ $\setMO$)
	 $\implies$ $\to{\tau}{e_1}{e_2}$; 
	 
	 \item if $\so{\tau}{e_1}{e_2}$ generated by rule \hlref{soFF} then
	 $(e_1,e_2) \in \onsc{\setHB}$ (if events corresponding to $wr_1$, 
	 $wr_2$ of Fig~\ref{fig:so rules}\hlref{soFF} $\in$ $\setHB$ or $\setRF$)
	 and $\to{\tau}{e_1}{e_2}$ (if events corresponding to $wr_1$, 
	 $wr_2$ of Fig~\ref{fig:so rules}\hlref{soFF} $\in$ $\setMO$)
	 $\implies$ $\to{\tau}{e_1}{e_2}$; 
	 
	 \item if $\so{\tau}{e_1}{e_2}$ generated by rule \hlref{soRFfr},
	 \hlref{soFWfr}, \hlref{soFFfr} then $\to{\tau}{e_1}{e_2}$.
	\end{itemize}

	\noindent
	Thus, $(\setSO)^+ \implies \setTO$. 
	\qed
\end{proof}

{\definition $\setWS$ relation}
The relation represents the absence of synchronization by $\setSW$ or $\setDOB$ 
\ie

If $\exists e_1 \in \ordwrites{\sc}_\tau \union \ordfences{\sc}_\tau$, 
$e_2 \in \ordreads{\sc}_\tau \union \ordfences{\sc}_\tau$ then
$\ws{\tau}{e_2}{e_1}$ if
$\exists (e_2',e_1') \in \setRF^{-1}$ \st
$\seqb{\tau}{e_1'}{e_1}$ and $\seqb{\tau}{e_2}{e_2'}$.

{\corollary
$\exists (e_2',e_1') \in \setRF^{-1}$ $\implies$
$(e_1',e_2') \nin \setITHB$ $\implies$
$\neg(\sw{\tau}{e_1}{e_2} \v \dob{\tau}{e_1}{e_2})$ $\implies$
$\ws{\tau}{e_2}{e_1}$.
}


{\theorem {\lws-rules are exhaustive}}
\begin{proof}
	The $\setWS$ relation signifies the absence of synchronization,
	\ie, 
	
	$\forall$
	$\fr{\tau}{r_1}{w_1}$ $\implies$
	if $\exists$ $e^\rel_1 \in \ordevents{\moge\rel}_\tau$ and
	$e^\acq_2 \in \ordevents{\moge\acq}_\tau$ \st 
	$\seqb{\tau}{w_1}{e^\rel_1}$ and $\seqb{\tau}{e^\acq_2}{r_1}$
	then $\nsw{\tau}{e^\rel_1}{e^\acq_2}$ $\^$ 
	$\ndob{\tau}{e^\rel_1}{e^\acq_2}$
	(where $\moge\rel$ = $\{\rel,\acqrel,\sc\}$ and 
	$\moge\acq$ = $\{\acq,\acqrel,\sc\}$).
	
	I] The $\setSW$ relation can be formed (or be absent) 
	between the following combinations of type of $e^\rel_1$ 
	and $e^\acq_2$
	(covered in rules \hlref{sw}, \hlref{swEF}, \hlref{swFE} and
	\hlref{swFF}):
	
	\begin{itemize}[label=Case EE,align=left,leftmargin=*]
		\item [Case EE]: $e^\rel_1 \in \ordwrites{\moge\rel}_\tau$
			and $e^\acq_2 \in \ordreads{\moge\acq}_\tau$ (as in 
			rule \hlref{sw}) 
			$\implies$ $\ws{\tau}{e^\acq_2}{e^\rel_1}$
			(by definition of \hlref{ws})
			
		\item [Case EF]: $e^\rel_1 \in \ordwrites{\moge\rel}_\tau$
			and $e^\acq_2 \in \ordfences{\moge\acq}_\tau$ (as in 
			rule \hlref{swEF}) 
			$\implies$ $\ws{\tau}{e^\acq_2}{e^\rel_1}$
			(by definition of \hlref{wsEF})
		
		\item [Case FE]: $e^\rel_1 \in \ordfences{\moge\rel}_\tau$
			and $e^\acq_2 \in \ordreads{\moge\acq}_\tau$ (as in 
			rule \hlref{swFE}) 
			$\implies$ $\ws{\tau}{e^\acq_2}{e^\rel_1}$
			(by definition of \hlref{wsFE})
		
		\item [Case FF]: $e^\rel_1 \in \ordfences{\moge\rel}_\tau$
			and $e^\acq_2 \in \ordfences{\moge\acq}_\tau$ (as in 
			rule \hlref{swFF}) 
			$\implies$ $\ws{\tau}{e^\acq_2}{e^\rel_1}$
			(by definition of \hlref{wsFF})
	\end{itemize}

	The $\setDOB$ relation can be formed (or be absent) 
	between the following combinations of type of $e^\rel_1$ 
	and $e^\acq_2$
	(covered in rules \hlref{dob} and \hlref{dobEF}):
	
	\begin{itemize}[label=Case EE,align=left,leftmargin=*]
		\item [Case EE]: $e^\rel_1 \in \ordwrites{\moge\rel}_\tau$
		and $e^\acq_2 \in \ordreads{\moge\acq}_\tau$ (as in 
		rule \hlref{dob}) 
		$\implies$ $\ws{\tau}{e^\acq_2}{e^\rel_1}$
		(by definition of \hlref{ws})
		
		\item [Case EF]: $e^\rel_1 \in \ordwrites{\moge\rel}_\tau$
		and $e^\acq_2 \in \ordfences{\moge\acq}_\tau$ (as in 
		rule \hlref{dobEF}) 
		$\implies$ $\ws{\tau}{e^\acq_2}{e^\rel_1}$
		(by definition of \hlref{wsEF})
	\end{itemize}

	Thus, every case of $\setSW$ and $\setDOB$ is covered to form
	\lws-rules. Therefore, the rules are exhaustive.
\end{proof}