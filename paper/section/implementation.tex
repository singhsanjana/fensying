\ourtechnique attempts to invalidate a buggy trace 
by inserting \cc fences in the input program $P$
by the technique discussed in Sections
\ref{sec:invalidating ce} and \ref{sec:methodology}.
%
The technique relies on a {\em counter example generator} 
to provide buggy traces (or counter examples) containing
the $\setMO$, $\setHB$ and $\setRF$ relations between the 
events of each buggy trace (line 3, 
Algorithm~\ref{alg:main algo}).
%
We recognized \cds \cite{cds} as the closest match to the
specification of a counter example generator
for the following reasons;

\begin{enumerate}
	\item \cds is a stateless model checker that returns traces 
		with assert violations (or buggy traces) if any.
	
	\item Unlike competing alternatives, \cds does not halt at the detection 
		of the first buggy trace instead it continues to provide all buggy 
		traces as required by \ourtechnique.
	
	\item \cds also provides $\setRF$ and $\setSB$ relations between the events 
		of the buggy trace. The information returned by \cds is sufficient
		to compute the $\setHB$ and $\setMO$ relations.
	
	\item Competing alternatives that return buggy traces for an input
		program, namely \genmc \cite{genmc-PLDI19} and \tracer \cite{tracer2018}, 
		do not support our requirements.
		%
		\genmc is designed for a variant of \cc \cite{LahavVafeiadis-PLDI17} and
		does not support an option for returning all buggy traces. \tracer is
		designed for a subset of \cc that has no support for \sc, \na or \rel 
		memory orders. \snj{What about Herd?}
\end{enumerate}

To bridge the gap between \cds's output and \ourtechnique's requirement,
we perform a pre-processing step that uses the $\setSB$ and $\setRF$ relations 
provided by \cds to compute the $\setHB$ and $\setMO$ relations on the events
of the buggy traces, based on the coherence rules provided by \cc model.
%
The buggy traces along with $\setSB$, $\setRF$ and computed $\setHB$ 
and $\setMO$ relations are provided as input to the core technique 
(lines 4-14, Algorithm~\ref{alg:main algo}) for synthesizing fences as
discussed in Section~\ref{sec:methodology}.

The core technique and the pre-processing has been implemented in 
{\tt Python}.
\ourtechnique uses {\em Johnson's} algorithm provided in {\em networkx}
library to detect cycles in \hlref{coherence conditions} and
\sc total order over 
events of the intermediate trace that includes the 
candidate fences (lines 7-8, Algorithm~\ref{alg:main algo}).
%
Upon finding cycles, \ourtechnique generates a SAT query 
representing the non-optimal set of candidate fences and
uses \z theorem prover to find the {\em min-model} of the 
query (line 11,13, Algorithm~\ref{alg:main algo}).

\noindent
{\bf Time efficient alternative technique:} It was observed that analysis
time of the core technique and \z computation of min-model were impacted
by the number of buggy traces returned by \cds.
%
As a counter measure we propose a time efficient alternative to the core 
technique that considers one buggy trace at a time.
%
This technique picks any one buggy trace of input program $P$,
fixes the same by inserting fences and  transforms $P$ to $P'$,
where $P'$ contains the newly added fences along with the events 
originally in $P$. 
%
The technique then looks for the next buggy trace by invoking \cds 
with $P'$ as the input program. 
The process repeats till \cds returns no buggy traces, or
finds a buggy trace that cannot be fixed with \cc fences.

\begin{wrapfigure}{l}{0.45\textwidth}
	\vspace{-2.5em}
	\begin{tabular}{|l|}
		\multicolumn{1}{r}{\scriptsize $e_i$: program events,
			$\mathbb{F}_i$: candidate fences} \\
		
		\hline
		cycles in $\tau_1$ (${\tt C}_{\tau_1}$)\\
		\tab $\mathbb{F}_1$ $R$ $\mathbb{F}_2$ $R$ $e_1$ $R$ $\mathbb{F}_1$\\
		\tab $\mathbb{F}_3$ $R$ $\mathbb{F}_4$ $R$ $\mathbb{F}_1$ $R$ $\mathbb{F}_3$\\
		$\formula{{\tt C}_{\tau_1}}$ = ($\mathbb{F}_1 \^ \mathbb{F}_2$) $\v$ ($\mathbb{F}_1 \^ \mathbb{F}_3 \^ \mathbb{F}_4$)\\
		\\
		cycles in $\tau_2$ (${\tt C}_{\tau_2}$)\\
		\tab $\mathbb{F}_3$ $R$ $\mathbb{F}_4$ $R$ $\mathbb{F}_3$\\
		$\formula{{\tt C}_{\tau_2}}$ = ($\mathbb{F}_3 \^ \mathbb{F}_4$)\\
		\hline
				
		\multicolumn{1}{c}{\footnotesize\hl{3-fence}}
	\end{tabular}
	\vspace{-3em}
\end{wrapfigure}

The technique is sound, \ie if the buggy traces can be stopped by 
inserting \cc fences in the program then the technique will successfully
invalidate those traces;
but the technique is not optimal in the number of fences inserted.
The non-optimality of the efficient version is a result of the reduction
in information that the technique has to work with (from all buggy traces
to just one at a time). As a consequence of considering one trace per
iteration the \z formula is attenuated with terms only from one trace.
%
Consider the example in Figure~\hlref{3-fence} 
depicting cycles in two buggy traces $\tau_1$ and $\tau_2$ of an 
input program. The primary \ourtechnique 
technique would provide the formula $\formula{{\tt C}_{\tau_2}}$ $\^$ 
$\formula{{\tt C}_{\tau_2}}$ to \z and thus recognize ($\mathbb{F}_1 \^ 
\mathbb{F}_3 \^ \mathbb{F}_4$) as the optimal solution.
The time efficient version, however, may consider the formula 
$\formula{{\tt C}_{\tau_1}}$ in the first iteration and 
$\formula{{\tt C}_{\tau_2}}$ in the next iteration and return
($\mathbb{F}_1 \^ \mathbb{F}_2$) $\^$ ($\mathbb{F}_3 \^ \mathbb{F}_4$) 
as the result which is not optimal in the number of fences.

However, it was observed from the experimental results that the time
efficient technique significantly outperforms the primary technique in terms of 
the analysis time while adding a reasonably small number of extra
fences. The experimental results have been discussed in the 
following section.