\ourtechnique attempts to invalidate a buggy trace (\aka counter
example) by one of the following two strategies. \svs{why two separate strategies
are required (one for sc and one for non-sc)? You must motivate this first.}:
\begin{itemize}[label=strategy1,align=left,leftmargin=*]
	\item [Strategy1]:
		Invalidate by violating \sc-total-order.
	\item [Strategy2]:
		Invalidate by inter-thread synchronization.
\end{itemize}

\svs{I don't like the word invalidating traces when semantically you are
suggesting that these invalidated traces are actually fixed traces. Suggest
some alternatives. }
\snj{Invalidated traces are not fixed traces, they are no-traces. The program is 
fixed the trace has been removed from the set of traces.}
\noindent
{\bf Strategy1: Invalidate by violating \sc-total-order. 
	(\stfence)}\newline
Consider a buggy input program $P$.
%
As discussed in Section~\ref{sec:c11}, in a valid trace of $P$ 
(including a buggy trace), 
the \sc ordered events must form a total order ($\setTO$).
%
Contrarily, if we synthesize \sc ordered fences in the input program 
such that the total order requirement in the buggy trace gets violated 
then we can invalidate the trace and stop the program behavior.

We introduce an irreflexive and possibly cyclic relation
on \sc ordered events of invalidated version $\inv{\tau}$ of $\tau$ 
called \sc-order ($\setSO$).
%
The $\setSO$ relation is directly implied from
the \lto-rules, \hlref{toHb}, \hlref{toMo} \hlref{toFr} 
and \hlref{toRf} (Section~\ref{sec:c11}) wherever applicable
%
and extended to relaxed events with appropriately placed fences 
as described in \cc \cite{C11}\cite{Batty-POPL12}.

\begin{definition}{\bf \sc-order ($\setSO$)}\newline
	$\forall e_1, e_2 \in \events_\tau$ \st 
	$(e_1,e_2) \in$ $\setHB$ $\union$ $\setMO$ $\union$ $\setRF$ 
	$\union$ $\setFR$
	
	if
	$e_1, e_2 \in \ordevents{\sc}_\tau$ then 
	$\so{\inv{\tau}}{e_1}{e_2}$;
	
	if
	$e_1 \in \ordevents{\sc}_\tau$, 
	$\exists \mathbb{F}^\sc \in \ordfences{\sc}_{\inv{\tau}}$ where
	$\seqb{\inv{\tau}}{e_2}{\mathbb{F}^\sc}$ then
	$\so{\inv{\tau}}{e_1}{\mathbb{F}^\sc}$;
	
	if
	$e_2 \in \ordevents{\sc}_\tau$, 
	$\exists \mathbb{F}^\sc \in \ordfences{\sc}_{\inv{\tau}}$ where
	$\seqb{\inv{\tau}}{\mathbb{F}^\sc}{e_1}$ then
	$\so{\inv{\tau}}{\mathbb{F}^\sc}{e_2}$;
	
	if
	$\exists \mathbb{F}^\sc_1$, $\mathbb{F}^\sc_2$ 
	$\in \ordfences{\sc}_{\inv{\tau}}$ where
	$\seqb{\inv{\tau}}{\mathbb{F}^\sc_1}{e_1}$ and 
	$\seqb{\inv{\tau}}{e_2}{\mathbb{F}^\sc_2}$ then
	$\so{\inv{\tau}}{\mathbb{F}^\sc_1}{\mathbb{F}^\sc_1}$.
\end{definition}

Consider a \cc trace $\tau$ and a transformation
$\inv{\tau}$ of $\tau$ \st $\events_{\inv{\tau}}$ = $\events_\tau$
$\union$ set of synthesized fences, then;
%
\begin{theorem} \label{thm:to-so}
	$\to{\inv{\tau}}{}{}$ = (transitive closure of $\setSO$).
	\snj{proof in appendix.}
\end{theorem}
%
Using Theorem~\ref{thm:to-so} we can state that,
a cyclic $\setSO$ implies that there does not exist a 
total order on \sc ordered events of $\inv{\tau}$. 
The trace $\inv{\tau}$
then is not a valid \cc trace and the buggy trace $\tau$ 
has been invalidated.
%
Thus, the aim of \stfence is to synthesis \sc-fences in the input program $P$
at appropriate locations that would force a cyclic $\setSO$ 
 in the \sc ordered events of a transformed trace $\inv{\tau}$. 
The cycle invalidates the trace for the transformed program $\fx{P}$.

\setlength{\textfloatsep}{0pt}
\begin{wrapfigure}{l}{0.5\textwidth}
	\vspace{-2.5em}
	\begin{tabular}{|c|}
		\multicolumn{1}{c}{Initially $Flag_1 = 0, Flag_2 = 0$} \\
		\hline
		\resizebox{0.45\textwidth}{!}{\tikzset{every picture/.style={line width=0.75pt}} %set default line width to 0.75pt        
\begin{tikzpicture}[x=1em,y=1em,yscale=-1,xscale=-1]
\tikzstyle{every node}=[font=\normalfont]
\node (fl1) [inner sep=2pt] {$ Flag_1 :=_\sc 1 $};
\node (fl2) [right=35pt of fl1, inner sep=2pt] {$Flag_2 :=_\sc 1$};
\node (rfl2) [below=52pt of fl1, inner sep=2pt] {if $\neg Flag_2$:};
\node (rfl1) [below=52pt of fl2, inner sep=2pt] {if $\neg Flag_1$:};
\node (cs11) [below right=-1pt and -40pt of rfl2] {(critical};
\node (cs12) [below right=-4pt and -40pt of cs11] {section1)};
\node (cs21) [below right=-1pt and -40pt of rfl1] {(critical};
\node (cs22) [below right=-4pt and -40pt of cs21] {section2)};
\node (t) [below right=0pt and -30pt of cs12] {(i) {\tt Buggy trace ($\tau$)}};
%
\draw [->,>=stealth,color=CarnationPink] (fl1) -- node[midway,right=-2pt,font=\scriptsize,color=black] { $\lsb$ } (rfl2);
\draw [->,>=stealth,color=CarnationPink] (fl2) -- node[midway,left=-2pt,font=\scriptsize,color=black] { $\lsb$ } (rfl1);

\end{tikzpicture}
} \\
		\hline
		\resizebox{0.49\textwidth}{!}{\tikzset{every picture/.style={line width=0.75pt}} %set default line width to 0.75pt        
\begin{tikzpicture}[x=1em,y=1em,yscale=-1,xscale=-1]
\tikzstyle{every node}=[font=\normalfont]
\node (fl1) [inner sep=2pt] {$ Flag_1 :=_\sc 1 $};
\node (fl2) [right=35pt of fl1, inner sep=2pt] {$Flag_2 :=_\sc 1$};
\node (f1) [below=20pt of fl1, inner sep=2pt] {$\mathbb{F}^\sc_1$};
\node (f2) [below=20pt of fl2, inner sep=2pt] {$\mathbb{F}^\sc_2$};
\node (rfl2) [below=20pt of f1, inner sep=2pt] {if $\neg Flag_2$:};
\node (rfl1) [below=20pt of f2, inner sep=2pt] {if $\neg Flag_1$:};
\node (cs11) [below right=-1pt and -40pt of rfl2] {(critical};
\node (cs12) [below right=-4pt and -40pt of cs11] {section1)};
\node (cs21) [below right=-1pt and -40pt of rfl1] {(critical};
\node (cs22) [below right=-4pt and -40pt of cs21] {section2)};
\node (t) [below right=0pt and -35pt of cs12] {(ii) {\tt Invalidated ($\inv{\tau}$)}};
%
\draw [->,>=stealth,color=Mahogany] (fl1) -- node[midway,right=-2pt,font=\scriptsize,color=black] { $\lsb$ } node[pos=.25,left=-2pt,font=\scriptsize,color=black] { $\lso$ } node[pos=.65,left=-2pt,font=\scriptsize,color=black] { \hlref{sohb} } (f1);
\draw [->,>=stealth,color=Mahogany] (fl2) -- node[midway,left=-2pt,font=\scriptsize,color=black] { $\lsb$ } node[pos=.25,right=-2pt,font=\scriptsize,color=black] { $\lso$ } node[pos=.65,right=-2pt,font=\scriptsize,color=black] { \hlref{sohb} } (f2);
\draw [->,>=stealth,color=Mahogany] (f1) -- node[midway,right=-2pt,font=\scriptsize,color=black] { $\lsb$ } node[pos=.25,left=-2pt,font=\scriptsize,color=black] { $\lso$ } node[pos=.65,left=-2pt,font=\scriptsize,color=black] { \hlref{sohb} } (rfl2);
\draw [->,>=stealth,color=Mahogany] (f2) -- node[midway,left=-2pt,font=\scriptsize,color=black] { $\lsb$ } node[pos=.25,right=-2pt,font=\scriptsize,color=black] { $\lso$ } node[pos=.65,right=-2pt,font=\scriptsize,color=black] { \hlref{sohb} } (rfl1);
\draw [->,>=stealth,color=Mahogany] (rfl1) -- node[pos=.25,sloped,above=-2pt,font=\scriptsize,color=black] { $\lfr$ $\lso$ } node[pos=.25,sloped,below=-2pt,font=\scriptsize,color=black] { \hlref{soFF} } (fl1);
\draw [->,>=stealth,color=Mahogany] (rfl2) -- node[pos=.25,sloped,above=-2pt,font=\scriptsize,color=black] { $\lfr$ $\lso$ } node[pos=.25,sloped,below=-2pt,font=\scriptsize,color=black] { \hlref{soFF} } (fl2);

\end{tikzpicture}
} \\
		\hline
		\multicolumn{1}{c}{\hl{mutex}}
	\end{tabular}
	\vspace{-2.5em}
\end{wrapfigure}

Consider the example \hlref{mutex}. If the reads of $Flag_1$ and $Flag_2$ 
read the initial value $0$ (as shown in \hlref{mutex}(i)) 
then the program violates mutual exclusion
property. Note that, the relation $\setFR$ is not a \cc relation, it has
been introduced in this work to assist with forming $\setSO$ relation.
In the absence of $\setFR$, the trace $\tau$ does not violate any coherence
rule and is possible under \cc.
%
The buggy trace can be invalidated by introducing \sc-fences $\mathbb{F}^\sc_1$
and $\mathbb{F}^\sc_2$ as shown in \hlref{mutex}(ii). 
$\setSO$ edges between the program events 
and synthesized fences to form a cycle
shown in \hlref{mutex}(ii).
The cycle in $\setSO$ relation indicates that a valid 
total-order cannot be formed on \sc-events of $\inv{\tau}$ and thus the 
trace cannot be produced under \cc.\newline

\noindent
{\bf Strategy2: Invalidate by inter-thread synchronization.
(\wkfence)}\newline
In the second strategy we attempt to introduce synchronization
between threads by synthesizing fences. The fences form
$\setHB$ relation with program events \st coherence
requirement
$\neg(\hb{\inv{\tau}}{e_1}{e_2}$ $\^$ $\fr{\inv{\tau}}{e_2}{e_1})$ is
violated in the invalidated trace $\inv{\tau}$ of $\tau$.
%
Note that, this strategy invalidates a buggy trace by introducing 
weaker fences (of memory orders \rel, \acq and \acqrel) and 
is preferred over Strategy1 wherever applicable.

Two threads synchronize over \rel and \acq (or stricter) ordered
events when an appropriate $\setRF$ is formed. Such synchronization
form $\setSW$ ({\it synchronizes-with}) and $\setDOB$ 
({\it dependency-ordered-before}) relations between program events,
\cite{batty2011mathematizing}\cite{C11}
where $\setSW \subseteq \setITHB$ and $\setDOB \subseteq \setITHB$.
%
The $\setSW$ and $\setDOB$ relations can similarly be formed between
fences based on appropriate $\setRF$.

\begin{definition}{\bf synchronizes-with $\setSW$}\newline
		$\forall e_1, e_2 \in \events_\tau$ \st $\rf{\tau}{e_1}{e_2}$
		
		if
		$e_1 \in \ordwrites{\moge\rel}_\tau$ and
		$e_2 \in \ordreads{\moge\acq}_\tau$ then 
		$\sw{\inv{\tau}}{e_1}{e_2}$;
		
		if
		$e_1 \in \ordwrites{\moge\rel}_\tau$, 
		$\exists \mathbb{F}^\acq \in \ordfences{\moge\acq}_{\inv{\tau}}$ where
		$\seqb{\inv{\tau}}{e_2}{\mathbb{F}^\acq}$ then
		$\sw{\inv{\tau}}{e_1}{\mathbb{F}^\acq}$;
		
		if
		$e_2 \in \ordreads{\moge\acq}_\tau$, 
		$\exists \mathbb{F}^\rel \in \ordfences{\moge\rel}_{\inv{\tau}}$ where
		$\seqb{\inv{\tau}}{\mathbb{F}^\rel}{e_1}$ then
		$\sw{\inv{\tau}}{\mathbb{F}^\rel}{e_2}$;
		
		if
		$\exists \mathbb{F}^\rel \in \ordfences{\moge\rel}_{\inv{\tau}}$,
		$\mathbb{F}^\acq \in \ordfences{\moge\acq}_{\inv{\tau}}$ where
		$\seqb{\inv{\tau}}{\mathbb{F}^\rel}{e_1}$,
		$\seqb{\inv{\tau}}{e_2}{\mathbb{F}^\acq}$ 
		
		then
		$\sw{\inv{\tau}}{\mathbb{F}^\rel}{\mathbb{F}^\acq}$.
\end{definition}

\begin{definition}{dependency-ordered-before $\setDOB$}\newline
	$\forall e_1, e_2 \in \events_\tau$, 
	$e_1' \in \ordwrites{\moge\rel}_\tau$ \st $\rf{\tau}{e_1}{e_2}$
	and $e_1 \in$ {\em release-sequence}($e_1'$)\cite{C11}
	
	if $e_2 \in \ordreads{\moge\acq}_\tau$ then 
	$\dob{\inv{\tau}}{e_1'}{e_2}$;
	
	if $\exists \mathbb{F}^\acq \in \ordfences{\moge\acq}_{\inv{\tau}}$ 
	where $\seqb{\inv{\tau}}{e_2}{\mathbb{F}^\acq}$ then
	$\dob{\inv{\tau}}{e_1'}{\mathbb{F}^\acq}$.
\end{definition}

\noindent
The sets $\moge\rel$ = $\{\rel,\acqrel,\sc\}$ and
$\moge\acq$ = $\{\acq,\acqrel,\sc\}$.

\setlength{\textfloatsep}{0pt}
\begin{wrapfigure}{l}{0.6\textwidth}
	\vspace{-2.5em}
	\begin{tabular}{|c|c|}
		\multicolumn{2}{c}{Initially $x = 0, y = 0$} \\
		
		\hline
		\resizebox{0.29\textwidth}{!}{\tikzset{every picture/.style={line width=0.75pt}} %set default line width to 0.75pt        
\begin{tikzpicture}[x=1em,y=1em,yscale=-1,xscale=-1]
\tikzstyle{every node}=[font=\normalfont]
\node (wy) [inner sep=2pt] {$W^\rlx(y,1)$};
\node (rx) [right=10pt of wy, inner sep=2pt] {$R^\rlx(x,1)$};
\node (wx) [below=52pt of wy, inner sep=2pt] {$W^\rlx(x,1)$};
\node (ry) [below=52pt of rx, inner sep=2pt] {$R^\rlx(y,0)$};
\node (t) [below right=1pt and -45pt of wx] {(i) {\tt Buggy trace ($\tau$)}};
%
\draw [->,>=stealth,color=CarnationPink] (wy) -- node[midway,right=-2pt,font=\small,color=black] { $\lsb$ } (wx);
\draw [->,>=stealth,color=CarnationPink] (rx) -- node[midway,left=-2pt,font=\small,color=black] { $\lsb$ } (ry);
\draw [->,>=stealth,color=PineGreen] (wx) -- node[pos=.75,left,font=\small,color=black] { $\lrf$ } (rx);

\end{tikzpicture}
} &
		\resizebox{0.29\textwidth}{!}{\tikzset{every picture/.style={line width=0.75pt}} %set default line width to 0.75pt        
\begin{tikzpicture}[x=1em,y=1em,yscale=-1,xscale=-1]
\tikzstyle{every node}=[font=\normalfont]
\node (wy) [inner sep=2pt] {$W^\rlx(y,1)$};
\node (rx) [right=10pt of wy, inner sep=2pt] {$R^\rlx(x,1)$};
\node (f1) [below=20pt of wy, inner sep=2pt] {$\mathbb{F}^\rel_1$};
\node (f2) [below=20pt of rx, inner sep=2pt] {$\mathbb{F}^\acq_2$};
\node (wx) [below=20pt of f1, inner sep=2pt] {$W^\rlx(x,1)$};
\node (ry) [below=20pt of f2, inner sep=2pt] {$R^\rlx(y,0)$};
\node (t) [below right=-1pt and -42pt of wx] {(ii) {\tt Invalid ($\inv{\tau}$)}};
%
\draw [->,>=stealth,color=Mahogany] (wy) -- node[midway,left=-2pt,font=\small,color=black] { $\lsb$ } (f1);
\draw [->,>=stealth,color=Mahogany] (rx) -- node[midway,right=-2pt,font=\small,color=black] { $\lsb$ } (f2);
\draw [->,>=stealth,color=Mahogany] (f1) -- node[midway,left=-2pt,font=\small,color=black] { $\lsb$ } (wx);
\draw [->,>=stealth,color=Mahogany] (f2) -- node[midway,right=-2pt,font=\small,color=black] { $\lsb$ } (ry);
\draw [->,>=stealth,color=PineGreen,opacity=0.5] (wx) -- node[midway,left=-2pt,font=\small,color=black] { $\lrf$ } (rx);
\draw [->,>=stealth,color=Magenta] (f1.north east) to[out=-165,in=-15] node[midway,above=5pt,font=\small,color=black] { $\lsw$ } node[midway,above=-1pt,font=\small,color=black] {\tt \textcolor{Sepia}{(sw-dobFF)} } (f2.north west);
\draw [->,>=stealth,color=Mahogany] (f2.south west) to[out=15,in=165] node[midway,below=0pt,font=\small,color=black] { $\lws$ } node[midway,below=5pt,font=\small,color=black] {\tt \textcolor{Sepia}{(wsFF)} } (f1.south east);

%(wr1.south east)+(.3,-5pt) $) to[out=135,in=-135]

\end{tikzpicture}
} \\
		\hline

		\multicolumn{2}{r}{\scriptsize 
		$W/R^m(o,v)$: $m$ ordered write/read of object $o$ and value $v$} \\
		\multicolumn{2}{c}{\hl{sync}}
	\end{tabular}
	\vspace{-2.5em}
\end{wrapfigure}

\noindent
The synchronizations further form transitive $\setITHB$ relations
(as described in Section~\ref{sec:c11})
making previously available writes now unavailable to read events.
%
Consider the example \hlref{sync}. As all memory accesses in the
buggy trace $\tau$ are relaxed, 
$\rf{\tau}{W^\rlx(x,1)}{R^\rlx(x,1)}$ does not form a synchronization
between the respective threads. As a result, 
$\nithb{\tau}{W^\rlx(y,1)}{R^\rlx(y,0)}$ and the trace $\tau$ is valid
under \cc. 
%
The buggy trace can be invalidated by introducing weaker fences
$\mathbb{F}^\rel_1$ and $\mathbb{F}^\acq_2$ as shown in \hlref{sync}(ii).
The fences, related as 
$\sw{\inv{\tau}}{\mathbb{F}^\rel_1}{\mathbb{F}^\acq_2}$, form
$\ithb{\inv{\tau}}{W(y,1)}{Ry(y,0)}$, which violates the coherence of 
$\inv{\tau}$ as $\ithb{\inv{\tau}}{W(y,1)}{Ry(y,0)}$ $\^$ 
$\fr{\inv{\tau}}{Ry(y,0)}{W(y,1)}$. 

To successfully synthesize fences $\mathbb{F}^\rel_1$ and 
$\mathbb{F}^\acq_2$ in \hlref{sync} and similar cases our
technique must recognize candidate $\setSW$ or $\setDOB$
related fence(s) that would form the necessary $\setITHB$ 
and invalidate a buggy trace. In other words, we need to 
recognize the program locations where a desired 
synchronization was absent. 
%
To do so we introduce $\setWS$ relation
that essentially captures the absence of a synchronization.

\begin{definition}{synchronization-absence $\setWS$}\newline
	$\forall e_1,e_2 \in  \events_\tau$ \st
	$\fr{\tau}{e_2}{e_1}$,
	%
	if $\exists e_1' \in$ $\ordwrites{\moge\rel}_{\inv{\tau}}$ 
	$\union$ $\ordfences{\moge\rel}_{\inv{\tau}}$ and
	$e_2' \in$ $\ordreads{\moge\acq}_{\inv{\tau}}$ $\union$ 
	$\ordfences{\moge\acq}_{\inv{\tau}}$ where 
	$\seqb{\inv{\tau}}{e_1}{e_1'}$ and 
	$\seqb{\inv{\tau}}{e_2'}{e_2}$
	then
	$\ws{\inv{\tau}}{e_2'}{e_1'}$.
\end{definition}

\noindent
The definition of $\setWS$ has been formed with the following rationale,
	$\exists$ $\fr{\tau}{e_2}{e_1}$ $\implies$
	$\nithb{\tau}{e_1}{e_2}$ $\implies$
	$\neg(\sw{\inv{\tau}}{e_1'}{e_2'} \v \dob{\inv{\tau}}{e_1'}{e_2'})$ 
	$\implies$ $\ws{\inv{\tau}}{e_2'}{e_1'}$.

\begin{theorem} \label{thm:ws exhaustive}
	\lws-rules exhaustively capture absence of synchronization.
	\snj{proof in appendix.}
\end{theorem}

To summarize, Strategy2 builds on two sets of events:
(i) candidate events that can form a synchronization 
(captured with $\sw{\inv{\tau}}{}{}$ and 
$\dob{\inv{\tau}}{}{}$); and
(ii) candidate events that are originally missing a desired
synchronization (captured with $\setWS$).
If the two sets intersect we get the program locations
for synthesizing fences that would prevent the buggy trace.
We recognize the intersection by a cycle in  
$\setWS$ $\union$ $\hb{\inv{\tau}}{}{}$ 
(that contains $\sw{\inv{\tau}}{}{}$ and $\dob{\inv{\tau}}{}{}$), 
such as the cycle between $\mathbb{F}^\rel_1$ and 
$\mathbb{F}^\acq_2$ in the example \hlref{sync}.

Note that, combining \stfence and \wkfence for synthesizing only
\sc ordered fences offers solution that is optimal in the number
of fences synthesized. 
%
The \wkfence strategy provides opportunity to synthesize weaker
fences offering optimality in the number and type of fences.
%
In this technique if we recognize a $\hb{\inv{\tau}}{}{}$ $\union$ 
$\ws{\inv{\tau}}{}{}$ cycle in $\events_{\inv{\tau}}$ then we
synthesize appropriate weaker fences at the program locations  
of synthesized fences.
