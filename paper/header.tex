\usepackage{xspace}
\usepackage[dvipsnames]{xcolor}
\usepackage{amsmath}
\usepackage{afterpage}  
\usepackage{figlatex,wrapfig}
\usepackage[dvipsnames]{xcolor}
\usepackage{listings,amssymb,mathtools}
\usepackage{mathrsfs}
\usepackage{array,multirow}
\usepackage{caption}
\usepackage{longtable}
\usepackage{algorithm}
\usepackage[noend]{algpseudocode}
\usepackage{framed,enumitem}
\usepackage{wrapfig}

% for colours
\usepackage{xspace}
\usepackage[colorlinks]{hyperref}
\hypersetup{
	colorlinks = true,
	citecolor = {Magenta},
	linkcolor = {blue},
	urlcolor  = {blue}
}

% for arrow diagrams
\usepackage{amsmath}
\usepackage{amssymb}
\usepackage{smartdiagram}
\usepackage{tikz}
\usetikzlibrary{arrows,positioning}

%\usepackage[parfill]{parskip}

% for bib handline
%\usepackage[numbers]{natbib}
%\usepackage{url}

% commonly used abbreviations and expressions
\renewcommand{\th}{^{th}\xspace} % superscript th for numbers eg i^{th}
\newcommand{\definedas}{\triangleq\xspace}
\newcommand{\ie}{{\em i.e.}\xspace}
\newcommand{\st}{\ \mbox{s.t.}\ }
\newcommand{\viz}{\textit{viz}.\@\xspace}
\newcommand{\wrt}{\textit{wrt}\xspace}
\newcommand{\wkt}{we know that,\xspace}
\newcommand{\aka}{a.k.a\xspace}
\newcommand{\sota}{state-of-the-art\xspace}
\newcommand{\Sota}{State-of-the-art\xspace}

% common operators
\renewcommand{\^}{\xspace\wedge\xspace}
\renewcommand{\v}{\xspace\vee\xspace}
\newcommand{\xor}{\xspace\veebar\xspace}
\renewcommand{\|}{\ |\ }
\newcommand{\intersection}{\xspace\cap\xspace}
\newcommand{\union}{\xspace\cup\xspace}
\newcommand{\intersectioneq}{\xspace\cap=\xspace}
\newcommand{\unioneq}{\ {\cup}{=}\ }
\newcommand{\nin}{\not\in\xspace}

% highlighted hyperlinks
\newcommand{\hl}[1]{{\textcolor{darkgray}{\texttt{(#1)}}}\xspace} % hyperlink target
\newcommand{\hlref}[1]{\hyperlink{#1}{\textcolor{Sepia}{\small \texttt{(#1)}}}}
\newcommand{\tab}{\quad\quad}

%%%%%%%%%%%%%%%%%%%%%%%%% Document Specific %%%%%%%%%%%%%%%%%%%%%%

% tools and techniques
\newcommand{\ourtechniquename}{FenSying}
\newcommand{\ourtechnique}{\textcolor{RubineRed}{\texttt{\ourtechniquename}}\xspace}
\newcommand{\ourtool}{\ourtechnique{-}tool\xspace}
\newcommand{\cc}{\textit{C11}\xspace}
\newcommand{\cds}{CDSChecker\xspace}
\newcommand{\genmc}{GenMC\xspace}
\newcommand{\tracer}{Tracer\xspace}
\newcommand{\z}{\texttt{Z3}\xspace}
\newcommand{\sfence}{\textcolor{black}{Strong-\ourtechniquename}}
\newcommand{\wfence}{\textcolor{black}{Weak-\ourtechniquename}}

% sets and entitites
\newcommand{\program}{$P$\xspace} %input program
\newcommand{\programhat}{$\widehat{P}$\xspace} %transformed/fixed program
\newcommand{\inv}[1]{{#1}^{\mathtt{inv}}\xspace}
\newcommand{\imm}[1]{{#1}^{\mathtt{imm}}\xspace}
\newcommand{\fx}[1]{{#1}^{\mathtt{fx}}\xspace}
\newcommand{\formula}[1]{\mathscr{F}(#1)\xspace}
\newcommand{\threads}{\mathcal{T}\xspace}
\newcommand{\states}{\Sigma\xspace}
\newcommand{\moset}{\mathcal{M}\xspace}
\newcommand{\actions}{\mathcal{A}\xspace}
\newcommand{\objects}{\mathcal{O}\xspace}
\newcommand{\s}[1]{s_{[#1]}\xspace} % state reached after exploring sequence #1
% events' sets aux
\newcommand{\wt}[1]{{\mathbb{W}#1}}
\newcommand{\rd}[1]{{\mathbb{R}#1}}
\newcommand{\fn}[1]{{\mathbb{F}#1}}
% events' sets
\newcommand{\events}{\mathcal{E}\xspace}
\newcommand{\writes}{\events^\wt{}\xspace}
\newcommand{\reads}{\events^\rd{}\xspace}
\newcommand{\fences}{\events^\fn{}\xspace}
\newcommand{\ordevents}[1]{\events^{(#1)}\xspace}
\newcommand{\ordwrites}[1]{\events^\wt{(#1)}\xspace}
\newcommand{\ordreads}[1]{\events^\rd{(#1)}\xspace}
\newcommand{\ordfences}[1]{\events^\fn{(#1)}\xspace}
% cycles
\newcommand{\scycles}[1]{\texttt{Scycles}(#1)\xspace}
\newcommand{\rcycles}[1]{\texttt{Rcycles}(#1)\xspace}
\newcommand{\cycle}[1]{\texttt{cycles}(#1)\xspace}

% memory orders
\newcommand{\mosc}{\texttt{seq\_cst}\xspace}
\newcommand{\moar}{\texttt{acq\_rel}\xspace}
\newcommand{\morel}{\texttt{release}\xspace}
\newcommand{\moacq}{\texttt{acquire}\xspace}
\newcommand{\mocon}{\texttt{consume}\xspace}
\newcommand{\morlx}{\texttt{relaxed}\xspace}

% operators
\newcommand{\molt}{{\sqsubset}\xspace}
\newcommand{\mole}{{\sqsubseteq}\xspace}
\newcommand{\mogt}{{\sqsupset}\xspace}
\newcommand{\moge}{{\sqsupseteq}\xspace}

% relations
\newcommand{\reln}[4]{#3 {\rightarrow^{#1}_{#2}} #4\xspace} % any relation specified as #1
\newcommand{\nreln}[4]{#3 \nrightarrow^{#1}_{#2} #4\xspace} % not of any relation specified as #1

% relation with events
\newcommand{\seqb}[3]{\reln{\textbf{\textcolor{CarnationPink}{sb}}}{#1}{#2}{#3}\xspace}
\newcommand{\rf}[3]{\reln{\textbf{\textcolor{PineGreen}{rf}}}{#1}{#2}{#3}\xspace} 
\newcommand{\dob}[3]{\reln{\textbf{\textcolor{Mulberry}{dob}}}{#1}{#2}{#3}\xspace}
\newcommand{\sw}[3]{\reln{\textbf{\textcolor{Magenta}{sw}}}{#1}{#2}{#3}\xspace}
\newcommand{\ithb}[3]{\reln{\textbf{\textcolor{NavyBlue}{ithb}}}{#1}{#2}{#3}\xspace}
\newcommand{\hb}[3]{\reln{\textbf{\textcolor{Cerulean}{hb}}}{#1}{#2}{#3}\xspace}
\newcommand{\nhb}[3]{\nreln{\textbf{\textcolor{Cerulean}{hb}}}{#1}{#2}{#3}\xspace}
\newcommand{\mo}[3]{\reln{\textbf{\textcolor{RedOrange}{mo}}}{#1}{#2}{#3}\xspace}
\newcommand{\nmo}[3]{\nreln{\textbf{\textcolor{RedOrange}{mo}}}{#1}{#2}{#3}\xspace}
\renewcommand{\to}[3]{\reln{\textbf{\textcolor{Brown}{to}}}{#1}{#2}{#3}\xspace}
\newcommand{\so}[3]{\reln{\textbf{\textcolor{Mahogany}{so}}}{#1}{#2}{#3}\xspace} %sc
\newcommand{\fr}[3]{\reln{\textbf{\textcolor{RoyalPurple}{fr}}}{#1}{#2}{#3}} %rb/fr
\newcommand{\nfr}[3]{\nreln{\textbf{\textcolor{RoyalPurple}{fr}}}{#1}{#2}{#3}} %~rb/fr
\newcommand{\ws}[3]{\reln{\textbf{\textcolor{Mahogany}{ws}}}{#1}{#2}{#3}} %sc-sw
\newcommand{\nithb}[3]{\nreln{\textbf{\textcolor{NavyBlue}{ithb}}}{#1}{#2}{#3}\xspace}

% relation name without events
\newcommand{\setSB}{\seqb{\tau}{}{}\xspace}
\newcommand{\setRF}{\rf{\tau}{}{}\xspace}
\newcommand{\setSW}{\sw{\tau}{}{}\xspace}
\newcommand{\setDOB}{\dob{\tau}{}{}\xspace}
\newcommand{\setITHB}{\ithb{\tau}{}{}\xspace}
\newcommand{\setHB}{\hb{\tau}{}{}\xspace}
\newcommand{\setMO}{\mo{\tau}{}{}\xspace}
\newcommand{\setTO}{\to{\tau}{}{}\xspace}
\newcommand{\setSO}{\so{\tau}{}{}\xspace}
\newcommand{\setWS}{\ws{\tau}{}{}\xspace}
\newcommand{\nsetHB}{\nhb{\tau}{}{}\xspace}
\newcommand{\nsetMO}{\nmo{\tau}{}{}\xspace}
\newcommand{\setFR}{\fr{\tau}{}{}\xspace}
\newcommand{\nsetFR}{\nfr{\tau}{}{}\xspace}

% relation label 
\newcommand{\lsb}{\textbf{\textcolor{CarnationPink}{sb}}\xspace}
\newcommand{\lrf}{\textbf{\textcolor{PineGreen}{rf}}\xspace} 
\newcommand{\ldob}{\textbf{\textcolor{Mulberry}{dob}}\xspace}
\newcommand{\lsw}{\textbf{\textcolor{Magenta}{sw}}\xspace}
\newcommand{\lithb}{\textbf{\textcolor{NavyBlue}{ithb}}\xspace}
\newcommand{\lhb}{\textbf{\textcolor{Cerulean}{hb}}\xspace}
\newcommand{\lmo}{\textbf{\textcolor{RedOrange}{mo}}\xspace}
\newcommand{\lto}{\textbf{\textcolor{Brown}{to}}\xspace}
\newcommand{\lso}{\textbf{\textcolor{Mahogany}{so}}\xspace}
\newcommand{\lws}{\textbf{\textcolor{Mahogany}{ws}}\xspace}
\newcommand{\lfr}{\textbf{\textcolor{RoyalPurple}{fr}}\xspace}

\newcommand{\var}[1]{\color{OliveGreen}\texttt{#1}\color{black}\xspace}
\newcommand{\fun}[2]{\color{Sepia}\texttt{#1(\color{Gray}\textit{#2}\color{Sepia})}\color{black}\xspace}
\newcommand{\class}[1]{\color{DarkOrchid}\texttt{#1}\color{black}\xspace}

% memory orders
\newcommand{\na}{\texttt{na}\xspace}
\newcommand{\rlx}{\texttt{rlx}\xspace}
\newcommand{\rel}{\texttt{rel}\xspace}
\newcommand{\acq}{\texttt{acq}\xspace}
\newcommand{\acqrel}{\texttt{acq-rel}\xspace}
\renewcommand{\sc}{\texttt{sc}\xspace}
\newcommand{\onsc}[1]{#1|_\sc \xspace}

% load/store events and instructions
\newcommand{\load}[3]{#1 := #2_{#3}}
\newcommand{\store}[3]{#1 := #2_{#3}}
\newcommand{\loadev}[3]{\mathtt{R^{#3}({#1},{#2})}}
\newcommand{\storeev}[3]{\mathtt{W^{#3}({#1},{#2})}}
\newcommand{\fenceev}[1]{\textcolor{Brown}{--\fn{#1}--}}
% tikz edges
\newcommand{\rfedge}[3]{\draw [->,>=stealth,color=PineGreen,thin] ({#1}) -- node[{#3}] { rf} ({#2});}
\newcommand{\moedge}[3]{\draw [->,>=stealth,color=RedOrange,thin] ({#1}) -- node[{#3}] {mo} ({#2});}

\newcommand{\cycles}[1]{\mathcal{C}_{#1}}


%snj: Have to use the ones in format
%\newtheorem{theorem}{Theorem}[section]
%\newtheorem{corollary}{Corollary}[theorem]
%\newtheorem{lemma}[theorem]{Lemma}

\newcommand{\ishComment}[1]{\textit{\color{red}\tiny{#1}}}
\newcommand{\divComment}[1]{\textcolor{ForestGreen}{[div: #1]}}
\newcommand{\snj}[1]{\textcolor{RubineRed}{[snj]: #1}}
\newcommand{\svs}[1]{\textcolor{Maroon}{[svs]:#1}}

%svs -- added to track changes -- for the benefit of snj,div,ish!  use
% the [final] option to clear the changes and show the last changes
% only.
\usepackage[commentmarkup=todo,highlightmarkup=background]{changes}