\documentclass[dvipsnames]{article}
\setlength{\parindent}{0pt}

\usepackage{amsmath}
\usepackage{amssymb}
\usepackage{smartdiagram}
\usepackage{tikz}
\usetikzlibrary{arrows,positioning}

\usepackage[colorlinks]{hyperref}
\hypersetup{
	colorlinks = true,
	citecolor = {violet},
	linkcolor = {blue},
	urlcolor  = {MidnightBlue}
}

\newcommand{\var}[1]{\color{OliveGreen} \texttt{#1}\color{black}}
\newcommand{\fun}[2]{\color{Sepia}\texttt{#1(\color{Gray}\textit{#2}\color{Sepia})}\color{black}}
\newcommand{\varinfo}[1]{\scriptsize \texttt{#1} \normalsize}
\newcommand{\class}[1]{\color{DarkOrchid}\texttt{#1}\color{black}}

\newcommand{\rf}[2]{{#1} {\color{Blue}\rightarrow$^{rf}$} \color{black}{#2}}
\newcommand{\mo}[2]{{#1} {\color{Blue}\rightarrow$^{mo}$} \color{black}{#2}}
\newcommand{\tor}[2]{{#1} {\color{OliveGreen}\rightarrow$^{to}$} \color{black}{#2}}

\date{}
\begin{document}

\title{to.py}
\maketitle

Total order (TO) is a binary relation on some set, which is antisymmetric, transitive, and a connex relation. Formally, a binary relation $\leq$  is a total order on a set X if the following statements hold for all a, b and c in X:
\begin{enumerate}
    \item \textbf{Antisymmetry}\\
        if a $\leq$ b and b $\leq$ a, then a = b
    \item \textbf{Transitivity}\\
        if a $\leq$ b and b $\leq$ c then a $\leq$ c
    \item \textbf{Connexity}\\
        a $\leq$ b or b $\leq$ a
\end{enumerate}

For C/C++11, these rules translate to the following:
%\tikzset{
%    %Define standard arrow tip
%    >=stealth',
%    %Define style for boxes
%    punkt/.style={
%           rectangle,
%           minimum height=2em,
%           text centered},
%}

\begin{enumerate}
\setcounter{enumi}{-1}
\item
	For a given variable, if there is a \textit{read-from} relation between a 
	read and a write instruction, then all \mosc fences \textit{sequenced-before} 
	the write instruction are in \setSC with the all \mosc fences \textit{sequenced-after} 
	the read instruction.\\
	\begin{tikzpicture}[baseline = (current bounding box.north)]
    \node[punkt] (x) {X:$F_{sc}$};
    \node[punkt,below=of x] (a) {A:$Wx$};
    \node[punkt,right=of x] (b) {B:$Rx$};
    \node[punkt, below=of b] (y) {Y:$F_{sc}$};
  	\draw [->] (a) -> (b) node[midway,above] {rf};
  	\draw [->] (x) -> (a) node[midway,left] {sb};
  	\draw [->] (b) -> (y) node[midway,right] {sb};
    \end{tikzpicture}\\
    \rf{\tau}{A}{B} $\implies$ \to{\tau}{X}{Y}

\item
	\begin{enumerate}[a.]
	\item
		For a given variable, for an \mosc read instruction \textit{B} reading 
		from an \mosc write instruction \textit{A}, there is a \setSC relation 
		between the read and the write instructions.\\
		$\rf{\tau}{A:W_{sc}}{B:R_{sc}}$\\
		$\to{\tau}{A}{B}$
	
	\item
		For a given variable, for an \mosc read instruction \textit{B} reading 
		from a write instruction \textit{M}, the read instruction is in \setSC 
		with all \mosc write instructions which \textit{happen-after} \textit{M}.\\
		A:$W_{sc}x$, M:$Wx$, B:$R_{sc}x$\\
		$\hb{\tau}{M}{A}$ $\land$ $\rf{\tau}{M}{B}$ $\implies$ $\to{\tau}{B}{A}$
	\end{enumerate}

\item
	For a given variable, all fences \textit{sequenced-before} a read instruction 
	\textit{B} reading from a write instruction \textit{$M_1$} are in \setSC with 
	all \mosc write instructions modifying the variable after \textit{$M_1$}.\\
	\begin{tikzpicture}[baseline = (current bounding box.north)]
	\node[punkt](x) {X:$F_{sc}$};
	\node[punkt, below=of x](b) {B:$Rx$};
	\node[punkt, right=of b](m1) {M$_1:Wx$};
	\node[punkt, below=of m1](m2) {M$_2:W_{sc}$};
  	\draw [->] (x) -> (b) node[midway,left] {sb};
  	\draw [->] (m1) -> (b) node[midway,above] {rf};
  	\draw [->] (m1) -> (m2) node[midway,right] {mo};
  	\end{tikzpicture}\\
  	\too{X}{M$_2$}
 
\item
	For a given variable, if a read instruction \textit{B} \textit{reads-from} 
	a write instruction \textit{M}, and \textit{M} has \setMO
	relations with write instruction \textit{A}, then all the \mosc fences 
	\textit{sequenced-after} \textit{A} are in \setSC with all the \mosc 
	fences \textit{sequenced-before} \textit{B}.\\
	\begin{tikzpicture}[baseline = (current bounding box.north)]
	\node[punkt](a) {A:$Wx$};
	\node[punkt, below=of a](x) {X:$F_{sc}$};
	\node[punkt, right=of a](y) {Y:$F_{sc}$};
	\node[punkt, below=of y](b) {B:$Rx$};
	\node[punkt, right=of y](m) {M:$Wx$};
  	\draw [->] (a) -> (x) node[midway,left] {sb};
  	\draw [->] (y) -> (b) node[midway,right] {sb};
	\end{tikzpicture}\\
	\rf{M}{B} $\land$ \mo{M}{A} $\implies$ \too{Y}{X}

\item
	Given two write instructions \textit{A} and \textit{B} for some variable, 
	if there is a \setMO relation between the two, \setSC relations will be formed.\\
	\begin{tikzpicture}[baseline = (current bounding box.north)]
	\node[punkt](a) {A:$Wx$};
	\node[punkt, below=of a](x) {X:$F_{sc}$};
	\node[punkt, right=of a](y) {Y:$F_{sc}$};
	\node[punkt, below=of y](b) {B:$Wx$};
  	\draw [->] (a) -> (x) node[midway,left] {sb};
  	\draw [->] (y) -> (b) node[midway,right] {sb};
	\end{tikzpicture}\\
	\begin{enumerate}[a.]
	\item if B:$W_{sc}x$\\
		\mo{\tau}{B}{A} $\implies$ \to{\tau}{B}{X}
	
	\item if A:$W_{sc}x$\\
		\mo{\tau}{B}{A} $\implies$ \to{\tau}{Y}{A}
	
	\item
		\mo{\tau}{B}{A} $\implies$ \to{\tau}{Y}{X}
	\end{enumerate}

\item 
%	\setSC edges caused by \setSB:
%	\begin{enumerate}[a.]
%		\item 
		Any two \mosc instructions that are in \setSB are also in \setSC \\
			\seqb{\tau}{A}{B} $\implies$ \to{\tau}{A}{B} where A and B are $(R/W)_{sc}x$
%		
%		\item 
%		\too{A}{B} $ \land $ \seqb{B}{C} $\implies$ \too{A}{C}
%		
%		\item
%		\seqb{A}{B} $ \land $ \too{B}{C} $\implies$ \too{A}{C}
%	\end{enumerate}
	
\end{enumerate}

\section{Finding TO edges between fences}
Each rule has a function that runs according to it and determines TO relations between fences. After immediate relations are calculated, transitive TO relations between the fences are also calculated in function \fun{all\_to}{}.

\end{document}